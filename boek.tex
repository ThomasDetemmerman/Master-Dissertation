%
%  THESISBOEK
%
%  Dit bestand zorgt voor algemene (layout)definities, en groepeert de
%  afzonderlijke LaTeX-files tot een geheel.
%
%  @author Erwin Six, David De Reu, Brecht Vermeulen
%

\documentclass[11pt,a4paper,oneside,notitlepage]{book}
\usepackage[english,dutch]{babel}

% marges aanpassen
% (opmerking: moet *voor* inclusie van fancyhdr package komen)
\setlength{\hoffset}{-1in}
\setlength{\voffset}{-1in}
\setlength{\topmargin}{2cm}
\setlength{\headheight}{0.5cm}
\setlength{\headsep}{1cm}
\setlength{\oddsidemargin}{3.5cm}
\setlength{\evensidemargin}{3.5cm}
\setlength{\textwidth}{16cm}
\setlength{\textheight}{23.3cm}
\setlength{\footskip}{1.5cm}

\usepackage{fancyhdr}
\usepackage{graphicx}
% \usepackage[colorlinks]{hyperref}

\pagestyle{fancy}

\renewcommand{\chaptermark}[1]{\markright{\MakeUppercase{#1}}}
\renewcommand{\sectionmark}[1]{\markright{\thesection~#1}}

\newcommand{\headerfmt}[1]{\textsl{\textsf{#1}}}
\newcommand{\headerfmtpage}[1]{\textsf{#1}}

\fancyhf{}
\fancyhead[LE,RO]{\headerfmtpage{\thepage}}
\fancyhead[LO]{\headerfmt{\rightmark}}
\fancyhead[RE]{\headerfmt{\leftmark}}
\renewcommand{\headrulewidth}{0.5pt}
\renewcommand{\footrulewidth}{0pt}

\fancypagestyle{plain}{ % eerste bladzijde van een hoofdstuk
  \fancyhf{}
  \fancyhead[LE,RO]{\headerfmtpage{\thepage}}
  \fancyhead[LO]{\headerfmt{\rightmark}}
  \fancyhead[RE]{\headerfmt{\leftmark}}
  \renewcommand{\headrulewidth}{0.5pt}
  \renewcommand{\footrulewidth}{0pt}
}

% anderhalve interlinie (opm: titelblad gaat uit van 1.5)
\renewcommand{\baselinestretch}{1.5}

% indien LaTeX niet goed splitst, neem je woord hierin op, of evt om splitsen 
% te voorkomen
\hyphenation{ditmagnooitgesplitstworden dit-woord-splitst-hier}

\begin{document}

% titelblad (voor kaft)
%  Titelblad

% Opmerking: gaat uit van een \baselinestretch waarde van 1.5 (die moet
% ingesteld worden voor het begin van de document environment)

\begin{titlepage}

\setlength{\hoffset}{-1in}
\setlength{\voffset}{-1in}
\setlength{\topmargin}{1.5cm}
\setlength{\headheight}{0.5cm}
\setlength{\headsep}{1cm}
\setlength{\oddsidemargin}{3cm}
\setlength{\evensidemargin}{3cm}
\setlength{\footskip}{1.5cm}
\enlargethispage{1cm}
% \textwidth en \textheight hier aanpassen blijkt niet te werken

\fontsize{12pt}{14pt}
\selectfont

\begin{center}

\includegraphics[height=2cm]{fig/ruglogo}

\vspace{0.5cm}

Faculteit Toegepaste Wetenschappen\\
Vakgroep Informatietechnologie\\
Voorzitter: Prof.~Dr.~Ir.~P.~LAGASSE

\vspace{3.5cm}

\fontseries{bx}
\fontsize{17.28pt}{21pt}
\selectfont

Ontwerp van een performante\\
gedistribueerde CORBA--monitor

\fontseries{m}
\fontsize{12pt}{14pt}
\selectfont

\vspace{.6cm}

door 

\vspace{.4cm}

David DE REU

\vspace{3.5cm}

Promotor: Prof.~Dr.~Ir.~P.~DEMEESTER\\
Scriptiebegeleider: Ir.~B.~VERMEULEN\\

\vspace{2cm}

Scriptie ingediend tot het behalen van de academische graad van\\
burgerlijk ingenieur in de computerwetenschappen

\vspace{1cm}

Academiejaar 2001--2002

\end{center}
\end{titlepage}


% lege pagina (!!)

% titelblad (!!)

% geen paginanummering tot we aan de inhoudsopgave komen
\pagestyle{empty}

% voorwoord met dankwoord en toelating tot bruikleen (ondertekend)
%  Voorwoord (dankwoord) en toelating tot bruikleen

\newpage

\noindent \textbf{\huge Voorwoord}

\vspace{1.5cm}

\noindent
Hier komt wat tekst.

\addvspace{4cm}

\noindent David De Reu, mei 2002\newpage

\noindent \textbf{\huge Toelating tot bruikleen}

\vspace{1.5cm}

\noindent
``De auteur geeft de toelating deze scriptie voor consultatie beschikbaar
te stellen en delen van de scriptie te kopi\"eren voor persoonlijk
gebruik.\\
Elk ander gebruik valt onder de beperkingen van het auteursrecht,
in het bijzonder met betrekking tot de verplichting de bron uitdrukkelijk
te vermelden bij het aanhalen van resultaten uit deze scriptie.''

\addvspace{4cm}

\noindent David De Reu, mei 2002


% overzicht
%  Overzichtsbladzijde met samenvatting

\newpage

{
\setlength{\baselineskip}{14pt}
\setlength{\parindent}{0pt}
\setlength{\parskip}{8pt}

\begin{center}

\noindent \textbf{\huge
Ontwerp van een performante\\[8pt]
gedistribueerde CORBA--monitor
}

door 

David DE REU

Scriptie ingediend tot het behalen van de academische graad van\\
burgerlijk ingenieur in de computerwetenschappen

Academiejaar 2001--2002

Promotor: Prof.~Dr.~Ir.~P.~DEMEESTER\\
Scriptiebegeleider: Ir.~B.~VERMEULEN

Faculteit Toegepaste Wetenschappen\\
Universiteit Gent

Vakgroep Informatietechnologie\\
Voorzitter: Prof.~Dr.~Ir.~P.~LAGASSE

\end{center}

\section*{Samenvatting}

% TODO: samenvatting

Hier komt de samenvatting.


\section*{Trefwoorden}

% TODO: trefwoorden

Hier komen trefwoorden.

}

\newpage % strikt noodzakelijk om een header op deze blz. te vermijden


\pagestyle{fancy}
\frontmatter

% inhoudstafel
\tableofcontents

% opmaak voor het eigenlijke boek; onderstaande lijnen
% weglaten als de eerste regel van een nieuwe alinea moet
% inspringen in plaats van extra tussenruimte
%\setlength{\parindent}{0pt}
%\setlength{\parskip}{0.5\baselineskip plus 0.5ex minus 0.2ex}
%\setlength{\parskip}{1ex plus 0.5ex minus 0.2ex}

% hoofdstukken
\mainmatter

% hier worden de hoofdstukken ingevoegd (\includes)
\chapter{Dit is een hoofdstuk}

\section{Sectie}

Een sectie.

\section{Nog een sectie}
\label{sec:nogeensectie}

Nog een.

\subsection{Een subsectie}

\begin{figure}[htb]
\begin{center}
%optioneel om evt wat witruimte van je figuur af te snoepen
%\vspace{-.3cm}
 \includegraphics[keepaspectratio,width=0.5\textwidth]{fig/ruglogo}
% analoog
%\vspace{-0.6cm}
 \caption{Het RUG logo}
 \label{fig:ruglogo}
%analoog
%\vspace{-.6cm}
\end{center}
\end{figure}

En wat tekst natuurlijk.

\section{En nu de nuttige dingen}

Wist je dat omniORB\cite{omniorb} nergens in een RFC, zoals in \cite{2-BIT}, beschreven staat ? Om meer referenties\cite{omniorb, 2-BIT} te showen. Bovendien zorgt $\backslash$ gevolgd door een spatie ervoor dat een spatie na
een punt niet als begin van een nieuwe zin gezien wordt, zoals bv.\ hier (normaal zou je dit bv. hebben). Bij het begin van een zin laat LateX iets meer ruimte. Een tilde daarentegen
zorgt ervoor dat een spatie nooit opgebroken wordt in een nieuwe lijn, te gebruiken bv.\ in Figuur~\ref{fig:ruglogo} zodat het figuurnummer niet
gesplitst wordt van ``Figuur''.

Meteen is het invoegen van figuren (inclusief caption en labelen) ook gedemonstreerd. En men kan ook secties labelen om er naar te verwijzen: bv.\ in
sectie~\ref{sec:nogeensectie} staat er eigenlijk niks.

Ook zeer mooi is het gebruik van ldots om 3 puntjes --- o ja, voor ik het vergeet, meerdere ``-'' na elkaar geven langere horizontale strepen, bv.\ 1--2 --- op een
deftige manier te zetten,~\ldots\ Die laatste $\backslash$ is enkel nodig omdat het het einde van de zin is, en LaTeX anders daar geen witruimte
invoegt. Zie je de handigheid van LaTeX al (of wordt het net te ingewikkeld) ?

In verband met splitsen, ditmagnooitgesplitstworden zal nooit gesplitst worden. Ditwoordsplitsthier daarentegen wel. ditmagnooitgesplitstworden
ditmagnooitgesplitstworden ditmagnooitgesplitstworden ditmagnooitgesplitstworden (zie hier waarom je soms zal *moeten* aanduiden waar je een woord kan
splitsen. Of je krijgt een Overfull hbox (78.19391pt too wide) in paragraph at lines 43--45. Dit kan met hyphenation in boek.tex.








% appendices
\appendix

% hier worden de appendices ingevoegd (\includes)


%dit moet nog binnen dutch komen, anders staat er Bibliography
\begin{thebibliography}{99}

%referenties zijn typisch in het Engels
\selectlanguage{english}
 
\bibitem{leos2000} K. Steenbergen, F. Janssen, J. Wellen, R. Smets, T.
Koonen,
\newblock ``Fast wavelength-and-time slot routing in hybrid fiber-access
networks for IP-based services'',
\newblock in {\em IEEE LEOS Symposium}, Delft, The Netherlands,
October~2000.
 
\bibitem{2-BIT} K. Nichols, V. Jacobson, L. Zhang,
\newblock ``A two-bit differentiated services architecture for the
Internet'',
\newblock {\em IETF RFC~2638},
July~1999.                                       

\bibitem{omniorb} http://www.omniorb.org

\end{thebibliography}

\selectlanguage{dutch}

\backmatter

% eventueel: lijst van figuren en tabellen
%\listoffigures
%\listoftables

% lege pagina (!!)

% kaft

\end{document}
