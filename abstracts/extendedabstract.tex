%%%%%%%%%%%%%%%%%%%%%%%%%%  phdsymp_sample2e.tex %%%%%%%%%%%%%%%%%%%%%%%%%%%%%%
%% changes for phdsymp.cls marked with !PN
%% except all occ. of phdsymp.sty changed phdsymp.cls
%%%%%%%%%%                                                       %%%%%%%%%%%%%
%%%%%%%%%%    More information: see the header of phdsymp.cls   %%%%%%%%%%%%%
%%%%%%%%%%                                                       %%%%%%%%%%%%%
%%%%%%%%%%%%%%%%%%%%%%%%%%%%%%%%%%%%%%%%%%%%%%%%%%%%%%%%%%%%%%%%%%%%%%%%%%%%%%%


%\documentclass[10pt]{phdsymp} %!PN
\documentclass[twocolumn]{phdsymp} %!PN
%\documentclass[12pt,draft]{phdsymp} %!PN
%\documentstyle[twocolumn]{phdsymp}
%\documentstyle[12pt,twoside,draft]{phdsymp}
%\documentstyle[9pt,twocolumn,technote,twoside]{phdsymp}

\usepackage[english]{babel}       % Voor nederlandstalige hyphenatie (woordsplitsing)

\usepackage{graphicx}                   % Om figuren te kunnen verwerken
\usepackage{graphics}			% Om figuren te verwerken.
\graphicspath{{figuren/}}               % De plaats waar latex zijn figuren gaat halen.

\usepackage{times}

\hyphenation{si-mu-la-ted re-a-lis-tic packets really in-clu-ding}

\def\BibTeX{{\rm B\kern-.05em{\sc i\kern-.025em b}\kern-.08em
    T\kern-.1667em\lower.7ex\hbox{E}\kern-.125emX}}

\newtheorem{theorem}{Theorem}

\begin{document}

\title{Evaluatie van de totale electromagnetische blootstelling van de mens in een netwerk van drones}

\author{Thomas Detemmerman}

\supervisor{Wout Joseph, Luc Martens, Luc Martens, German Dario Castellanos Tache}

\maketitle

\begin{abstract}
TODO: abstract in Dutch
\end{abstract}

\begin{keywords}
LTE, electromagnetische blootstelling, energieverbruik, drone, femtocell, microstrip patch antenna, radiation pattern
\end{keywords}

\section{Introductie}
\PARstart{T}{he} 
Introduction in Dutch

\section{Section}

\subsection{Gerelateerd werk}
TODO

\subsection{Scenario's}
todo 

\subsection{Electromagnetische blootstelling}
todo

\section{Resultaten}
todo

\section{Conclusie}
todo

\subsection{Referencies}
todo


\nocite{*}
\bibliographystyle{phdsymp}
%%%%%\bibliography{bib-file}  % commented if *.bbl file included, as
%%%%%see below


%%%%%%%%%%%%%%%%% BIBLIOGRAPHY IN THE LaTeX file !!!!! %%%%%%%%%%%%%%%%%%%%%%%%
%% This is nothing else than the phdsymp_sample2e.bbl file that you would%%
%% obtain with BibTeX: you do not need to send around the *.bbl file        
%%
%%---------------------------------------------------------------------------%%
%
\begin{thebibliography}{1}
\bibitem{paper}
Bart Lannoo, Didier Colle, Mario Pickavet, Piet Demeester,
\newblock {\em Optical Switching Architecture to Implement Moveable Cells in a Multimedia Train Environment},
\newblock Proc. of ECOC 2004, 30th European Conf. on Optical Communication, vol. 3, pp. 344-345, Stockholm, Sweden, 5-9 Sep. 2004.

\bibitem{ns-click}
Michael Neufeld, Ashish Jain, Dirk Grunwald,
\newblock {\em Nsclick:: bridging network simulation and deployment},
\newblock http://systems.cs.colorado.edu/Networking/nsclick/

\bibitem{click}
\newblock {\em The Click Modular Router Project},
\newblock http://www.read.cs.ucla.edu/click/

\bibitem{ns}
\newblock {\em {NS} -- {N}etwork {S}imulator},
\newblock http://nsnam.isi.edu/nsnam/

\end{thebibliography}
%
%%---------------------------------------------------------------------------%%

\end{document}

%%%%%%%%%%%%%%%%%%%%%  End of phdsymp_sample2e.tex  %%%%%%%%%%%%%%%%%%%%%%%%%%%
