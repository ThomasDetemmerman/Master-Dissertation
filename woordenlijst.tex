


%-------------------------------- acroniemen

\newacronym{ICNIRP}{ICNIRP}{International Commission on Non-Ionizing Radiation Protection}
\newacronym{UAV}{UAV}{Unmanned Arial Vehicle}
\newacronym{UABS}{UABS}{Unmanned Arial Base Station}
\newacronym{EIRP}{EIRP}{Equivalent Isotropical Radiation Power}
\newacronym{UE}{UE}{User Equipment}
\newacronym{IEC}{IEC}{International Electrotechnical Commission}
\newacronym{SAR}{SAR}{Specific Absorption Rate}
\newacronym{whipp}{WHIPP}{WiCa Heuristic Indoor Propagation Prediction}
\newacronym{whipp}{WHIPP}{WiCa Heuristic Indoor Propagation Prediction}
\newacronym{DL}{DL}{downlink}
\newacronym{UL}{UL}{uplink}
\newacronym{LTE}{LTE}{Long-Term Evolution}
\newacronym{FDD}{FDD}{frequency division duplex}
\newacronym{TDD}{TDD}{time division duplex}
\newacronym{SAR}{SAR}{specific absorption rate}
\newacronym{ICNIRP}{ICNIRP}{International Commission on Non-Ionizing Radiation Protection}
%--------------------------------- woordenlijst
\newglossaryentry{isotropicradiator}{
	name = equivalent isotropic radiator,
	text = equivalent isotropic radiator,
	description = A theoretical source of electromagnetic waves which radiates the same intensity for all directions.
}

\newglossaryentry{spuriousradiation}{
	name = spurious radiation,
	text = spurious radiation,
	description = according to the thefreedictionary.com: Any emission from a radio transmitter at frequencies outside its frequency band. Also known as spurious emission.
}

\newglossaryentry{RRP}{
	name = RRP,
	text = RRP,
	description = RRP is an abreviation used in this paper to indicate an extension on EIRP and stands for Real Radiation Pattern. An RRP value indicates the power (in dBm) for a certain location unlike an EIRP where the power (in dBm) is independent of the location.
}


%\printglossary[type=\acronymtype,title={Lijst van acroniemen}]
%\addcontentsline{toc}{chapter}{\textcolor{maincolor}{Lijst van acroniemen}}
%\printglossary
%\addcontentsline{toc}{chapter}{\textcolor{maincolor}{Verklarende woordenlijst}}




