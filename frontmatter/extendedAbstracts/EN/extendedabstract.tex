%%%%%%%%%%%%%%%%%%%%%%%%%%  phdsymp_sample2e.tex %%%%%%%%%%%%%%%%%%%%%%%%%%%%%%
%% changes for phdsymp.cls marked with !PN
%% except all occ. of phdsymp.sty changed phdsymp.cls
%%%%%%%%%%                                                       %%%%%%%%%%%%%
%%%%%%%%%%    More information: see the header of phdsymp.cls   %%%%%%%%%%%%%
%%%%%%%%%%                                                       %%%%%%%%%%%%%
%%%%%%%%%%%%%%%%%%%%%%%%%%%%%%%%%%%%%%%%%%%%%%%%%%%%%%%%%%%%%%%%%%%%%%%%%%%%%%%


%\documentclass[10pt]{phdsymp} %!PN
\documentclass[twocolumn]{phdsymp} %!PN
%\documentclass[12pt,draft]{phdsymp} %!PN
%\documentstyle[twocolumn]{phdsymp}
%\documentstyle[12pt,twoside,draft]{phdsymp}
%\documentstyle[9pt,twocolumn,technote,twoside]{phdsymp}

\usepackage[english]{babel}       % Voor nederlandstalige hyphenatie (woordsplitsing)

\usepackage{graphicx}			% Om figuren te verwerken.
\graphicspath{{../../../images/}}
\usepackage{booktabs}
\usepackage{times}
\usepackage{siunitx}
\usepackage{xcolor}
\usepackage{amsmath}
\PassOptionsToPackage{hyphens}{url}
\usepackage{hyperref}
\usepackage{url}

\usepackage[acronym,toc,shortcuts]{glossaries}
\setglossarystyle{super}
\renewcommand{\glsnamefont}[1]{\textbf{#1}}
\makeglossaries




%-------------------------------- acroniemen
\newacronym{UABS}{UABS}{Unmanned Arial Base Station}
\newacronym{EIRP}{EIRP}{equivalent isotropic radiation power}
\newacronym{UE}{UE}{User Equipment}
\newacronym{IEC}{IEC}{International Electrotechnical Commission}
\newacronym{SAR}{SAR}{Specific Absorption Rate}
\newacronym{whipp}{WHIPP}{WiCa Heuristic Indoor Propagation Prediction}
\newacronym{DL}{DL}{downlink}
\newacronym{UL}{UL}{uplink}
\newacronym{LTE}{LTE}{Long-Term Evolution}
\newacronym{FDD}{FDD}{Frequency Division Duplex}
\newacronym{TDD}{TDD}{Time Division Duplex}
\newacronym{ICNIRP}{ICNIRP}{International Commission on Non-Ionizing Radiation Protection}
\newacronym{LOS}{LOS}{line of sigh}
\newacronym{NLOS}{NLOS}{non line of sigh}
\newacronym{FCC}{FCC}{Federal Communications Commission}
\newacronym{Exp Opt}{Exp Opt}{exposure optimized network}
\newacronym{PwrC Opt}{PwrC Opt}{power consuption optimized network}
\newacronym{WHO}{WHO}{World Health Organization}
\newacronym{USA}{USA}{United States of America}
\newacronym{IOT}{IoT}{Internet of Things}
\newacronym{UAV}{UAV}{Unmanned Aerial Vehicle}
\newacronym{EU}{EU}{European Union}
%--------------------------------- woordenlijst
\newglossaryentry{isotropicradiator}{
	name = equivalent isotropic radiator,
	text = equivalent isotropic radiator,
	description = A theoretical source of electromagnetic waves which radiates the same intensity for all directions
}

\newglossaryentry{spuriousradiation}{
	name = spurious radiation,
	text = spurious radiation,
	description = According to the thefreedictionary.com: Any emission from a radio transmitter at frequencies outside its frequency band. Also known as spurious emission
}

\newglossaryentry{RRP}{
	name = RRP,
	text = RRP,
	description = RRP is an abreviation used in this paper to indicate an extension on EIRP and stands for Real Radiation Pattern. An RRP value indicates the power (in dBm) for a certain location unlike an EIRP where the power (in dBm) is independent of the location
}

\newglossaryentry{power flux density}{
	name = power flux density,
	text = power flux density,
	description = Magnitude of power ($W$) that travels through a curtain area ($m^2$)
}

\newglossaryentry{thermoregulatory capacity}{
	name = thermoregulatory capacity,
	text = thermoregulatory capacity,
	description = The capacity of an organism to regulate body temperture
}

\newglossaryentry{exact algorithm}{
	name = exact algorithm,
	text = exact algorithm,
	description = An exact algorithm solves an optimization problem optimally
}






\hyphenation{si-mu-la-ted re-a-lis-tic packets really in-clu-ding}

\def\BibTeX{{\rm B\kern-.05em{\sc i\kern-.025em b}\kern-.08em
    T\kern-.1667em\lower.7ex\hbox{E}\kern-.125emX}}

\newtheorem{theorem}{Theorem}

\begin{document}
\title{Evaluating Human Electromagnetic Exposure in a \gls{UAV}-aided Network}

\author{Thomas Detemmerman}

\supervisor{Prof. dr. ir. Wout Joseph, Prof. dr. ir. Luc Martens} %Dr. ir. Margot Deruyck, Mphil. German Dario Castellanos Tache

\maketitle

\begin{abstract}
Society relies more than ever on the availability of the wireless networks but is at the same time also 
concerned about the potential health effects of the electromagnetic radiation caused by these networks.
The government has enforced strict legislations to which mobile devices and base stations have to satisfy.

This research investigates the specific absorption rate caused by these electromagnetic waves by taking all mobile devices and base stations into account.
To accomplish this goal, the deployment tool developed by the WAVES research group at Ghent University will be used. This tool simulates an entire network 
where transmission towers are represented by femtocell base stations attached to drones. This research also investigates how these drones can be guided 
in order to reach certain goals like minimizing power consumption or electromagnetic exposure.

It looks from the results that ... (todo)
\end{abstract}

\begin{keywords}
LTE, Electromagnetic Radiation, power consumption, drones, femtocell, microstrip patch antenna, radiation pattern, specific absorption rate (SAR)
\end{keywords}

\section{Introduction}
\PARstart{S}{ociety} is constantly getting more and more dependent on wireless communication. 
On any given moment, in any given location, an electronic device
can request to connect to the bigger network. Devices need more than ever to be connected, 
starting from small \gls{IOT} up to self-driving cars
which all need to be supported by the existing infrastructure. 

Also in exceptional and possibly life threatening situations, the public relies on the cellular network. 
One solution for a fast temporarily deployable network is with the usage of a \gls{UAV}. A base station can be attached to 
these flying \gls{UAV}s to support the damaged network over a limited area. 
This approach is also useful in case of an unexpected increase in traffic. 
For example during the terrorist attacks at Brussels Airport,
mobile network operators saw all telecommunications drastically increasing causing moments of contention. 
Some operators even decided to temporarily exceed the exposure limits in
order to handle all connections \cite{baseZaventem}.
Electromagnetic exposure can however not be neglected. 
Research shows how excessive electromagnetic radiation can cause diverse biological side effects \cite{bioeffects, WHO}.
It becomes clear that electromagnetic exposure is a key value when designing a \gls{UAV}-aided network and should definitely 
not surpass the limits predefined by the government.

\gls{UAV}-aided networks can, thanks to their mobility, easily be repositioned towards a certain goal. Several papers 
explain how a network can be optimized towards different goals like power consumption.
However, very limited
research has been done where a \gls{UAV}-aided network is optimized towards electromagnetic exposure.
While several publications exist, discussing how the electromagnetic exposure can be calculated. 
Most of them only consider a limited number of sources like only base stations or only mobile phones.
Papers who cover electromagnetic exposure from all the different sources and convert it into a single value are rather limited.

This research proposes a method to optimize the network towards electromagnetic exposure and power consumption
when considering all four sources of radiation in a telecommunications network, being: the user's own phone,
 the base station that is serving this user, 
all devices from other users in the network and all 
other active base stations that are not serving this user. In this way, the contribution of each source towards the total 
electromagnetic exposure can easily be identified. 

The behaviour of the electromagnetic exposure and power consumption of the network will be analysed
 by applying the tool in different scenarios like different types of antennae and various flying height and population 
densities.
Values like \gls{SAR}, electromagnetic exposure and power consumption will 
 give insight in how the network behaves so the network could be optimized accordingly.

To make this research possible, 
an existing capacity based deployment tool developed by the WAVES research group at Ghent University is used.
This planning tool describes a fully configured \gls{UAV}-network which is a suitable starting point for this research.


\section{State of the Art}
\subsection{Electromagnetic exposure}

Users in a telecommunication network are exposed to various sources of electromagnetic radiation, expressed in $V/m$. Once this exposure is absorbed by the human 
body we are talking about specific absorption rate (SAR) and is expressed in $W/kg$. All these values are 
subjected to limitations enforced by the government. This research is done in Ghent, 
a Flemish city in Belgium, where in the 2.6 GHz frequency band, an individual antenna cannot exceed 4.5 V/m and the cumulative sum of all 
fixed sources has its maximum at 31 V/m \cite{J23, S13_normenBelgie}. The maximum SAR-values for the whole body 
over a 10 g tissue ($SAR_{10g}$) is $2 W/kg$ \cite{J30}. 

Several papers calculate exposure originating from certain sources  \cite{J6_originalExposureFormula, J1, J10_RDP, J10.1} 
where some convert the \gls{UL} radiation into localized specific absorption rates \cite{ J10_RDP, J10.1}. 
With the advent of 5G, paper \cite{J17_kuehn2019modelling} describes 
how the localized \gls{SAR}-values are achieved from different sources.
Finally, \cite{J22_plets2015joint} describes how both \gls{UL} and \gls{DL} traffic can be converted into a single whole body SAR value.

In a realistic network, some users are calling while others are using other types of telecommunication services like browsing the web.
Therefore, all absorbed electromagnetic exposure should be expressed in whole body SAR while still covering all sources.

\subsection{Optimized \gls{UAV}-aided networks}

A \gls{UAV} knows several applications. It was originally mainly used to support the military for surveillance and remote attacks without 
endangering pilots \cite{U12}. However, \gls{UAV}s have recently become more accessible by the general public due to decreasing costs. This 
allowed \gls{UAV}s to be researched for various applications.

A \gls{UAV} equipped with a femtocell base station antenna will be called a \gls{UABS}
which brings several advantages like mobility and rapid deployment. 
However, it brings also challenges like limited weight of the payload and sparse power supply.

Kawamoto et al. introduced in \cite{U11} a WiFi network with the support of  \gls{UAV}s while considering resource allocation 
and antenna directivity. 
Gangula et al. illustrates in \cite{U10} how \gls{UAV}s can be used as a relay for LTE. 
Zeng et al. proposes in  \cite{U12} a tutorial in 5G-and-beyond wireless systems where challenges like 
energy consumption, mobility and antenna direction are discussed. 
Deruyck et al. designed in \cite{J2} a capacity based deployment tool for UAV-aided emergency
networks for large-scale disaster scenarios where an ideal flying height of 100 m is suggested. This was expanded 
in \cite{U1} with a performance evaluation of the direct-link backhaul of this tool where a slightly lower 
flying height of 80 metres is recommended.


Mozaffari et al. provides in \cite{U3} guidelines on how to optimize and analyse \gls{UAV}s equipped for 
wireless communication equipment.
One path that has been excessively researched are location optimization solutions where the network is 
designed in such a way that certain goals like minimal power consumption are achieved \cite{U6,U7,U8,U9}.
These optimizations can be achieved through different implementation methods like exact algorithms or machine learning \cite{U3,U5}.

Research where network is optimized towards electromagnetic exposure is rather limited.
Deruyck et al. discusses in \cite{J1} how a terrestrial network can be optimized towards either a minimal exposure or minimal power consumption of the entire network.
However, to the best of the author knowledge, no research has been done where a \gls{UABS}-network has been optimized towards electromagnetic exposure.

\subsection{Technologies}

For the deployment of the network, the more robust \gls{UAV}s from \cite{J2} will be used and will be operating in the 2.6 Ghz 
bandwidth. Since the users are assumed to experience a constant electromagnetic exposure without interruptions, frequency division duplex is used.

% problem antennae on drones
The onboard antenna of the \gls{UAV} will act as the gateway between the UE and the backhaul network.
However, determining which antenna to use and how to position it, can be challenging.
The radiation pattern from the antenna can be influenced by the \gls{UAV} \cite{A1}.
Also the fact that the \gls{UAV} will hover above the user makes tradional 2D modelling insufficient.
A 3D-model which accounts for both elevation and azimuth directivity 
will be required \cite{U12}.

The easiest radiation pattern is a hypothetical isotropic radiator which radiates equally in all directions.
Antennae that radiate equal quantities for a certain plane are called omnidirectional antennae \cite{U12} and several types 
have been investigated for \gls{UAV} usage like monopoles, dipoles and wing antennae 
have been considered \cite{A4,A10,A11,A12}.
Another type of antennae are directional antennae. One type 
that has excessively been researched in various array-configurations are microstrip patch antennae \cite{A5,A6,A8}.
They provide several advantages compared to traditional antennae \cite{J13_microstripadvantages, J14_antennadesign}
like lightweightness, low in cost and thin causing them to be more aerodynamic. 

A basic microstrip antenna consists of a ground plane and
a radiating patch, both separated with a dielectric substrate. 
Several variations exist like microstrip patch antenna, microstrip slot antenna and printed dipole antenna which
all have similar characteristics \cite{J13_microstripadvantages, J14_antennadesign}. They are all thin, support dual frequency operation and they all have the disadvantage that 
they 
will transmit at frequencies outside the aimed band which is also known as
spurious radiation. The microstrip patch and slot antenna support both linear
and circular polarization while the printed dipole only supports linear polarization. Further is the fabrication of a microstrip patch antenna considered to be the easiest 
of the considered patch antennae \cite{J13_microstripadvantages}. 


\section{Scenarios}
The default configuration is given below and is always applicable unless mentioned otherwise. 

\begin{table}[!htb]
\centering
\begin{tabular}[t]{ll}
        \toprule
        \multicolumn{2}{l}{\textbf{Broadband cellular network}} \\
        \hline
        \hspace{3mm}  technology        & LTE     \\
        \hspace{3mm}  frequency         & 2.6 GHz \\
        \hline
        \multicolumn{2}{l}{\textbf{Carrier}} \\
        \hline  
        \hspace{3mm}  carrier power        & 13.0 A   \\
        \hspace{3mm}  average carrier speed        & 12.0 m/s \\
        \hspace{3mm}  average carrier power usage      & 17.33 Ah    \\
        \hspace{3mm}  carrier battery voltage       & 22.2 V \\
        \hline
        \multicolumn{2}{l}{\textbf{Femtocell antenna}} \\
        \hline  
        \hspace{3mm}  maximum $P_{tx}$          & 33 dBm   \\
        \hspace{3mm}  antenna  direction        & downwards (az: \ang{0}; el: \ang{90})    \\ 
        \hspace{3mm}  gain                      & 4 dBm   \\ 
        \hspace{3mm}  feeder loss               & 2 dBm   \\ 
        \hspace{3mm}  implementation loss       & 0 dBm   \\
        \hspace{3mm}  radiation pattern         & EIRP or microstrip patch\\
        \hspace{3mm}  height                    & 100m  \\
        \hline
        \multicolumn{2}{l}{\textbf{UE Antenna}} \\
        \hline 
        \hspace{3mm} height                     & 1.5m from the floor       \\ 
        \hspace{3mm} gain                      & 0 dBm   \\ 
        \hspace{3mm} feeder loss               & 0 dBm   \\ 
        \hspace{3mm} radiation pattern         & EIRP  \\
        \hspace{3mm} number present in the network         & 224  \\
        \toprule
\end{tabular}
\caption{Overview of default configuration values.}
\end{table}


Four possible configurations  will be investigated in different scenarios. There are two possible antennae, namely EIRP 
and microstrip pach antenna, which can both be applied in a power consumption optimized network or an exposure optimized network.

\begin{figure}[h!]
  \includegraphics[width=\linewidth]{fourCasesMatrix.pdf}
  \caption{Matrix with the four possible configurations}
  \label{fig:fourCasesMatrix}
\end{figure}

Three main scenarios will be investigated. 
First, only one user with one drone will be present in the network. SAR, electromagnetic exposure, power consumption 
and antenna transmission power are investigated at different flying heights.

In a second scenario, the network is expanded for multiple users but with still only one drone available. 
The scenario is divided into two cases. One with a variable flying height but with a fixed 
number of 224 users which is an average day 
at 5 p.m. in Ghent. In the other case, the number of users varies but the flying height is set to 100 \cite{J2}.
The power consumption, electromagnetic exposure and specific 
absorption rate are investigated in the four different configurations.

The third scenario is quite similar to the previous scenario. It has the same 
two cases with each its four configurations. The only exception is that an unlimited number of UABSs are available.

\section{Methodology}

\subsection{Electromagnetic Exposure}


The total whole body SAR ($SAR^{wb,total}_{10g}$) (expressed in $W/kg$) of a user can be calculated by a simple sum of individual SAR values from the different sources
and is based on \cite{J17_kuehn2019modelling} where SAR values are used that are induced into the head.
Using $SAR^{head}_{10g}$ would however result into incorrect conclusions since 
the position of the phone relative to the user is unknown. 
The position of the phone can be next to the head but also in front of the user.
The induced electromagnetic radiation will therefore be expressed in function of the entire body.


\begin{equation} 
\begin{aligned}
SAR^{wb,total}_{10g} = SAR^{wb,my\_UE}_{10g} +  SAR^{wb,my\_UABS}_{10g} \\
+ SAR^{wb,other\_UE}_{10g} + SAR^{wb,other\_UABSs}_{10g}
\end{aligned}
\label{eq:overallSARwb}
\end{equation}

The first parameter, $SAR^{wb,my\_UE}_{10g}$, indicates the absorbed electromagnetic radiation by the whole body originating from the user's own device.
 However the 
\gls{UL} radiation is destined for the serving \gls{UABS}, a portion of that radiation is directly absorbed by its user, due to the omnidirectional nature of the mobile 
antenna.
The second parameter, $SAR^{wb,my\_UABS}_{10g}$, represents the \gls{DL} radiation caused by the \gls{UABS} who is serving the user.
As the third parameter, we have the $SAR^{wb,other\_UE}_{10g}$ which is radiation caused by the device of other people. The radiation of these devices is once again 
destinated for a specific \gls{UABS} but again, a portion of that \gls{UL} radiation will also be absorbed by our user.
Finally, $SAR^{wb,other\_UABSs}_{10g}$ represents the \gls{DL} radiation by the other UABSs to which our user is exposed to but not served by.
An illustration is given in figure \ref{fig:networkIllustration} where the green arrow is a type near field radiation while 
the others represent far field radiation.

\begin{figure}[h!]
\centering
  \includegraphics[width=\linewidth]{networkIllustrationSARSources.png}
  \caption{Illustration of the network that shows how the average user (here shown in the center) is influenced by different types of sources. }
  \label{fig:networkIllustration}
\end{figure}

\subsection{Electromagnetic Exposure Caused by Far-Field Radiation} % (fold)
\label{sub:Calculatingdownlinkexposure}

\subsubsection{Electromagnetic Radiation from a Single Source}
\label{sec:calculatingexposure}

Calculating far field exposure needs to be done for all UABSs and UE that do not belong to the user.
To determine the total exposure $E$ (expressed in V/m) of this single user $u$ from a single radiator $i$ can be calculated
as follows.

\begin{equation}
E_i(u) [V/m] = 10^{\frac{RRP(u)[dBm] - 43.15 + 20*\log(f [Mhz])- PL(u) [dB]}{20}}
\label{eq:singleexposure}
\end{equation}

Caculating the real radiation power (RRP) for a certain user $u$, requires first the EIRP-value to be calculated  \cite{J6_originalExposureFormula, J1}.
This is achieved by adding the transmit power $P_t$ to the transmitter gain $G_t$ and thereafter subtracting the feeder loss $L_t$.
This formula needs to be expanded to also account for attenuation. This value depends on the angle 
between this user and the antenna's main beam. This leads to the following formula.

\begin{equation}
\begin{aligned}
RRP [dBm] = P_t [dBm] + G_t [dBi]- L_t [dB]\\
     - attenuation(u) [dB]
\end{aligned}
\label{eq:eirp}
\end{equation}
The used frequency in formula \ref{eq:singleexposure} is denoted as $f$ and is expressed in MHz. Since LTE is used, this value will be 2600 MHz.

At last, formula \ref{eq:eirp} requires the path loss (in dB). In order to calculate this, an appropriate propagation model -of which several exist- is required .
The Walfish-Ikegami model is used since it performs well for femtocell networks in urban areas \cite{J2}. %optioneel kan je hier dezelfde bron gebruiken als dat ze in thesis van de vorige gebruikten. Bron nummer 32
%It consists of two formulas depending on whether a free LOS between the user and the base station exists or not. Both formulas expect a distance in kilometre. %bron?

\subsubsection{Combining Exposure}
The electromagnetic exposure, in a certain spot, originating from different sources can be calculated with formula \ref{eq:totalexposure}. $E_i$ stands for 
the electromagnetic exposure from source $i$ and
$n$ stands for all far-field radiators of a certain category which will either be UABSs or UE from other people.
$E_{tot}$ will be calculated for each location where a user is positioned.  
\begin{equation}
E_{tot} [V/m] = \sqrt{\sum_{i=1}^{n} (E_i [V/m]) ^2}
\label{eq:totalexposure}
\end{equation}

\subsubsection{Converting electromagnetic radiation into SAR-values}

Formula \ref{eq:overallSARwb} expects that the radiation is expressed in whole body \gls{SAR}.
To make this calculation possible, a distinction has to be made between near field \gls{SAR}
$SAR^{wb,nf}$ and far field \gls{SAR} $SAR^{wb,ff}$.
%$SAR^{wb,my\_UABS}_{10g}$, $SAR^{wb,other\_UE}_{10g}$ and $SAR^{wb,other\_UABSs}_{10g}$ are forms 
%of far field radiation while $SAR^{wb,my\_UABS}_{10g}$ is a form of near field radiation.

Converting the electromagnetic radiation is done with a conversion factor which is based 
on Duke of the Virtual Family. Duke is a 34-year old male with a weight of 72 kg, a height of 1.74 m and body
mass index of 23.1 kg/m \cite{J22_plets2015joint}. 
Research shows that the conversion factor for WiFi in the far field is $0.0028 \frac{W/kg}{W/m^2}$
and for 0.0070 $\frac{W/kg}{W}$ \cite{J22_plets2015joint} in the near field.

Since WiFi, at a frequency of 2400 MHz,
is very close to LTE, at 2600 MHz, it is assumed in \cite{J22_plets2015joint} that this value is also applicable for \gls{LTE}.

Caculating \gls{SAR} from far field radiation is done as follows:

\begin{equation}
S [W/m^2]= \frac{(E_{tot} [V/m])^2}{337}
\label{eq:flux}
\end{equation}
\begin{equation}
SAR^{wb,ff}_{10g} [W/kg]= S [W/m^2]* 0.0028 \left[\frac{W/kg}{W/m^2}\right]
\label{eq:DLconversion}
\end{equation}

The constant in \ref{eq:DLconversion} converts the \gls{power flux density} $S$ to the required $SAR^{ff,wb}_{10g}$.
To make this possible, the electromagnetic radiation
from formula \ref{eq:totalexposure} should first be converted to the  \gls{power flux density} with formula 
\ref{eq:flux}.

The SAR caused by near field radiation is calculated by multiplying the constant with the used transmission
power of the \gls{UE} (eq. \ref{eq:ulToSar}) 
as follows:

\begin{equation} 
SAR^{wb,nf}_{10g} \left[\frac{W}{kg}\right] = 0.0070 \left[\frac{W/kg}{W}\right] * P_{tx} [W]
\label{eq:ulToSar}
\end{equation}

\subsection{Microstrip Patch antenna}
A microstrip patch antenna is chosen because it allows easy production but more important, it has a low weight 
and has a thin profile causing it to be very aerodynamic which is useful when attaching it to a drone \cite{J13_microstripadvantages}.

The dimensions of the antenna depend on the frequency it is operating at and the characteristics of the used substrate.
The antenna will be radiating at a centre frequency $f_0$ of 2.6 GHz. Each substrate has a dielectric constant $\epsilon_r$ representing 
the permittivity of the substrate and depends on the used material.
Substrates with a high dielectric constant and low height 
reduce the dimensions of the antenna
while a lower dielectric constant with a high height improves the performance of the antenna. 
In this document, a substrate like glass 
is chosen because of the higher dielectric constant of $\epsilon_r = 4.4$ compared to materials like teflon with only a dielectric 
constant of $\epsilon_r = 2.2$ \cite{J14_antennadesign}. 
Doing this in combination with an antenna height of 2.87 mm will decrease the dimensions of the entire antenna surface.
This comes in handy since drones only have limited space available.

\begin{table}[h!]
\centering
\begin{tabular}{|l|c|l|}
\hline
 description            & symbol          & value         \\    \hline
 centre frequency       & $f_0$           & 2600 Hz       \\ 
 dielectric constant    & $\epsilon_r$    & 4.4         \\ 
 height of the substrate & $h$             & 0.00287 m    \\ \hline
\end{tabular}
\caption{Overview of configuration parameters.}
\label{table:antennaparas}
\end{table}

The dimensions of the radiating patch can be calculated with the formulas from \cite{J14_antennadesign, J15_antennadesign}.
Doing so will result in a radiating patch of 35.09 mm by 26.55 mm and a groundplane of at least 52.40 mm by 43.80 mm.
The microstrip patch antenna as illustrated in fig. \ref{fig:basicpatchantenna} will result in the radiation pattern of fig. \ref{fig:radpattern}.
\begin{figure}[h!]
\centering
  \includegraphics[width=\linewidth]{MicrostripAntenna.png}
  \caption{Design of the microstrip patch antenna.}
  \label{fig:basicpatchantenna}
\end{figure}


\begin{figure}[!htb]
\minipage{0.50\linewidth}
  \includegraphics[width=\linewidth]{pattern2/ep.png} 
\endminipage\hfill
\minipage{0.50\linewidth}%
  \includegraphics[width=\linewidth]{pattern2/hp.png}
\endminipage
  \caption{Radiation pattern 1: On the left the radiation pattern of the E-plane and at the right for the H-plane.}
\label{fig:radpattern}
\end{figure}

\subsection{Optimizing the network}

Margot et al. discusses in \cite{J1} how a terrestrial  telecommunication network either can be optimized towards electromagnetic 
exposure of an individual or towards power consumption of the entire network. 
However an increasing transmission power of an antenna comes with an increasing electromagnetic exposure. This is not the case considering
both values for an entire network. In fact, the authors from \cite{J1}  prove that both become inversely equivalent.
The reason the network behaves like this is because it is often cheaper to increase the exposure of an already active base station 
than activating a new one. 
This leads to the following fitness function which is based on \cite{J1}.

\begin{equation} 
f = w * \left(1 - \frac{E_m}{E_{max}}\right) + (1 - w)*\left(1 - \frac{P}{P_{max}}\right) * 100
\label{eq:fitnessfunction}
\end{equation}

Formula \ref{eq:fitnessfunction} returns a fitness value which represents the performance of the entire network. 
$w$ is the importance factor of electromagnetic exposure ranging from 0 to 1, boundaries included. A $w$ set to zero means that electromagnetic 
exposure is not important. Such a network will therefore be called a power consumption optimized network. 
Likewise, a $w$ set to one means that minimizing exposure is top priority and will result in an exposure optimized network. $P_{max}$ is the power consumption of all UABSs, 
both active and inactive, when radiating at the highest level possible 
while $P$ is the effective power used by the current designed network. 
This will be the power required for the flying drones themselves and their antennae.
$E_m$ will be the weighted exposure of the average user for the current designed network and $E_{max}$ the weighted average electromagnetic exposure when all antennae 
are at their highest power level.

When optimizing the network, it is not only important to consider the average exposure of all users, but also to limit high extremes \cite{J1}. A weighted average 
will be used not only considering the median but also the 95 percentile from all users' \gls{DL} exposure using formula \ref{eq:em}. 
Since both values are considered to have equal importance, the weight factors $w_1$ and $w_2$ will both have an equal importance of 50\%. 

\begin{equation} 
E_m = \frac{w_1 * E_{50} + w_2 * E_{95}}{w_1 + w_2}
\label{eq:em}
\end{equation}




\section{Results}
\subsection{One User and One \gls{UAV}}

The  results show that for a varying flying height, a logarithmic relationship exists between the $P_{tx}$ and the flying height. 
This is mainly caused by the logarithmic 
scale in which the decibels of the $P_{tx}$ are expressed. So while 10 dBm equals 10 mW, 20 dBm equals 100 mW. 
Each time the flying height becomes too large to cover, the 
$P_{tx}$ increases with one dBm. 
When using the default configuration where the maximum allowed $P_{tx}$ is set to 33 dBm,
a \gls{UABS} can fly up to 387 m before losing connection in a free \gls{LOS} scenario.

This scenario is investigated with a microstrip patch antenna using power consumption optimization. 
 However, the chosen optimization strategy doesn't really matter because the decision 
 algorithm decides which user 
needs to be connected to which \gls{UABS}. Since only one \gls{UABS} is available, both optimization strategies will behave identical.
Further, also the used antenna will not make any difference.
The user is namely positioned in the perfect centre of the main beam where there is 
no attenuation experienced for both antennae.

When investigating this scenario at different flying heights, we notice 
that the \gls{UL} radiation 
increases exponentially while 
the \gls{DL} radiation remains constant during the entire time. The reason that the \gls{DL} radiation
remains constant is because of power control which makes sure that no more power is used than strictly necessary. 
So at lower flying altitudes, there is less path loss and the \gls{UABS} 
will therefore reduce the $P_{tx}$. This results in formula \ref{eq:exposureBasicFormula} where the electromagnetic exposure is a constant fraction of power and distance.
The \gls{UL} radiation starts very low but surpasses the \gls{DL} radiation 
around 80 metres.

\begin{equation}
\vec{E} (V/m) = \frac{\Delta U (V) }{\Delta x (m)}
\label{eq:exposureBasicFormula}
\end{equation}

\subsection{Increased Traffic}

A power consumption optimized network has the highest exposure, a behaviour that was already proven by \cite{J1}. 
However, a power consumption optimized network will have higher power consumption than an exposure optimized network. 
To understand this, the behaviour of the deployment tool needs to be understood first. 
A power consumption optimized network will result in a few high powered \gls{UABS}s because increasing the input power of an antenna costs 
less than activating a new  \gls{UAV}. Likewise, an exposure optimized network 
generates a lot of low powered \gls{UABS}s because the lower the power of the antenna, the lower the exposure. This has the consequence that the cover radius 
is less and therefore requires more \gls{UAV}s which costs more energy.
When only a limited amount of \gls{UABS}s are available, 
like only one in this scenario, the tool will only keep \gls{UABS}s which cover the most users. 
Therefore, the power consumption in a power consumption optimized network is much more higher. 

Further, the results also show that the exposure increases with higher flying altitudes. At low flying altitudes, 
there is a lot of path loss by obstructing buildings. When 
\gls{UABS}s fly higher, the exposure increases and more users become covered. 
The increasing electromagnetic radiation is however not unlimited.
At even higher
flying altitudes, the distance between a given \gls{UABS} and users further away becomes too large causing the 
coverage to decrease again. When this decrease occurs depends on the configuration. A power consumption optimized 
network tends to decrease earlier than an exposure optimized network.

When replacing the fictional \gls{EIRP} antenna by a microstrip patch antenna, the percentage of covered users drops for both 
optimization strategies. This is because users, who have a higher horizontal distance between themselves and the \gls{UABS}, 
experience a higher attenuation.

The results further show  
that the radiation from the \gls{UABS} is the main factor followed by the near field radiation from the user's own device.
The far field radiation from other \gls{UE} barely contribute anything.

\subsubsection{Influence of the Number of Users}

Also these results show how EIRP antennae designs are able to cover more users than microstrip patch antennae
just like power consumption optimized networks will reach more users than exposure optimized networks.
The contribution to the total \gls{SAR} from each individual source is identical to the previous scenario. \textcolor{red}{ << deze zin weg?}

There is still only one \gls{UABS} available. So when population grows, more users become uncovered and 
therefore the average electromagnetic exposure decreases.
For example, an EIRP power consumption optimized network
will have the highest exposure and therefore covers the most users as opposed to a microstrip patch antenna in 
an exposure optimized network which will radiate the least and thus has the lowest number of covered users.

While the population grows, more and more users become uncovered causing the average SAR to drop. 
However, this does not conclude that by increasing the population, the SAR of a user who is directly beneath a \gls{UABS} would be less.
To investigate this, a user is positioned in the middle of the city centre of Ghent and a \gls{UAV} is positioned above him. Initially, only 
49 people are active around him. The \gls{SAR} of our central user is monitored while the population around him is growing.
Figure \ref{fig:connectionMap} shows with the black lines which users are connected. The left map is for only 50 users and 
shows that only one user is connected besides our central user. The map on the right is taken with 600 users and shows much more connected users.
The results show how the \gls{UL} \gls{SAR} remains constant. A normal behaviour since the flying altitude does not change.
The \gls{SAR} from the \gls{UABS} experiences a slight increase. When the population grows, more users become available 
and some will spawn near the central user. The \gls{UABS} will likely decide to cover these users as well as visible in figure \ref{fig:connectionMap}.
These users might have a slightly 
worse path loss because of obstructing buildings or somewhat bigger distance. The \gls{UABS} reacts to this by increasing 
his power consumption causing an increase in the \gls{DL} \gls{SAR} for the central user. The results further also show 
that the \gls{SAR} from other \gls{UE} increases when the population increases. But as mentioned  before, it is much less 
compared to the other sources.

\begin{figure}[!htb]
\minipage{0.50\linewidth}
  \includegraphics[width=\linewidth]{../images/connectionsMap50Users.png}
\endminipage\hfill
\minipage{0.50\linewidth}%
  \includegraphics[width=\linewidth]{../images/connectionsMap600Users.png}
\endminipage
  \caption{Overview of which users are connected to the \gls{UABS}. The map on the left is for 50 active users while the map on the right is with 600 active users.}
  \label{fig:connectionMap}
\end{figure}

\subsection{Unlimited Number of UABSs}
The same scenario  as the previous one is investigated. Only now, an unlimited number of \gls{UABS}s are available.
The results prove that the different optimization strategies work as intended.
Power consumption optimized networks have indeed a lower power consumption but therefore result in higher electromagnetic radiation.
On the other hand, an exposure optimized network will reduce the electromagnetic exposure by using more \gls{UAV}s and thence also increasing the network's power consumption.
This conclusion was already made  in \cite{J1} and is supported by these results.
\begin{figure}[h!]
  \includegraphics[width=\linewidth]{../results/s3/fhvsdlAndPc.png}
  \caption{These two figures show how the flying height influences the downlink electromagnetic radiation of the average user (left) and 
  power consumption of the entire network (rights) for an unlimited number of drones.}
  \label{fig:s3a_dlAndPc}
\end{figure}

The exposure in an exposure optimized network increases logarithmically while the power consumption optimized network rather 
shows a concave relationship with it's lowest point around 70 metres.
At a flying height of 20 meters, the exposure optimized network has on average 220 to 224 \gls{UABS}s. That is (almost) one \gls{UABS} for each user
so it's logical that the electromagnetic exposure is very low.
The number of \gls{UAV}s in a power consumption optimized network is much less in order 
to save energy but the same percentage of coverage is still achieved.
So these \gls{UAV}s will try to cover more users requiring higher level of radiation. This is certainly the case 
since there are a lot of obstructing buildings at this flying altitude.
Because of this, users who are close and in \gls{LOS} will experience much higher electromagnetic radiation.

The results further show that the network profits from increasing the flying altitude. Not only
less \gls{UAV}s are needed but also the power consumption is lower. Both can be explained by the
lower path loss when UABSs fly higher. At higher flying altitudes, 
the exposure from the \gls{UE} increases. A behaviour also explained in the first scenario.
The SAR from the serving \gls{UABS} is identical to the exposure which has been explained earlier.
When looking at the exposure from `other \gls{UABS}s', an increase in electromagnetic radiation at higher 
flying altitudes is notices.
Also here the lower path loss from less obstructing buildings will be the reason.
The figures from \ref{fig:s3a_fourSourcesMatrix} further also clearly show that this increase 
in electromagnetic radiation will be less for a microstrip patch antenna. The reason behind this is that energy 
will be more focussed towards the ground and there is less sideways radiation because of attenuation.

\begin{figure}[h!]
  \includegraphics[width=\linewidth]{../results/s3/fhFourSources.png}
  \caption{Each chart corresponds with one of the four possible configurations. The contribution of each source towards towards the total 
  \gls{SAR} for a varying flying height is shown.}
  \label{fig:s3a_fourSourcesMatrix}
\end{figure}

\subsubsection{Variable Number of Users}
When the flying height of the \gls{UABS}s are fixed to 100 metres and the density of the population increases, also 
the number of required drones increases in order to reach a 100 \% coverage. When the number of drones increases, 
so does the electromagnetic exposure and power consumption.
Once again, the EIRP antenna in a power consumption network has the highest exposure for the lowest power consumption
and a microstrip patch antenna in an exposure optimized network the lowest exposure for the highest power consumption.
The two other combinations are in the middle and behave very similar.

\begin{figure}[h!]
  \includegraphics[width=\linewidth]{../results/s3/uvsdlAndPc.png}
  \caption{These two figuress show how the number of users influences the downlink electromagnetic radiation of the average user (left) and 
  power consumption of the entire network (rights) for an unlimited number of drones.}
  \label{fig:s3b_dlAndPC}
\end{figure}

When looking at the different contributions to the total \gls{SAR} in figure \ref{fig:s3b_fourSourcesMatrix}, 
we see that the weighted average 
\gls{SAR} from the users own device and from the serving \gls{UABS} remains constant. The flying altitude is always the same so 
also the required energy to cover that distance will remain the same. 
The only \gls{SAR} value that increases are the \gls{DL} \gls{SAR} from other \gls{UABS}s and the \gls{UL} \gls{SAR} from other \gls{UE}. 
When more users come online, also more \gls{UAV}s will be radiating. Moreover, there is very little path loss because the flying height is above the average building.

\begin{figure}[h!]
  \includegraphics[width=\linewidth]{../results/s3/uFourSources.png}
  \caption{Each chart corresponds with one of the four possible configurations. The contribution of each source towards towards the total 
  \gls{SAR} for a varying number of users shown.}
  \label{fig:s3b_fourSourcesMatrix}
\end{figure}

\section{Conclusion}
%How can a UABS network be optimized to minimize global exposure or overall power consumption? 
Literature showed that a network can be optimized towards either the power consumption of the entire network 
or the electromagnetic exposure of the average user using a fitness function \cite{J1}.
However, the fitness function should be used with care considering that \gls{UABS}s can be placed anywhere as opposed to 
the transmission towers from \cite{J1} who have a predetermined position. This causes that in an exposure optimized network, 
a lot of users get a \gls{UABS}
all by themselves because this is the best approach to minimize exposure.
A power consumption optimized network on the other hand will try to limit the number of drones 
in order to save energy. 
So as a rule of thumb: an exposure optimized network will result in a lot of low powered devices (increasing the overall power consumption)
while a power consumption optimized network results in a few high powered devices (increasing the exposure of the average user).
A power consumption optimized network is thus cheaper because less drones are involved. 
Moreover, the results show that the electromagnetic radiation in a power consumption optimized network (with high powered \gls{UABS}s)
is far below the thresholds enforced by the Flemish government.

The user's main sources of exposure are the user's own device and the \gls{UABS} who is serving him followed by all
other \gls{UABS}s in the network. 
When the population increases, there is not only more radiation from \gls{UE} but also 
from more \gls{UABS}s that are serving the other users.
The exposure from other people their \gls{UE} is so low that it can be neglected.
An exposure optimized network will limits the total exposure mainly by trying to reduce the exposure from other \gls{UABS}s.

%1)	How does the network behave differently after the introduction of a realistic antenna?
A directional microstrip patch antenna is introduced because it gives several advantages compared to omnidirectional antennae.
Directional antennae are able to focus their energy there where it is need, namely towards the ground. Microstrip patch antennae 
further benefit from their thin and lightweight design. The performance 
of this directional microstrip patch antenna has been compared to a 
fictional \gls{isotropicradiator}.
This \gls{isotropicradiator} has higher exposure and coverage for less power compared to realistic antennae like microstrip patch antenna
because of the absence of attenuation and can hypothetically be compared with an antenna with a very big aperture angle.
This type of antenna can achieve the same coverage with less
resources like power and number of drones. 
A microstrip patch antenna  with a more limited aperture angle of \ang{90} requires more resources but 
causes less sideways radiation. So the exposure from other \gls{UABS}s will be way less.

\begin{figure}[h!]
  \includegraphics[width=\linewidth]{fourCasesMatrixSol.png}
  \caption{Matrix with the four possible configurations. Colour-coded based on the results.}
  \label{fig:resultIllustration}
\end{figure}

Remarkable is that an \gls{EIRP} exposure optimized network behave very similar to a microstrip power consumption optimized network as shown 
in figure \ref{fig:resultIllustration}.
This results in the best of both worlds. 
The microstrip patch antenna will generate less electromagnetic radiation by design and
 the power consumption optimization reduces the number of required drones and power. A microstrip patch antenna with an aperture 
 angle of \ang{90} is considered a good solution but if cost is more restricted, an antenna with a larger aperture angle 
 would further reduce cost without interfering with the Flemish legislation regarding electromagnetic exposure.

The electromagnetic radiation of an exposure optimized network increases with higher flying altitudes.
Around 80 metres, the exposure from the  user's device surpasses the exposure from the serving \gls{UABS}.
On the other hand, a power consumption optimized network shows that the lowest exposure is measured around 70 to 80 metres.
Further, the results also show that the number of required drones decrease when the flying height becomes larger. 
A conclusion that was also made in \cite{J2}.
When also considering the results from \cite{U1} where a flying altitude from 
80 metres is suggested for an optimal access and backhaul connectivity, a flying height 
of 80 metres is also here proposed for the city centre of Ghent.

In short, a power consumption optimized network is proposed with a fixed flying height of 80 metres. A microstrip patch 
antenna with a sufficiently large aperture angle is a good starting point. However, different antenna configurations should 
be investigated 

\section*{Acknowledgement}
Special thanks to the WAVES research group at Ghent University for providing 
access to their capacity based deployment tool and therefore making this research possible.

\bibliographystyle{ieeetr}
\bibliography{referenties}


\end{document}
