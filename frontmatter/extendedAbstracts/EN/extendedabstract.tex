%%%%%%%%%%%%%%%%%%%%%%%%%%  phdsymp_sample2e.tex %%%%%%%%%%%%%%%%%%%%%%%%%%%%%%
%% changes for phdsymp.cls marked with !PN
%% except all occ. of phdsymp.sty changed phdsymp.cls
%%%%%%%%%%                                                       %%%%%%%%%%%%%
%%%%%%%%%%    More information: see the header of phdsymp.cls   %%%%%%%%%%%%%
%%%%%%%%%%                                                       %%%%%%%%%%%%%
%%%%%%%%%%%%%%%%%%%%%%%%%%%%%%%%%%%%%%%%%%%%%%%%%%%%%%%%%%%%%%%%%%%%%%%%%%%%%%%


%\documentclass[10pt]{phdsymp} %!PN
\documentclass[twocolumn]{phdsymp} %!PN
%\documentclass[12pt,draft]{phdsymp} %!PN
%\documentstyle[twocolumn]{phdsymp}
%\documentstyle[12pt,twoside,draft]{phdsymp}
%\documentstyle[9pt,twocolumn,technote,twoside]{phdsymp}

\usepackage[english]{babel}       % Voor nederlandstalige hyphenatie (woordsplitsing)

\usepackage{graphicx}                   % Om figuren te kunnen verwerken
\usepackage{graphics}			% Om figuren te verwerken.
\graphicspath{{figuren/}}               % De plaats waar latex zijn figuren gaat halen.

\usepackage{times}

\hyphenation{si-mu-la-ted re-a-lis-tic packets really in-clu-ding}

\def\BibTeX{{\rm B\kern-.05em{\sc i\kern-.025em b}\kern-.08em
    T\kern-.1667em\lower.7ex\hbox{E}\kern-.125emX}}

\newtheorem{theorem}{Theorem}

\begin{document}

\title{Evaluating Human Electromagnetic Exposure in a UAV-aided Network}

\author{Thomas Detemmerman}

\supervisor{Wout Joseph, Luc Martens, Luc Martens, German Dario Castellanos Tache}

\maketitle

\begin{abstract}
Society relies more than ever on the availability of the wireless networks but is at the same time also 
concerned about the potential health effects of the electromagnetic radiation caused by these networks.
The government has enforced strict legislations to which mobile devices and base stations have to satisfy.

This research investigates the specific absorption rate caused by these electromagnetic waves by taking all mobile devices and base stations into account.
To accomplish this goal, the deployment tool developed by the WAVES research group at Ghent University will be used. This tool simulates an entire network 
where transmission towers are represented by femtocell base stations attached to drones. This research also investigates how these drones can be guided 
in order to reach certain goals like minimizing power consumption or electromagnetic exposure.

It looks from the results that ... (todo)
\end{abstract}

\begin{keywords}
LTE, Electromagnetic Radiation, power consumption, drones, femtocell, microstrip patch antenna, radiation pattern, specific absorption rate (SAR)
\end{keywords}

\section{Introduction}
\PARstart{S}{ociety} is constantly getting more and more dependent on wireless communication. On any
given moment, in any given location, an electronic device can request to connect to the bigger
network. Devices need more than ever to be connected. Also in exceptional and possibly life-threatening situations, the public relies on the cellular
network. For example during the terrorist attacks at Brussels Airport, mobile network operators
saw all telecommunications drastically increasing causing moments of contention. Some
operators decided to temporarily exceed the exposure limits in order to handle all connections.
Electromagnetic exposure can however not be neglected. Research shows how excessive electromagnetic
radiation can cause diverse biological side effects [3].
This research tries to map  the electromagnetic exposure of the average user. In order 
for this to work, an existing planning tool is used and the
three prominent sources of radiation in a telecommunication network are investigated, being: the user’s own
phone, all base stations and all devices from other users in the network.
The electromagnetic behaviour of the network will be analysed by applying the tool in different
scenarios to give insight which variables influence the exposure and how the network can be
optimized accordingly.



\section{State of the Art}

\subsection{Gerelateerd werk}
TODO

\subsection{Scenario's}
todo 

\subsection{Electromagnetische blootstelling}
todo

\section{Resultaten}
todo

\section{Conclusie}
todo

\subsection{Referencies}
todo


\nocite{*}
\bibliographystyle{phdsymp}
%%%%%\bibliography{bib-file}  % commented if *.bbl file included, as
%%%%%see below


%%%%%%%%%%%%%%%%% BIBLIOGRAPHY IN THE LaTeX file !!!!! %%%%%%%%%%%%%%%%%%%%%%%%
%% This is nothing else than the phdsymp_sample2e.bbl file that you would%%
%% obtain with BibTeX: you do not need to send around the *.bbl file        
%%
%%---------------------------------------------------------------------------%%
%
\begin{thebibliography}{1}
\bibitem{paper}
Bart Lannoo, Didier Colle, Mario Pickavet, Piet Demeester,
\newblock {\em Optical Switching Architecture to Implement Moveable Cells in a Multimedia Train Environment},
\newblock Proc. of ECOC 2004, 30th European Conf. on Optical Communication, vol. 3, pp. 344-345, Stockholm, Sweden, 5-9 Sep. 2004.

\bibitem{ns-click}
Michael Neufeld, Ashish Jain, Dirk Grunwald,
\newblock {\em Nsclick:: bridging network simulation and deployment},
\newblock http://systems.cs.colorado.edu/Networking/nsclick/

\bibitem{click}
\newblock {\em The Click Modular Router Project},
\newblock http://www.read.cs.ucla.edu/click/

\bibitem{ns}
\newblock {\em {NS} -- {N}etwork {S}imulator},
\newblock http://nsnam.isi.edu/nsnam/

\end{thebibliography}
%
%%---------------------------------------------------------------------------%%

\end{document}

%%%%%%%%%%%%%%%%%%%%%  End of phdsymp_sample2e.tex  %%%%%%%%%%%%%%%%%%%%%%%%%%%
