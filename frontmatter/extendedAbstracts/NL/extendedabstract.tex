%%%%%%%%%%%%%%%%%%%%%%%%%%  phdsymp_sample2e.tex %%%%%%%%%%%%%%%%%%%%%%%%%%%%%%
%% changes for phdsymp.cls marked with !PN
%% except all occ. of phdsymp.sty changed phdsymp.cls
%%%%%%%%%%                                                       %%%%%%%%%%%%%
%%%%%%%%%%    More information: see the header of phdsymp.cls   %%%%%%%%%%%%%
%%%%%%%%%%                                                       %%%%%%%%%%%%%
%%%%%%%%%%%%%%%%%%%%%%%%%%%%%%%%%%%%%%%%%%%%%%%%%%%%%%%%%%%%%%%%%%%%%%%%%%%%%%%



\documentclass[twocolumn]{phdsymp_dutch}
\usepackage[dutch]{babel}   % anders werken de kleuren niet op de pi chart
\hyphenation{mo-ge-lijk tij-de-lijk si-tu-a-tie ideale WAVES af-stand bij-voor-beeld}

\usepackage{graphicx}			% Om figuren te verwerken.
\graphicspath{{./}{../../../images/}{../../../results/}}
\usepackage{booktabs}
\usepackage{times}
\usepackage{siunitx}
\usepackage{xcolor}
\usepackage{amsmath}
\usepackage{graphicx}
\PassOptionsToPackage{hyphens}{url}
\usepackage{url}
\usepackage{subfig}
\usepackage{placeins} %\FloatBarrier
\usepackage{tikz}


\usepackage[acronym,toc,shortcuts]{glossaries}
\setglossarystyle{super}
\renewcommand{\glsnamefont}[1]{\textbf{#1}}
\makeglossaries




%-------------------------------- acroniemen
\newacronym{UABS}{UABS}{Unmanned Arial Base Station}
\newacronym{EIRP}{EIRP}{equivalent isotropic radiation power}
\newacronym{UE}{UE}{User Equipment}
\newacronym{IEC}{IEC}{International Electrotechnical Commission}
\newacronym{SAR}{SAR}{Specific Absorption Rate}
\newacronym{whipp}{WHIPP}{WiCa Heuristic Indoor Propagation Prediction}
\newacronym{DL}{DL}{downlink}
\newacronym{UL}{UL}{uplink}
\newacronym{LTE}{LTE}{Long-Term Evolution}
\newacronym{FDD}{FDD}{Frequency Division Duplex}
\newacronym{TDD}{TDD}{Time Division Duplex}
\newacronym{ICNIRP}{ICNIRP}{International Commission on Non-Ionizing Radiation Protection}
\newacronym{LOS}{LOS}{line of sigh}
\newacronym{NLOS}{NLOS}{non line of sigh}
\newacronym{FCC}{FCC}{Federal Communications Commission}
\newacronym{Exp Opt}{Exp Opt}{exposure optimized network}
\newacronym{PwrC Opt}{PwrC Opt}{power consuption optimized network}
\newacronym{WHO}{WHO}{World Health Organization}
\newacronym{USA}{USA}{United States of America}
\newacronym{IOT}{IoT}{Internet of Things}
\newacronym{UAV}{UAV}{Unmanned Aerial Vehicle}
\newacronym{EU}{EU}{European Union}
%--------------------------------- woordenlijst
\newglossaryentry{isotropicradiator}{
	name = equivalent isotropic radiator,
	text = equivalent isotropic radiator,
	description = A theoretical source of electromagnetic waves which radiates the same intensity for all directions
}

\newglossaryentry{spuriousradiation}{
	name = spurious radiation,
	text = spurious radiation,
	description = According to the thefreedictionary.com: Any emission from a radio transmitter at frequencies outside its frequency band. Also known as spurious emission
}

\newglossaryentry{RRP}{
	name = RRP,
	text = RRP,
	description = RRP is an abreviation used in this paper to indicate an extension on EIRP and stands for Real Radiation Pattern. An RRP value indicates the power (in dBm) for a certain location unlike an EIRP where the power (in dBm) is independent of the location
}

\newglossaryentry{power flux density}{
	name = power flux density,
	text = power flux density,
	description = Magnitude of power ($W$) that travels through a curtain area ($m^2$)
}

\newglossaryentry{thermoregulatory capacity}{
	name = thermoregulatory capacity,
	text = thermoregulatory capacity,
	description = The capacity of an organism to regulate body temperture
}

\newglossaryentry{exact algorithm}{
	name = exact algorithm,
	text = exact algorithm,
	description = An exact algorithm solves an optimization problem optimally
}








\def\BibTeX{{\rm B\kern-.05em{\sc i\kern-.025em b}\kern-.08em
    T\kern-.1667em\lower.7ex\hbox{E}\kern-.125emX}}

\newtheorem{theorem}{Theorem}

\begin{document}

\title{Evaluatie van de elektromagnetische blootstelling van de mens in een netwerk van drones}

\author{Thomas Detemmerman}

\supervisor{Prof. dr. ir. Wout Joseph, Prof. dr. ir. Luc Martens}

\maketitle

\begin{abstract}

De hedendaagse samenleving vertrouwt meer dan ooit op de aanwezigheid van draadloze netwerken. 
Dankzij de mobiliteit van  drones kan een drone-gestuurd netwerk de nodige mobiele data voorzien 
indien het bestaande netwerk beschadigd is.
Elke drone wordt daarom uitgerust met een femtocell base station.
Er is echter een groeiende vrees voor mogelijke gezondheidseffecten veroorzaakt door deze
mobiele netwerken. De overheid stelt strikte wetgevingen op waaraan deze mobiele netwerken dienen te voldoen.

Dit onderzoek bekijkt hoe verschillende scenario's het energieverbruik, elektromagnetische blootstelling en 
specifieke absorptietempo kunnen be\"invloeden.
Drie verschillende scenario's zijn gedefinieerd waarbij verschillende vlieghoogtes, aantal drones en 
populatiegroottes onderzocht worden.
Verder is er ook een microstrip patch antenne gedefinieerd en bevestigd op een drone. 
De antenne zal de communicatie tussen de drone en de gebruikers verzorgen.
De performantie van deze antenne zal vergeleken worden met een isotrope antenne.
Vervolgens zal het netwerk geoptimaliseerd worden naar elektromagnetische straling van het individu of 
naar het energieverbruik van het gehele netwerk. Deze twee doelstellingen resulteren in 
tegenstrijdige vereisten. 

Om dit doel te bereiken is de capacity based deployment tool van de onderzoeksgroep WAVES op de 
Universiteit Gent verder uitgebreid zodoende dat elektromagnetische straling berekend kan worden.
Verder is de tool nu ook in staat om te optimaliseren naar elektromagnetische straling of energieverbruik. 

Uit de resultaten blijkt dat een microstrip patch antenne
met een openingshoek van \ang{90} een geschikt startpunt is voor een antenne.
Deze directionele  antenne focust de elektromagnetische straling daar waar het nodig is.
Ongewenste zijwaartse straling wordt gereduceerd door het design.
Het wordt aangeraden om de antenne toe te passen in een netwerk dat energieverbruik minimaliseert
omdat hierbij minder drones nodig zijn en daardoor goedkoper is.
De optimale vlieghoogte voor het stadscentrum in Gent bevindt zich rond 80  meter.
Lagere vlieghoogtes vereisen veel meer drones terwijl hogere vlieghoogtes de 
elektromagnetische straling laten toenemen.
Wanneer deze configuratie toegepast wordt op een netwerk met 224 gebruikers zal de gewogen gemiddelde gebruiker 
een SAR ondervinden van  $0.2\ \mu W/kg$ en een downlink elektromagnetische straling van 
$114\ mV/m$. Het netwerk zal hiervoor gemiddeld 96 drones vereisen met een totaal energieverbruik 
van $69.5\ W$. Dat is $7.24\ W$ per drone.
  
\end{abstract}

\begin{keywords}
LTE, elektromagnetische blootstelling, 
Energieverbruik, Drone,
Femtocell, Microstrip patch antenne, Stralingspatronen, Specifiek absorptietempo (SAT).
\end{keywords}

\section{Introductie}
\PARstart{D}{e} samenleving is meer dan ooit afhankelijk van draadloze communicatie.
Een elektronisch apparaat kan op elk gegeven moment in elke willekeurige plaats beroep doen 
op het draadloos netwerk, gaande van kleine \gls{IOT} apparaten tot volwaardige zelf-rijdende auto's.

Ook in uitzonderlijke en zelfs mogelijks levensbedreigende situaties verwacht de samenleving de aanwezigheid 
van het mobiele netwerk. Desondanks het feit dat dit netwerk zelf mogelijk beschadigd kan zijn door de situatie.
Een mogelijk tijdelijke oplossing om een beschadigd netwerk bij te staan is met behulp van onbemande vliegtuigen zoals drones.
Een base station kan geplaatst worden op een drone en zo effici\"ent verplaatst worden naar de nodige locatie.

Deze aanpak is niet alleen handig als het bestaande netwerk beschadigd is maar ook voor een onverwachte toename aan gebruikers.
Bijvoorbeeld tijdens de aanslagen op de Brusselse luchthaven zagen alle mobiele operatoren een toename in data verkeer.
Sommige operatoren raakten zodanig verzadigd dat ze beslisten om de
elektromagnetische straling te laten toenemen boven de opgelegde limieten zodat toch iedereen behandeld kon worden \cite{baseZaventem}.

De elektromagnetische straling die vrijkomt bij netwerken kan echter niet met onachtzaamheid behandeld worden.
Onderzoek toont aan dat buitensporige elektromagnetische straling verscheidinge biologische neveneffecten kan veroorzaken \cite{J31_bioeffects,WHO}.
Het is dus duidelijk dat elektromagnetische straling een sleutelrol speelt bij het ontwikkelen van een met drones beholpen netwerk 
waarbij de wetgeving nauwkeurig nageleefd dient te worden.

Drone-gestuurde netwerken kunnen dankzij hun mobiliteit eenvoudig verplaatst worden. Verschillende onderzoeken 
tonen aan hoe deze netwerken geoptimaliseerd kunnen worden zodat bepaalde doelstellingen zoals minimaal energieverbruik
bereikt kunnen worden.

Niettemin is er zeer beperkt onderzoek gedaan waarbij een drone-gestuurd netwerk wordt geoptimaliseerd naar elektromagnetische straling.
Verscheidene publicaties bespreken hoe elektromagnetische straling berekend kan worden maar 
overwegen zelden alle verschillende bronnen van straling.

Dit onderzoek stelt een methode voor waarbij rekening gehouden wordt met 
elektromagnetische straling en energieverbruik voor alle bronnen in een mobiel netwerk, zijnde: de gebruiker zijn eigen 
mobiel apparaat, de base station dat deze gebruiker aan het behandelen is, alle andere mobiele apparaten en 
alle andere base stations die andere gebruikers behandelen. Op deze manier kan de bijdrage in elektromagnetische straling
 van elke bron duidelijk geïdentificeerd worden. 

Het gedrag van de elektromagnetische straling en het energieverbruik zullen geanalyseerd worden door de 
tool toe te passen op verschillende scenario's door gebruik te maken van verschillende soorten antennes, vlieghoogtes 
en populatiegroottes.
Waarden zoals \gls{SAR}, elektromagnetische straling en energieverbruik zullen inzicht 
geven in hoe het netwerk reageert op deze veranderende scenario's en hoe het netwerk 
ernaar geoptimaliseerd kan worden.

Om dit onderzoek mogelijk te maken zal een bestaande deployment tool, ontwikkeld
door de onderzoeksgroep WAVES van de universiteit van Gent, uitgebreid worden. Deze tool 
beschrijft een volledig geconfigureerd netwerk van drones wat een geschikt startpunt is voor dit onderzoek.

\section{State of the Art}
\subsection{Elektromagnetische Straling}

Personen in een mobiel netwerk worden blootgesteld aan verscheidene bronnen van elektromagnetische straling, uitgedrukt in $V/m$.
Eenmaal deze elektromagnetische straling geabsorbeerd wordt door het menselijk lichaam spreken we van het specifieke absorptietempo (\gls{SAR}) (dat uitgedrukt wordt in $W/kg$).
De \gls{ICNIRP} heeft geconcludeerd dat het drempeleffectwaarde voor $SAR^{wb}_{10g}$ zich bevindt op 
 4 W/kg wat inhoud dat elk hoger absorptietempo de \gls{thermoregulatory capacity} van de mens 
 zou overstijgen  \cite{J23,J24}.
Al deze waarden zijn onderworpen aan limieten opgelegd door de overheid.
Dit onderzoek vindt plaats in Gent, een Vlaamse stad in Belgi\"e, waarbij voor het 2.6 GHz spectrum een individuele zendmast 
is gelimiteerd tot 4.5 V/m en de totale elektrische veldsterkte voor elk punt niet meer dan 31 V/m mag bedragen.  \cite{J23,S13_normenBelgie}. 
De maximale \gls{SAR} voor het volledige lichaam komende van een mobiel apparaat verspreid over een 
10 g tissue ($SAR_{10g}$) is beperkt tot $0.08 W/kg$ \cite{J30,J23,S20}. 
De \gls{FCC} bepaalt de limieten voor de \gls{USA} en deze zijn gebaseerd op de 
\gls{IEEE} Std C95.1-1999 \cite{P1,P2} dewelke gebruik maken van een 1 g tissue.
De $SAR^{wb}_{1g}$ van het mobiele apparaat mag de 
1.6 $W/kg$ niet overschrijden. 
%Dit ondanks het feit dat deze waarde inmiddels herzien is door de  \gls{IEEE} en gedefinieerd wordt op 8 $W/kg$ in Std C95.1-2005 \cite{P2}.
Een overzicht wordt gegeven in tabel \ref{table:overviewSARValues}.
\begin{table}[h!]
\centering
\begin{tabular}{|c|l|c|c|}
\hline
\textbf{Instelling}  & \textbf{Description}                  & \textbf{Value}  & \textbf{Units} \\ \hline
\gls{ICNIRP}          & $SAR^{wb}_{10g}$                      &  $4$            & $W/kg$              \\  \hline
BE                    & $SAR^{wb}_{10g}$ van base stations     & $0.08$          & $W/kg$               \\ \hline
BE                    & $SAR_{10g}^{hoofd}$ for \acs{UE}       & $2$             & $W/kg$               \\ \hline
\gls{USA}             & $SAR_{1g}^{hoofd}$ for \acs{UE}        & $1.6$           & $W/kg$               \\ \hline
\end{tabular}
\caption{Overview of the different \acs{SAR} limitations.}
\label{table:overviewSARValues}
\end{table}

Verschillende onderzoeken berekenen de elektromagnetische veldsterkte van verschillende bronnen \cite{J6_originalExposureFormula,J1,J10_RDP,J10.1} 
waarbij sommigen de \gls{UL} elektromagnetische veldsterkte converteren naar lokale \gls{SAR} voor het hoofd en torso \cite{J10_RDP,J10.1}. 
Met de naderende 5G technologie werd \cite{J17_kuehn2019modelling} gepubliceerd waarbij beschreven wordt hoe 
deze lokale  \gls{SAR}-waarden van alle verschillende bronnen berekend kunnen worden en bij elkaar opgeteld worden.
Uiteindelijk beschrijft \cite{J22_plets2015joint} hoe de elektromagnetische veldsterkte omgezet kan worden 
naar \gls{SAR}-waarden voor het volledige lichaam.

In een realistisch netwerk kunnen sommige gebruikers telefoneren terwijl anderen andere vormen van telecommunicatie gebruiken zoals 
surfen op het internet. Aangezien de positie van het mobiel apparaat tegenover zijn gebruiker niet gekend is,
is het belangrijk dat de \gls{SAR}-waarden berekend worden in functie van het volledige lichaam. 

\subsection{Geoptimaliseerde drone-gestuurde netwerken}

Drones kennen verschillende toepassingen. Ze werden oorspronkelijk voornamelijk gebruikt door het leger waarbij ze dienst doen  
als camera ondersteuning of om aanvallen uit te voeren zonder piloten in gevaar te brengen \cite{U12}. 
Deze drones zijn de laatste jaren in prijs gedaald waardoor ze beter toegankelijk worden
voor het algemene publiek. Hierdoor is het onderzoek naar nieuwe toepassingen ervan sterk toegenomen.

Een drone uitgerust met een femtocell base station wordt een \gls{UABS} genoemd en 
geniet verschillende voordelen zoals mobiliteit en snelle inzetbaarheid.
Desondanks zijn er ook verschillende nadelen zoals het beperkte gewicht dat een drone kan dragen 
en de schaarse energievoorziening.

Kawamoto et al. introduceert in \cite{U11} een WiFi netwerk  met behulp van drones waarbij rekening gehouden wordt
met de richting van de geplaatste antennes op de drone. 
Gangula et al. illustreert in \cite{U10} hoe drones gebruikt kunnen worden voor \gls{LTE}
en
Zeng et al. presenteert in  \cite{U12} een handleiding waarbij uitdagingen   zoals energieverbruik, mobiliteit en 
de richting van de antenne voor een 5G netwerk besproken worden. 
In \cite{J2} ontwikkelt Deruyck et al. een deployment tool voor een drone gestuurd netwerk voor rampsituaties waarbij 
een ideale vlieghoogte van 100 meter aangeraden wordt.  
Dit wordt verder uitgebreid in \cite{U1} waarbij ook rekening gehouden wordt met 
 direct-link backhaul connecties waarbij een ietwat lagere vlieghoogte van 80 meter bekomen wordt.

Mozaffari et al. voorziet in \cite{U3} richtlijnen hoe een drone-gestuurd netwerk geoptimaliseerd en geanalyseerd kan worden.
Eén onderzoeksgebied dat uitgebreid onderzocht wordt, is het optimaliseren van de locaties waar drones zich moeten positioneren.
Deze algoritmen trachten bepaalde doelstellingen zoals minimaal energieverbruik of kortste vliegafstand te bereiken \cite{U6,U7,U8,U9}.
Deze optimalistatie kan gebeuren door verschillende implementaties waaronder exacte algoritmen of machinaal leren \cite{U3,U5}.

Onderzoek waarbij de elektromagnetische straling gelimiteerd wordt is echter beperkt.
Deruyck et al. bespreekt in \cite{J1} hoe een conventioneel mobiel netwerk geoptimaliseerd kan worden zodoende dat het energieverbruik 
van het volledige netwerk minimaal wordt of de elektromagnetische blootstelling van een individu geminimaliseerd wordt.
Echter, onderzoek waarbij een drone-gestuurd netwerk geoptimaliseerd wordt naar elektromagnetische straling is 
door de auteur niet gekend.

\subsection{Technologie\"en}
Voor het ontwikkelen van het netwerk zullen de meer robuuste drones uit \cite{J2} gebruikt worden (details in tabel \ref{table:defaultconf}). De 
gekoppelde antennes zullen opereren in het 2.6 GHz spectrum. Aangezien het aangenomen wordt dat de gebruikers een voortdurende blootstelling
van  elektromagnetische straling ondervinden, zonder onderbrekingen, wordt frequency division duplexing gebruikt. 

% problem antennae on drones
De antenne op de drone zal dienst doen als gateway tussen de mobiele apparaten op de grond en het backbone netwerk.
Bepalen welke antenne gebruikt moet worden en hoe deze vervolgens het beste gepositioneerd kan worden brengt 
verschillende uitdagingen met zich mee. Het stralingspatroon van de antenne kan be\"invloed worden door de drone \cite{A1}.
Maar ook het feit dat deze drones boven de gebruikers zullen vliegen zorgt er voor dat 2D modellering onvoldoende is.
Een 3D model waarbij rekening gehouden wordt met zowel horizontale als verticale richting zal een vereiste vormen \cite{U12}.

Het eenvoudigste stralingspatroon is een hypothetische isotrope antenne die straalt met gelijke hoeveelheid in elke richting.
Een antenne die gelijkwaardig straalt over een specifiek vlak wordt een omnidirectionele antenne genoemd 
\cite{U12}. Hiervan bestaan verschillende soorten voor te bevestigen op drones zoals monopoolantennes, dipoolantennes en vleugel antennes \cite{A4,A10,A11,A12}.
Een andere vorm van antenne zijn directionele antennes die energie besparen door de elektromagnetische straling 
te focussen daar waar het nodig is. Eén soort hiervan die uitgebreid onderzocht is in verscheidene antenne-arrays 
zijn de microstrip antennes \cite{A5,A6,A8}.
Deze bieden verschillende voordelen ten opzichte van meer traditionele antennes 
zoals het beperkte gewicht, lage productiekosten en aerodynamica \cite{J13_microstripadvantages,J14_antennadesign}.

Een microstrip antenne is opgebouwd uit een grondplaat en een stralingsplaat met daartussen een di\"electrisch substraat.
Verscheidene variaties bestaan zoals microstrip patch antennes, microstrip slot antennesengeprinte dipool antennes 
die allemaal gelijkende karakteristieken hebben \cite{J13_microstripadvantages,J14_antennadesign}. 
Ze zijn allemaal dun, ondersteunen dubbele frequentiesenhebben allemaal het nadeel dat 
ze interferentie kunnen veroorzaken op frequenties buiten het bedoelde spectrum. 
De microstrip patch en slot antenne ondersteunen beiden circulaire en lineaire polarisatie terwijl de geprinte dipool antenne enkel 
lineaire polarisatie ondersteunt. Verder is de microstrip patch antenne het eenvoudigste te produceren ten opzichte 
van de andere overwogen antennes \cite{J13_microstripadvantages}. Een voorbeeld wordt gegeven in fig. \ref{fig:exampleDrone}.

\begin{figure}[h]
\centering
  \includegraphics[width=0.8\linewidth]{drone.png}
  \caption{Afbeelding van een microstrip patch antenne bevestigd aan de onderkant van een drone. }
  \label{fig:exampleDrone}
\end{figure}
\vspace{10 mm}
Deze foto toont een microstrip patch antenne die bestaat 
uit aluminium en bevestigd is op een substraat van Teflon. De microstrip patch antenne 
wijst naar de grond aangezien de drone boven de mensen zal vliegen.

\section{Methodologie}

De eerste subsectie legt uit hoe elektromagnetische straling berekend kan worden voor elke bron 
en hoe deze om te zetten naar \gls{SAR}.
De tweede subsectie geeft een overzicht van hoe een microstrip patch antenne ontwikkeld kan worden. 
De derde subsectie bespreekt hoe het network geoptimaliseerd kan worden en als laatste wordt het algoritme
uitgelegd.

\subsection{Elektromagnetische Straling}
\subsubsection{Totale elektromagnetische Straling}
De totale \gls{SAR} voor het volledige lichaam ($SAT^{wb,totaal}_{10g}$) van een individu 
kan berekend worden als een eenvoudige som van de \gls{SAR}-waarden van de individuele bronnen. 
Dit is gebaseerd op de formule uit \cite{J17_kuehn2019modelling} die aanneemt dat het mobiele apparaat 
tegen het oor van zijn gebruiker gehouden wordt. Hierdoor worden alle waarden in locale \gls{SAR}-waarden voor het hoofd uitgedrukt.
In dit netwerk is de plaats van het mobiele apparaat echter  niet  gekend wat zou leiden tot onjuiste conclusies. Bijgevolg 
zal alles uitgedrukt worden in functie van het volledige lichaam.

\begin{equation} 
\begin{aligned}
SAT^{wb,totaal}_{10g} = SAT^{wb,myUE}_{10g} +  SAT^{wb,myUABS}_{10g} \\
+ SAT^{wb,otherUE}_{10g} + SAT^{wb,otherUABSs}_{10g}
\end{aligned}
\label{eq:overallSARwb}
\end{equation}

In bovenstaande formule staat $wb$ voor whole body ofwel het volledige lichaam en $UE$ voor User Equipment ofwel het mobiele apparaat op de grond.
De eerste parameter, $SAT^{wb,myUE}_{10g}$, duidt de geabsorbeerde elektromagnetische straling aan komende van de gebruiker zijn eigen apparaat.
Ondanks het feit dat de \gls{UL} straling bedoeld is voor de \gls{UABS} die deze gebruiker behandelt,
wordt een deel van deze straling ook geabsorbeerd door de gebruiker zelf.
Dit komt door de omnidirectionele antenne die door het mobiele apparaat gebruikt wordt.
Een tweede parameter is $SAT^{wb,myUABS}_{10g}$ die de straling aanduidt veroorzaakt door \gls{DL} dataverkeer, komende van de \gls{UABS} die deze gebruiker behandelt.
Als derde parameter hebben we $SAT^{wb,otherUE}_{10g}$ die de straling aanduidt veroorzaakt door andere gebruikers hun mobiel apparaat.
Als laatste stelt $SAT^{wb,otherUABSs}_{10g}$ de \gls{DL} straling voor komende van alle \gls{UABS}en die andere gebruikers behandelen.
Een illustratie is te vinden in fig. \ref{fig:netwerkIllustration} waarbij de groene pijl straling in het nabije veld voorstelt en alle andere 
pijlen straling in het verre veld voorstellen.

\begin{figure}[h!]
\centering
  \includegraphics[width=\linewidth]{networkIllustrationSARSources.png}
  \caption{Deze illustratie toont hoe de gemiddelde gebruiker (hier getoond in het midden) be\"invloed wordt door verschillende bronnen van elektromagnetische straling.}
  \label{fig:netwerkIllustration}
\end{figure}

\subsubsection{elektromagnetische straling van een indivuele bron}
\label{sec:calculatingexposure}

Om de totale elektromagnetische straling te vinden waaraan de gebruiker is blootgesteld, dient eerst
de straling van elke individuele bron berekend te worden.
Dit wordt gedaan met formule  \ref{eq:singleexposure} en is van toepassing voor alle bronnen in het verre veld.
Dit houdt in: alle  \gls{UABS}'s en alle mobiele apparaten die niet tot de gebruiker behoren.
De elektromagnetische veldsterkte $E$ voor het individu $u$ komende van een bron $i$ wordt berekend 
met formule \ref{eq:singleexposure}. 


\begin{equation}
E_i(u) [V/m] = 10^{\frac{ES(u)[dBm] - 43.15 + 20*\log(f [MHz])- PL(u) [dB]}{20}}
\label{eq:singleexposure}
\end{equation}

Het berekenen van de effectieve straling (ES) voor een gebruiker $u$ vereist eerst om de  \gls{EIRP} te hebben berekend 
 \cite{J6_originalExposureFormula,J1}. Dit kan bekomen worden door het zendvermogen $P_t$ op te tellen met de zendversterking $G_t$
 en het kabelverlies $L_t$ ervan af te trekken.
 Deze formule dient echter uitgebreid te worden zodoende dat er rekening gehouden wordt met signaalverzwakking wat afhankelijk is van 
 het gebruikte stralingspatronen. Deze waarde hangt af van de hoek tussen de gebruiker en de richting waarnaar de antenne wijst. 
 De signaalverzwakking bij een \gls{isotropicradiator} is altijd nul ongeacht de hoek.
 Dit leidt tot de volgende formule:

\begin{equation}
\begin{aligned}
RRP [dBm] = P_t [dBm] + G_t [dBi]- L_t [dB]\\
     - attenuation(u) [dB]
\end{aligned}
\label{eq:eirp}
\end{equation}

De gebruikte frequentie $f$ in formule \ref{eq:singleexposure} is uitgedrukt in MHz. Aangezien 
\gls{LTE} gebruikt wordt, zal deze waarde 2600 MHz bevatten.

Als laatste dient formule \ref{eq:singleexposure} ook het padverlies $PL$ te kennen.
Een geschikt propagatie model dient gekozen te worden. Hier wordt geopteerd voor het 
Walfish-Ikegami model aangezien die goed presteert voor femtocell netwerken in stedelijke gebieden \cite{J2}.

\subsubsection{Samenvoegen van meerdere bronnen}

De totale elektromagnetische straling $E_{tot}$ in een bepaald punt, komende van alle verschillende bronnen, kan berekend worden 
door formule \ref{eq:totalexposure}. Hierin staat $E_i$ voor de elektromagnetische veldsterkte voor dat punt komende van bron $i$
en $n$ staat voor alle bronnen in het verre veld van een bepaalde categorie wat hier ofwel \gls{UABS}'s of mobiele apparaten van andere personen zijn.
$E_{tot}$ zal berekend worden in elk punt waar er zich een gebruiker bevindt.
\begin{equation}
E_{tot} [V/m] = \sqrt{\sum_{i=1}^{n} (E_i [V/m]) ^2}
\label{eq:totalexposure}
\end{equation}

\subsubsection{Omzetten van elektromagnetische veldsterkrte naar \gls{SAR}}

Formule \ref{eq:overallSARwb}  verwacht dat de \gls{SAR} waarden in functie van het volledige lichaam uitgedrukt zijn.
Om de elektromagnetische veldsterkte te kunnen omzetten naar deze \gls{SAR}-waarden dient er een onderscheid gemaakt te worden 
tussen bronnen in het nabije veld ($SAR^{wb,nf}$) en het verre veld ($SAR^{wb,ff}$).
$SAR^{wb,myUE}_{10g}$ is een bron waarbij de gebruiker zich in het nabije veld bevindt terwijl 
de gebruiker zich voor alle andere bronnen in het verre veld bevindt.

Het omzetten van deze waarden gebeurt door middel van een conversie constante die gebaseerd is op 
Duke van de Virtual Family. Duke is een 34 jarige man met een gewicht van 72 kg, een lengte van 1.74 m en 
een BMI van 23.1 kg/m \cite{J22_plets2015joint}. 
Onderzoek toont aan dat de conversie constante voor WiFi in het verre veld $0.0028 \frac{W/kg}{W/m^2}$ bedraagt
en  0.0070 $\frac{W/kg}{W}$ in het nabije veld \cite{J22_plets2015joint}.
WiFi maakt gebruik van het 2400 MHz spectrum wat heel  dicht bij \gls{LTE} is met 2600 MHz. Daarom 
wordt in \cite{J22_plets2015joint} aangenomen dat de conversie  constante ook van toepassing is voor \gls{LTE}.
Het berekenen van \gls{SAR} in het verre veld  wordt als volgt gedaan:

\begin{equation}
S [W/m^2]= \frac{(E_{tot} [V/m])^2}{337}
\label{eq:flux}
\end{equation}
\begin{equation}
SAR^{wb,ff}_{10g} [W/kg]= S [W/m^2]* 0.0028 \left[\frac{W/kg}{W/m^2}\right]
\label{eq:DLconversion}
\end{equation}

De constante in vergelijking \ref{eq:DLconversion} zet de \gls{power flux density} $S$ om  naar de verwachte $SAR^{ff,wb}_{10g}$.
Om dit mogelijk te maken moet 
het resultaat van formule \ref{eq:totalexposure} eerst nog omgezet worden naar \gls{power flux density} met behulp van formule
\ref{eq:flux}.

De \gls{SAR} die veroorzaakt wordt door het mobiel apparaat in het nabije veld kan gevonden worden door 
het zendvermogen $P_{tx}$ van het  apparaat te vermenigvuldigen met de conversie constante voor het nabije veld
en is berekend als volgt:
\begin{equation} 
SAR^{wb,nf}_{10g} [W/kg] = 0.0070 \left[\frac{W/kg}{W}\right] * P_{tx} [W]
\label{eq:ulToSar}
\end{equation}

De energie die door het mobiele apparaat wordt gebruikt kan berekend worden met formule \ref{eq:powerUE} \cite{J22_plets2015joint}.
\begin{multline} 
P_{tx}^{UE} = min \big\{P_{max} [dBm] , \\
 P_{pusch} [dBm] + \alpha * PL [dB] + 10log(M) + \sigma \big\}
\label{eq:powerUE}
\end{multline}


Hierbij staat $P_{max}$ voor het maximaal toegestane zendvermogen van het mobiele apparaat wat voor LTE 23 dBm bedraagt.
Dit is echter in het slechtste geval. De effectieve waarde ligt dankzij power control doorgaans lager.
$P_{pusch}$ is de minimale energie vereist door de 
\gls{UABS} en bedraagt hier -120 dBm. 
$\alpha$ is de compensatiefactor voor het padverlies en is gelijk aan \'e\'en wat hier volledige compensatie betekent \cite{J32,J33}.
Voor het 20 MHz kanaal in deze paper zal $M$ gelijk zijn aan 100 en
 zal $\sigma$,  als correctiefactor, nul bedragen \cite{J22_plets2015joint,J32}.

\subsection{Microstrip Patch Antenne}
Een microstrip patch antenne is gekozen vanwege zijn eenvoudige productieproces maar voornamelijk vanwege het
 lage gewicht en aerodynamica wat heel voordelig is wanneer het aan een drone gekoppeld wordt \cite{J13_microstripadvantages}.

De dimensies van de antenne hangen af van de gebruikte frequentie en de eigenschappen van het di\"electrisch substraat.
De antenne zal opereren met een frequentie $f_0$ van 2.6 GHz. 
Elk substraat heeft een di\"electrische constante $\epsilon_r$ die de doorlaatbaarheid 
van het substraat aanduidt en hangt af van het gebruikte materiaal.
Substraten met een hoge di\"electrische constante en kleine hoogte zullen de dimensies van de antenne reduceren 
terwijl  een lager di\"electrische constante met een hogere hoogte de performantie van de 
antenne zullen bevorderen \cite{J14_antennadesign,J15_antennadesign}. 
Voor dit onderzoek is glas gekozen vanwege zijn hogere di\"electrische constante
 $\epsilon_r = 4.4$ ten opzichte van andere materialen zoals Teflon met een di\"electrische constante
van $\epsilon_r = 2.2$ \cite{J14_antennadesign}. 
Glas met een hoogte van 2.87 mm 
zal de dimensies van de volledige antenne opperlakte verminderen wat 
voordelig is bij de beperkte ruimte die beschikbaar is op een drone.

\begin{table}[h!]
\centering
\begin{tabular}{|l|c|l|}
\hline
 Beschrijving            & Symbool          & Waarde         \\    \hline
 Middenfrequentie      & $f_0$           & 2600 MHz       \\ 
 Di\"electrische constante    & $\epsilon_r$    & 4.4         \\ 
 Hoogte van het substraat & $h$             & 0.00287 m    \\ \hline
\end{tabular}
\caption{Overzicht van de configuratie parameters}
\label{table:antennaparas}
\end{table}

De dimensies van de stralingsplaat kunnen berekend worden met de formules uit \cite{J14_antennadesign,J15_antennadesign}.
Dit leidt tot een stralingsplaat van 35.09 mm bij 26.55 mm en  een grondplaat van minstens 52.40 mm bij 43.80 mm.
De resulterende microstrip patch antenne is ge\"illustreerd in fig. \ref{fig:basicpatchantenna} en zal resulteren 
in het stralingspatroon getekend in fig. \ref{fig:radpattern}.
\vspace{ 6.4 mm}
\begin{figure}[h!]
\centering
  \includegraphics[width=\linewidth]{MicrostripAntenna.png}
  \caption{Schema van de microstrip patch antenne.}
  \label{fig:basicpatchantenna}
\end{figure}

\begin{figure}[!htb]
\subfloat[E-vlak]{\includegraphics[width=0.49\linewidth]{pattern2/ep.png}}
\hfill
\subfloat[H-vlak]{\includegraphics[width=0.49\linewidth]{pattern2/hp.png}}
  \caption{Stralingspatroon gegenereerd door de microstrip patch antenne.}
\label{fig:radpattern}
\end{figure}


\subsection{Optimaliseren van het netwerk}

Deruyck et al. bespreekt in \cite{J1} hoe een traditioneel mobiel netwerk geoptimaliseerd kan worden naar elektromagnetische straling of energieconsumptie.
Hoewel een toenemend zendvermogen wel degelijk resulteert in hogere elektromagnetische veldsterkte is deze regel niet van 
toepassing indien we het energieverbruik bekijken over het hele netwerk heen. 
De auteurs van \cite{J1} tonen een omgekeerd equivalente relatie aan.
De reden hierachter is dat het vaak minder energie kost om de elektromagnetische straling van een reeds actieve base station 
verder te laten toenemen in plaats van  een nieuwe base station te activeren. Dit leidt tot de fitness functie 
in vergelijking \ref{eq:fitnessfunction} en is gebaseerd op \cite{J1}.

\begin{equation} 
f = w * \left(1 - \frac{E_m}{E_{max}}\right) + (1 - w)*\left(1 - \frac{P}{P_{max}}\right) * 100
\label{eq:fitnessfunction}
\end{equation}
\newline
Formule \ref{eq:fitnessfunction} geeft een score terug dat aanduidt hoe goed het netwerk preseteert.
$w$ is de belangrijkheidsfactor die loopt van 0 tot 1, grenzen inbegrepen. Een $w$ gelijk aan 0 betekent
dat elektromagnetische straling niet belangrijk is. Een dergelijk netwerk wordt een \gls{PwrC Opt} netwerk genoemd.
Aan de andere kant, een $w$ gelijk aan 1 impliceert dat het minimaliseren van elektromagnetische blootstelling top prioriteit is
en zal bijgevolg resulteren in een \gls{Exp Opt} netwerk. $P_{max}$ is het energieverbruik van alle \gls{UABS}'s op maximaal 
zendvermogen, ongeacht of ze 
op non-actief staan of niet.
$P$ stelt de effectieve verbruikte energie van het huidig ontwikkelde netwerk voor.
$E_m$ is de elektromagnetische straling van de gewogen gemiddelde gebruiker van het huidig ontwikkelde netwerk en 
$E_{max}$ is dezelfde waarde maar met alle \gls{UABS}'s op maximaal zendvermogen.

Bij het optimaliseren van het netwerk is het niet enkel belangrijk om  de gemiddelde gebruiker te overwegen maar ook het limiteren 
van extrema \cite{J1}. 
Daarom wordt er gebruik gemaakt van het gewogen gemiddelde waarbij niet enkel rekening gehouden wordt met de mediaan maar ook 
met het 95\textsuperscript{ste} percentiel. Dit leidt tot formule \ref{eq:em} waarbij 
  $w_1$ en  $w_2$ de gewichten zijn van respectievelijk de mediaan en het 95\textsuperscript{ste} percentiel.
 Aangezien verondersteld wordt dat beide waarden een gelijkwaardige rol spelen zullen beiden een gewicht van 0.5 krijgen. 

\begin{equation} 
E_m = \frac{w_1 * E_{50} + w_2 * E_{95}}{w_1 + w_2}
\label{eq:em}
\end{equation}
\newline
\subsection{Simulatie Tool}

\subsubsection{Hoofdalgoritme}
In eerste instantie dient een beschrijving van het gebied voorzien te worden. Dit wordt verwezenlijkt met behulp van 
zogenaamde shape-bestanden. Deze bestanden bevatten een volledige beschrijving van de vorm van elk gebouw. Vervolgens 
worden gebruikers uniform verdeeld over het gebied en zal er een tijdelijke \gls{UABS} geplaatst worden boven elke gebruiker.
Nu is het aan het beslissingsalgoritme om te bepalen welke \gls{UABS}'s effectief zullen blijven en hoe hoog het zendvermogen van elke \gls{UABS}
zal zijn. Eens het beslissingsalgoritme voltooid is zal de tool controleren of het nummer van online drones niet meer is dan 
de capaciteit van de stockageruimte toelaat. Indien dit wel het geval is zullen drones offline gehaald worden, beginnend bij 
drones die het minste personen behandelen.

\subsubsection{Beslissingsalgoritme}

Het oplossen van het netwerk is de verantwoordelijkheid van het beslissingsalgoritme en start met het berekenen van het padverlies tussen 
alle gebruikers en tussen gebruikers en drones. Hierna doorloopt het algoritme elke gebruiker waarbij getracht wordt deze te verbinden 
met elke mogelijke \gls{UABS}. Deze verbinding is niet altijd mogelijk omdat een \gls{UABS} al reeds verzadigd kan zijn met andere gebruikers of 
de \gls{UABS} is zo ver verwijderd van deze gebruiker dat de \gls{UABS} de maximale toegestane zendvermogen zou overschreiden.
Indien een verbinding toch mogelijk is, zal de gebruiker met deze \gls{UABS} verbonden worden en zal een score toegekend worden met behulp van 
de fitness functie uit vergelijking \ref{eq:fitnessfunction}. 
Dit proces wordt herhaald voor elke \gls{UABS}. Uitsluitend de verbinding die resulteert in de beste score voor het volledige netwerk 
zal gebruikt worden. 
Op deze mannier zal elke gebruiker de beste oplossing krijgen vanuit de huidige toestand van het netwerk.
Met andere woorden, elke gebruiker wordt geoptimaliseerd en niet het netwerk zelf. Er wordt echter wel aangenomen dat 
op deze manier het gemiddelde netwerk zelf ook optimaal zal zijn.
Wanneer de laatste gebruiker behandeld is geweest, bekomen we een volledig netwerk voor een ongelimiteerd aantal drones.
Het netwerk wordt vervolgens terug aan het hoofdalgoritme gegeven voor eventuele verdere afhandeling.
Een stroomdiagram van dit algoritme is gegeven in fig. \ref{fig:decisionAlgoFlowChart}.

\begin{figure}[h!]
\centering
  \includegraphics[height=0.92\textheight]{decisionAlgoFlowChart.png}
  \caption{Flowchart of the decision algorithm.}
  \label{fig:decisionAlgoFlowChart}
\end{figure}
%%%%%%%%%%%%%%%%%%%%%%%%%%%%%%%%%%%%%%%%%%%%%%%%%%%%%%%%%%%%%%%%%%%%%%%%%%%%%%%%%%%%%%%%%%%%%%%%%%%%%%%%%%%%%%%%%%%

\section{Scenario's}

De standaard configuratie is gegeven in tabel \ref{table:defaultconf} en is van toepassing 
in elk scenario tenzij anders vermeld door de restricties van dat specifieke scenario.

\begin{table}[!htb]
\centering
\begin{tabular}[t]{ll}
        \toprule
        \multicolumn{2}{l}{\textbf{Mobiel netwerk}} \\
        \hline
        \hspace{3mm}  Technologie        & LTE     \\
        \hspace{3mm}  Frequentie         & 2.6 GHz \\
        \hspace{3mm}  Power offset ($P_{pusch}$)            & -120 dBm  \\
        \hspace{3mm}  Compensatie voor padverlies ($\alpha$)   & 1  \\
        \hspace{3mm}  Correctie waarde                    & 0 dBm  \\
        \hspace{3mm}  Aantal resource blokken      & 100  \\
        \hline
        \multicolumn{2}{l}{\textbf{Drone}} \\
        \hline  
        \hspace{3mm}  Energie van de drone        & 13.0 A   \\
        \hspace{3mm}  Gemiddelde snelheid        & 12.0 m/s \\
        \hspace{3mm}  Gemiddeld energieverbruik      & 17.33 Ah    \\
        \hspace{3mm}  Voltage batterij       & 22.2 V \\
        \hline
        \multicolumn{2}{l}{\textbf{Femtocell antenna}} \\
        \hline  
        \hspace{3mm}  Maximum $P_{tx}$          & 33 dBm   \\
        \hspace{3mm}  Richting van de antenne   & neerwaarts   \\ 
        \hspace{3mm}  Zendversterking           & 4 dBm   \\ 
        \hspace{3mm}  Kabelverlies               & 2 dBm   \\ 
        \hspace{3mm}  Implementatieverlies       & 0 dBm   \\
        \hspace{3mm}  Stralingspatroon         & EIRP or\\
         \hspace{3mm}                           & microstrip patch\\
        \hspace{3mm}  Vlieghoogte                & 100m  \\
        \hline
        \multicolumn{2}{l}{\textbf{UE Antenna}} \\
        \hline 
        \hspace{3mm} Hoogte                     & 1.5m vanaf de vloer      \\ 
        \hspace{3mm} Zendversterking                      & 0 dBm   \\ 
        \hspace{3mm} Kabelverlies              & 0 dBm   \\ 
        \hspace{3mm} Stralingspatroon         & EIRP  \\
        \hspace{3mm} Aantal aanwezig in het netwerk         & 224  \\
        \toprule
\end{tabular}
\caption{Overzicht van de waarden voor een standaard configuratie.}
\label{table:defaultconf}
\end{table}

Drie scenario's zullen onderzocht worden. Het eerste zal \'e\'en enkele gebruiker en 
 \'e\'en enkele drone overwegen 
voor het gehele netwerk. De \gls{SAR}, elektromagnetische straling, energieverbruik  en 
het nodige zendvermogen van de antenne zullen onderzocht worden voor verschillende vlieghoogtes.

Bij het tweede scenario zal het netwerk uitgebreid worden met meerdere gebruikers maar 
er zal nog steeds uitsluitend  \'e\'en drone aanwezig zijn. De eerste onderzochte parameter 
is een varierende vlieghoogte gaande van 20 meter tot 200 meter. Hierbij zullen 224 gebruikers 
uniform verdeeld worden over het centrum van Gent. Dit is de gemiddelde populatiegrootte op 
een werkdag om 17 uur in Gent \cite{J2}.
De tweede parameter is een varierende populatie lopend van 50 tot 600 gebruikers. Hierbij 
zal de vlieghoogte vastgezet worden op 100 meter \cite{J2}.
Het energieverbruik, elektromagnetische straling en \gls{SAR} zullen onderzocht worden.

Een derde scenario is sterk gelijkend aan het vorige. Dezelfde twee parameters zullen onderzocht 
worden maar nu voor een onbeperkt aantal \gls{UABS}'s.

Vier configuraties zijn mogelijk voor elke onderzochte parameter in elk scenario.
Er zijn namelijk twee antennes, een \gls{isotropicradiator} en een microstrip patch antenne die
beide kunnen opereren in een \gls{PwrC Opt} netwerk en een \gls{Exp Opt} netwerk. Dit maakt een totaal van 4 configuraties.
Een overzicht is gegeven in fig. \ref{fig:fourCasesMatrix}.

Het is belangrijk om op te merken dat alle meetwaarden strikt gelimiteerd zijn tot de hiervoor vernoemde bronnen en 
bijgevolg enkel dataverkeer overwegen tussen de gebruiker zijn apparaat en de \gls{UABS}. 
Andere bronnen zoals connecties naar het backhaul netwerk of andere technologie\"en zullen niet overwogen worden.

\begin{figure}[h!]
\centering
  \includegraphics[width=0.9\linewidth]{fourCasesMatrix.png}
  \caption{Matrix met de vier mogelijke configuraties.}
  \label{fig:fourCasesMatrix}
\end{figure}

\section{Resultaten}
Vier configuraties zullen overwogen worden tijdens het evalueren van twee parameters, zijnde 
populatiegrootte en vlieghoogte. Deze parameters zullen onderzocht worden in drie scenario's 
door het gedrag van het energieverbruik, elektromagnetische straling en \gls{SAR}-waarden te monitoren.
De elektromagnetische straling en \gls{SAR} zullen genomen worden van de gewogen gemiddelde gebruiker
met behulp van vergelijking \ref{eq:em} waarbij $w_{1}$ en $w_{2}$ gelijk gesteld zijn aan 50\%. 
Elk resultaat wordt uitgemiddeld over 20 simulaties.


\subsection{E\'en gebruiker en \'e\'en \gls{UABS}}
Fig. \ref{fig:ptx} toont aan dat voor een variabele vlieghoogte, een logaritmische relatie bestaat tussen de 
 $P_{tx}$ en de vlieghoogte.
 Dit komt door de logaritmische schaal waarin de decibels van de $P_{tx}$ in zijn uitgedrukt.
Elke keer dat de vlieghoogte te hoog wordt, neemt de $P_{tx}$ met \'e\'en dBm toe.
Voor een standaard configuratie met een maximum $P_{tx}$ van 33 dBm en een \gls{LOS} verbinding kan een 
\gls{UABS} tot $387\ m$ hoogte vliegen zonder het verliezen van deze verbinding.

\begin{figure}[h!]
\centering
  \includegraphics[width=\linewidth]{s1/ptx.png}
  \caption{Het vereiste zendvermogen om de gebruiker op de grond te bereiken. Gemeten op verschillende vlieghoogtes.}
  \label{fig:ptx}
\end{figure}

Dit scenario is onderzocht voor een microstrip patch antenne die energieverbruik minimaliseert.
De gekozen optimalisatie maakt echter niet uit aangezien er uitsluitend \'e\'en \gls{UABS} beschikbaar is.
Het beslissingsalgoritme bepaalt welke gebruiker met welke \gls{UABS} verbonden wordt.
Aangezien er maar een \gls{UABS} beschikbaar is, zullen beide optimalisatie technieken gelijkaardig werken.
Verder zal de gebruikte antenne ook geen verschil maken. 
De gebruiker zal zich namelijk boor beide antennes in de hoofdstraal van het stralingspatroon 
bevinden waar voor beide antennes geen verzwakking van het signaal is.

Tijdens het onderzoeken van verschillende vlieghoogtes stellen de resultaten vast  dat
\gls{UL} straling exponentieel toeneemt terwijl de \gls{DL} straling constant blijft rond 
10 $nW/kg$ zoals te zien is in fig. \ref{fig:s1_fhsar}. De reden dat de \gls{DL} straling constant blijft is vanwege power control die ervoor zorgt
dat niet meer energie gebruikt wordt dan strikt noodzakelijk. 
Daardoor kan bevestigd worden dat de elektromagnetische straling een constante fractie is van energie en afstand.
De \gls{UL} straling start laag met 1 $nW/kg$ maar steekt de \gls{DL} straling voorbij rond de 80 meter.

\begin{figure}[h!]
\centering
  \includegraphics[width=\linewidth]{s1/fhvssar.png}
  \caption{
    Deze figuur toont hoe SAT-waarden van verschillende bronnen be\"invloed worden door een variabele vlieghoogte.}
  \label{fig:s1_fhsar}
\end{figure}

\FloatBarrier
\subsection{Toenemende populatie met \'e\'en UABS}
\subsubsection{Variabel vlieghoogte}
Een \gls{PwrC Opt} heeft een hogere elektromagnetische blootstelling in vergelijking met een
 \gls{Exp Opt} netwerk; een fenomeen dat reeds werd vastgesteld bij \cite{J1}. 
Uit de resultaten van dit scenario blijkt echter dat een
\gls{PwrC Opt} niet noodzakelijk resulteert in een lager energieverbruik.
Zo blijkt dat bij 100 m in een  \gls{EIRP} \gls{Exp Opt} netwerk
 de elektromagnetische straling van de gewogen gemiddelde gebruiker
 $1.5\ mV/m$ minder zal zijn maar dat het energieverbruik met $20\ mW$ zal toenemen.
Om dit te verstaan dient het algoritme eerst uitgelegd te worden.
Een  \gls{PwrC Opt} netwerk zal resulteren in enkele \gls{UABS}'s met een hoog energieverbruik 
omdat het toenemen van de $P_{tx}$ van de antenne minder energie kost dan het activeren van een nieuwe \gls{UABS}.
Op dezelfde manier zal een \gls{Exp Opt} netwerk meer \gls{UABS}'s gebruiken met een laag energie verbruik waardoor ook de elektromagnetische straling minder zal zijn.
Wanneer slechts een beperkt aantal \gls{UABS}'s beschikbaar is, zoals maar \'e\'en in dit netwerk, 
zullen enkel de \gls{UABS}'s gebruikt worden die de meeste mensen behandelen.
Aangezien het energieverbruik van een individuele \gls{UABS} hoger is in een \gls{PwrC Opt} network en er uitsluitend 
\'e\'en \gls{UABS} beschikbaar is in elke configuratie, zal 
 het energieverbruik in een \gls{PwrC Opt} netwerk vaak hoger liggen.

\begin{figure}[h!]
  \includegraphics[width=\linewidth]{s2/fhvsdlAndPc.png}
  \caption{Fig. (a) toont hoe de vlieghoogte be\"invloed wordt door \acs{DL} elektromagnetische straling van de 
  gewogen gemiddelde gebruiker en fig. (b) toont het En energieverbruik van het volledige netwerk wanneer een enkele drone beschikbaar is.}
  \label{fig:s2a_dlAndPc}
\end{figure}

Verder toont fig.  \ref{fig:s2a_dlAndPc} hoe de elektromagnetische blootstelling toeneemt wanneer de vlieghoogte hoger wordt. 
Dit komt omdat de waarschijnlijkheid op een \gls{NLOS} afneemt.
Dit leidt eveneens ook tot een hogere dekkingsgraad.
Wanneer de vlieghoogte toeneemt van 20 tot 100 m zal de dekkingsgraad met 1\% tot 2\% voor alle configuraties toenemen.
Deze toename in elektromagnetische straling is echter niet ongelimiteerd. Een microstrip \gls{PwrC Opt}
is op zijn hoogste punt rond 162 m en een \gls{EIRP} \gls{PwrC Opt} is op zijn hoogste punt rond 195 m.
De wederafname van elektromagnetische straling start later voor \gls{Exp Opt} netwerken 
en bevindt zich buiten de onderzochte vlieghoogtes. 
Deze wederafname is niet veroorzaakt door gebouwen maar door de grotere afstand in het algemeen.

Fig. \ref{fig:s2shfourSourcesMatrix} toont de $SAR^{wb}_{10g}$ van de gewogen gemiddelde gebruiker voor elke individuele bron.
De resultaten stellen vast dat de $SAR^{myUABS}$ dezelfde curve toont als deze van de elektromagnetische straling in fig. \ref{fig:s2a_dlAndPc}.a. Dit komt omdat vergelijking \ref{eq:DLconversion} de elektromagnetische straling converteert naar \gls{SAR}
door het te vermenigvuldigen met een constante. Gedurende de gehele tijd is de $SAR^{myUABS}$
de meest dominante factor gevolgd door de straling van de gebruiker zijn eigen mobiel apparaat.
Straling komende van andere personen hun mobiel apparaat heeft amper invloed.
Als voorbeeld, bij een vlieghoogte van 140 m en een \gls{EIRP} \gls{PwrC Opt} netwerk
zal de gewogen gemiddelde gebruiker een \gls{SAR} van
2.1 $nW/kg$ ondervinden van de  \gls{UABS} en rond
 0.2 $nW/kg$ van zijn eigen apparaat.
De blootstelling van andere mobiele apparaten kan verwaarloosd worden met een elektromagnetische straling van slecht
 0.03 $pW/kg$. Dit is een lage maar plausibele waarde aangezien de meeste mensen niet gedekt zijn 
en daardoor zelf geen straling veroorzaken.


\begin{figure}[h!]
\centering
  \includegraphics[width=\linewidth]{s2/fhFourSources.png}
  \caption{Elke grafiek komt overeen met een specifieke configuratie en toont hoe de 
     \acs{SAR} van verschillende bronnen be\"invloed wordt door een toenemende vlieghoogte.}
  \label{fig:s2shfourSourcesMatrix}
\end{figure}

\FloatBarrier
\subsubsection{Variabel aantal gebruikers}
Het onderzochte bereik loopt van 50 tot 600 gebruikers.
Het blijkt dat het aantal gedekte gebruikers lineair toeneemt met het aantal personen aanwezig in het netwerk zoals getoond wordt in fig.
\ref{fig:s2uvsnumcovusers}.b. Het toont hoe een \gls{isotropicradiator} in staat is om meer personen te behandelen in vergelijking tot een eenvoudige
 microstrip patch antenne. Eveneens is een energiezuinig netwerk in staat om meer mensen te behandelen dan een netwerk dat elektromagnetische straling minimaliseert.
 Bijvoorbeeld met 600 gebruikers zullen 5 tot 7 personen extra behandeld kunnen worden wanneer 
 een microstrip patch antenne vervangen wordt door een  \gls{isotropicradiator}.
Door van een \gls{Exp Opt} netwerk naar een \gls{PwrC Opt} netwerk te gaan, kan er \'e\'en extra persoon 
behandeld worden.

\begin{figure}[h!]
  \includegraphics[width=\linewidth]{s2/uvsnumdronesAndCov.png}
  \caption{De invloed van de populatiegrootte op de dekkingsgraad.}
  \label{fig:s2uvsnumcovusers}
\end{figure}

Fig. \ref{fig:s2b_dlAndPc}.a geeft de elektromagnetische blootstelling van de gewogen gemiddelde gebruiker weer 
bij verschillende populatiegroottes terwijl 
\ref{fig:s2b_dlAndPc}.b het energieverbruik weergeeft voor al deze populatiegroottes.
Fig. \ref{fig:s2b_dlAndPc}.a  is be\"invloed door fig. \ref{fig:s2uvsnumcovusers}.a. 
De elektromagnetische straling neemt af wanneer minder gebruikers behandeld worden.
Bijvoorbeeld, in een EIRP \gls{PwrC Opt} netwerk met 50 gebruikers heeft 6,75\%
dekking wat overeenkomt met een gewogen gemiddelde blootstelling van 18 $mV/m$. Dit terwijl
 600 gebruikers met een dekking van 2,75\% maar 9 $mV/m$ heeft.
Verder is fig. \ref{fig:s2b_dlAndPc}.b rechtstreeks be\"invloed door fig.  \ref{fig:s2uvsnumcovusers}.b.
Wanneer de \gls{UABS} meer mensen behandelt, neemt de kans op gebruikers met een ietwat slechter padverlies toe.
De  \gls{UABS} zal dit probleem oplossen door het energieverbruik toe te laten nemen.
Een toenemende populatie van 50 naar 600 gebruikers zal het energieverbruik tussen 0,05 en 0,1 $W$ verhogen.
Voor dit scenario is geen duidelijk verschil te merken tussen energieverbruik van de vier verschillende configuraties.

\begin{figure}[h!]
  \includegraphics[width=\linewidth]{../results/s2/uvsdlAndPc.png} %0.85
  \caption{Deze twee figuren tonen hoe de verschillende populatiegroottes invloed hebben op de \acs{DL} elektromagnetische straling (fig. a) en het energieverbruik \mbox{(fig. b)}.}
  \label{fig:s2b_dlAndPc}
\end{figure}

\FloatBarrier
De \gls{SAR} komende van de gebruiker zijn eigen apparaat is gemiddeld nul aangezien de meeste gebruikers niet behandeld worden.
Fig. \ref{fig:uvsulsarcentralUsers} toont de elektromagnetische blootstelling van de gedekte gebruiker die zich onmiddellijk onder de \gls{UABS} bevindt.
Het eerste scenario toont hoe de \gls{SAR} van de gebruiker zijn eigen mobiel apparaat enkel be\"invloed wordt door de vlieghoogte.
Dit wordt ook bevestigd door de resultaten in fig. \ref{fig:uvsulsarcentralUsers} waar een constante 
$SAR^{myUE}$ van 0.15 $\mu W/kg$ gemeten wordt.
De \gls{SAR} van de  \gls{UABS} ondervindt een kleine toename van 0,005  $\mu W/kg$.
\newline
\newline
\begin{figure}[h]
\centering
  \includegraphics[width=\linewidth]{../results/s2/uvsulsarcentralUser.png}
  \caption{SAR-waarden voor de gebruiker die zich onder de \acs{UABS} bevindt.}
  \label{fig:uvsulsarcentralUsers}
\end{figure}

Wanneer de populatie toeneemt, zullen meer gebruikers dichtbij de \gls{UABS} terechtkomen.
De \gls{UABS} zal waarschijnlijk beslissen om deze gebruiker ook te behandelen zoals zichtbaar is in fig. \ref{fig:connectionMap}.
Het is mogelijk dat deze gebruiker een slechter padverlies heeft door gebouwen of een ietwat grotere afstand. Hierdoor zal de
\gls{DL} \gls{SAR} van de gebruiker onder de drone toenemen.
De elektromagnetische straling van andere personen hun mobiel apparaat is heel laag zoals 
reeds vermeld en is daarom apart toegevoegd in fig.  \ref{fig:uvsulsarcentralUsers}.b.
De figuur toont hoe de \gls{SAR} van andere mobiele apparaten toeneemt van nul tot $0,15\ pW/kg$.
\begin{figure}[h]
\subfloat[50 gebruikers]{\includegraphics[width=0.49\linewidth]{../images/connectionsMap50Users.png}}
\hfill
\subfloat[600 gebruikers]{\includegraphics[width=0.49\linewidth]{../images/connectionsMap600Users.png}}
\caption{ Overzicht van welke gebruikers die verbonden zijn met de \acs{UABS}.}
  \label{fig:connectionMap}
\end{figure}

\FloatBarrier
\subsection{Ongelimiteerd aantal \gls{UABS}'s}
\subsubsection{Variabel vlieghoogte}
Hetzelfde scenario als in de voorgaande sectie wordt hier onderzocht. Enkel is er hier een ongelimiteerd aantal 
\gls{UABS}'s beschikbaar. De resultaten bewijzen dat de verschillende optimalisatiestrategie\"en werken zoals bedoeld.
Een \gls{PwrC Opt} netwerk heeft inderdaad een lager energieverbruik maar dit komt ten koste van een hogere elektromagnetische straling.
Aan de andere kant zal een  \gls{Exp Opt} netwerk de elektromagnetische blootstelling reduceren door meer drones te gebruiken waardoor tevens 
het energieverbruik zal toenemen. Deze conclusie werd reeds gemaakt in \cite{J1} en is bevestigd door deze resultaten.
Bijvoorbeeld bij het vergelijken van beide optimalisatiestrategieën zal, voor dezelfde
\gls{isotropicradiator} en dezelfde standaard vlieghoogte, een energiezuinig netwerk 51 $W$ verbruiken
en de gebruikers blootstellen aan $15\ mV/m$.
Terwijl optimaliseren naar elektromagnetische straling de blootstelling zal laten zakken naar 
$11.5\ mV/m$ ten koste van een hoger energieverbruik van $54\ W$.

De elektromagnetische blootstelling in fig. \ref{fig:s3a_dlAndPc} toont een logaritmische toename bij een 
\gls{Exp Opt} netwerk terwijl een \gls{PwrC Opt} netwerk eerder een concaaf verband met de vlieghoogte weergeeft 
waarbij het laagste punt zich op 70 meter bevindt.

Fig. \ref{fig:s3a_numDronesAndCov}.a toont aan dat een optimale dekking van 90\% bereikt wordt bij een lagere vlieghoogte van 40 m.
Hier is echter een nadeel aan verbonden. 
Fig. \ref{fig:s3a_numDronesAndCov}.b 
toont dat het aantal vereiste drones toeneemt wanneer de vlieghoogte lager wordt; 
een vaststelling die reeds gemaakt is in \cite{J2}.
Bijvoorbeeld, een microstrip \gls{Exp Opt} netwerk en een \gls{EIRP} \gls{PwrC Opt} netwerk 
vereisen respectievelijk 84 en 64 drones op een vlieghoogte van 200 m wat respectievelijk toeneemt naar 
211 en 162 drones bij een lagere vlieghoogte van 20 m.


\begin{figure}[h!]
  \includegraphics[width=\linewidth]{../results/s3/fhvsdlAndPc.png}
  \caption{Deze twee figuren tonen hoe de vlieghoogte invloed heeft op de \acs{DL} elektromagnetische straling (fig. a) en het energieverbruik \mbox{(fig. b)}.}
  \label{fig:s3a_dlAndPc}
\end{figure}
\vspace{2mm}
\begin{figure}[h!]
  \includegraphics[width=\linewidth]{../results/s3/fhvsnumdronesAndCov.png}
  \caption{Deze grafiek toont hoeveel drones vereist zijn op verschillende vlieghoogtes terwijl er getracht wordt om 100\% dekking te bereiken.}
  \label{fig:s3a_numDronesAndCov}
\end{figure}

Fig. \ref{fig:s3a_fourSourcesMatrix} toont de bijdrage van elke bron aan de totale \gls{SAR}.
Het eerste  gevolg van de vlieghoogte te laten toenemen van 20 naar 200 m is de 
  \gls{SAR}  van de gebruiker zijn eigen apparaat die toeneemt tussen 89 en 141 $nW/kg$;
een gedrag dat reeds geconstateerd werd in het eerste scenario.
Fig. \ref{fig:s3a_fourSourcesMatrix} toont aan dat de vlieghoogte hoger wordt dan de \gls{NLOS} van de gebouwen 
rond 70 tot 80 meter. Hierna blijft de $SAR^{myUABS}$ min of meer gelijk voor alle configuraties.
Voor een microstrip \gls{PwrC Opt} netwerk is dit rond  $160\ nW/kg$.
Een microstrip \gls{Exp Opt} en \gls{EIRP} \gls{PwrC Opt} netwerk is gemiddeld $98\ nW/kg$ en
de \gls{EIRP} \gls{Exp Opt} netwerk bevindt zich rond $47\ nW/kg$.
Deze hogere vlieghoogtes zullen tevens resulteren in een toegenomen elektromagnetische straling van andere
\gls{UABS}'s.
Wanneer de vlieghoogte toeneemt van 20 tot 200 m zal de $SAR^{otherUABS}$ tussen 115 en 140
$nW/kg$ voor \gls{EIRP} antennes bedragen en tussen 54 en 74 $nW/kg$ voor microstrip patch antennes.
Dit voor beide optimalisatiestrategie\"en.


\begin{figure}[h!]
  \includegraphics[width=\linewidth]{../results/s3/fhFourSources.png}
  \caption{Elke grafiek komt overeen met \'e\'en van de vier mogelijke configuraties.
   De bijdrage van elke bron aan de totale \acs{SAR} is voor een variërende vlieghoogte.}
  \label{fig:s3a_fourSourcesMatrix}
\end{figure}

\FloatBarrier
\subsubsection{Variabel aantal gebruikers}
De tweede onderzochte parameter van dit scenario is een vari\"erende populatiegrootte
terwijl de vlieghoogte zich op 100 m zal bevinden.
Fig.  \ref{fig:s3b_numdronesAndCov}.a toont hoe de tool tracht 100\% dekking te bereiken.
Voor slechts 50 personen zal de gemiddelde dekking
 rond 93\% bevinden terwijl een netwerk van 600 personen een dekking van 97\% heeft.
Fig. \ref{fig:s3b_numdronesAndCov}.b toont aan dat meer \gls{UABS}'s  vereist zijn voor grotere populaties.
Het verschil in optimalisatiestrategie is miniem voor kleine netwerken maar neemt snel toe. 
Wanneer de populatie groeit van 50 tot 600 gebruikers zullen 200 \gls{UABS}'s extra vereist zijn bij een
 microstrip \gls{Exp Opt} netwerk,
 rond 130 extra \gls{UABS}'s voor een \gls{EIRP} \gls{Exp Opt} netwerk of een microstrip \gls{PwrC Opt} netwerk
 en 110 extra \gls{UABS}'s voor een \gls{EIRP} \gls{PwrC Opt} netwerk.
Dit is een verwacht gedrag wanneer er gekeken wordt naar Scenario II,
waarbij het percentage van behandelde gebruikers afnam voor grotere populaties.



\begin{figure}[h]
  \includegraphics[width=\linewidth]{../results/s3/uvsnumdronesAndCov.png}
  \caption{Deze grafiek toont hoeveel drones vereist zijn op verschillende vlieghoogtes terwijl er getracht wordt een een dekking van 100\% te hebben.}
  \label{fig:s3b_numdronesAndCov}
\end{figure}

Fig. \ref{fig:s3b_dlAndPC} toont aan dat de elektromagnetische straling en het energieverbruik toenemen voor grotere populaties wat normaal is aangezien meer \gls{UABS}'s
gebruikt worden. Wanneer de populatie toeneemt van 50 naar 600 gebruikers zal 
de elektromagnetische straling toenemen tussen 80 en 130 $mV/m$, afhankelijk van de configuratie. 
Het energieverbruik bij 50 gebruikers is voor alle configuraties rond 20 $W$.
Eenmaal de populatie is toegenomen naar 600 gebruikers zal dit voor een
microstrip \gls{Exp Opt} netwerk 130 $W$ bedragen, 115 $W$ voor een microstrip \gls{PwrC Opt} netwerk,
102 $W$ voor een \gls{EIRP} \gls{Exp Opt} netwerk en 92 $W$ voor een  \gls{EIRP} \gls{PwrC Opt} netwerk.

Dat het beslissingsalgoritme werkt zoals bedoeld, werd reeds duidelijk in de voorgaande subsectie maar wordt
ook hier bevestigd. Wanneer beide optimalisatiestrategieën vergeleken worden,
blijkt dat een energiezuinig netwerk ongeveer 5 $W$ minder energie nodig heeft maar hierdoor de gebruikers wel 
blootstelt aan 27 $mV/m$ tot 30 $mV/m$ meer ten opzichte van \gls{Exp Opt} netwerk.
Verder zal een \gls{isotropicradiator} ook meer elektromagnetische straling veroorzaken 
voor minder energie in vergelijking met een microstrip patch antenne.
Wanneer beide antennes vergeleken worden voor 224 gebruikers blijkt 
dat de \gls{isotropicradiator} de gemiddelde gebruiker tussen 
 25 $mV/m$ tot 27 $mV/m$ extra zal blootstellen
 terwijl het gemiddeld 12 $W$ minder zal nodig hebben in vergelijking met de microstrip patch antenne.


\begin{figure}[h!]
  \includegraphics[width=\linewidth]{../results/s3/uvsdlAndPc.png}
  \caption{ De invloed van de populatiegrootte op de \acs{DL} elektromagnetische straling (a) en energieverbruik (b).
    }
  \label{fig:s3b_dlAndPC}
%\bigbreak
 % \includegraphics[width=\linewidth]{../results/s3/u_numdronesvsdlAndPc.png}
  %\caption{The influence of the number of UABSs on the downlink electromagnetic radiation (a)enpower consumption (b).}
  %\label{fig:s3b_dlAndPC2}
\end{figure}

Fig. \ref{fig:s3b_fourSourcesMatrix} stelt de  
\gls{SAR} van de gewogen gemiddelde gebruiker voor en toont aan hoe de 
\gls{SAR} van de gebruiker zijn eigen mobiele apparaat zo goed als constant is.
De vlieghoogte is namelijk altijd dezelfde waardoor ook de energie die nodig is om de afstand te overbruggen gelijk blijft.
Voor beide optimalisatiestrategie\"en zal de
 $SAR^{myUE}$ voor netwerken met  \gls{isotropicradiator}s vari\"eren rond $1.1\ \mu W/kg$  %between 1 $\mu W/kg$en1.2 $\mu W/kg$ 
 en rond 0.7 $\mu W/kg$ voor netwerken met een microstrip patch antenne.
De $SAR^{myUABS}$ neemt nauwelijks toe in een \gls{Exp Opt} netwerk en bevindt zich rond 0.5 $\mu W/kg$ voor beide antennes.
Een energiezuinig netwerk start ook rond
 0.5 $\mu W/kg$ maar neemt toe wanneer meer mensen online komen.
Dit is normaal aangezien deze \gls{UABS}'s trachten om meer mensen te behandelen.
Hierdoor zal de $SAR^{myUABS}$ 
voor 600 gebruikers toenemen tot 1 $\mu W/kg$ voor een \gls{isotropicradiator} en tot wel 2 $\mu W/kg$ voor een microstrip patch antenne.
De \gls{SAR}-waarde neemt het meeste toe bij $SAR^{otherUABS}$ die heel laag start rond minder  dan  
0.1 $\mu W/kg$ voor 50 gebruikers in alle configuraties. 
Deze \gls{SAR} neemt echter snel toe. De grootste toename wordt waargenomen in een 
\gls{EIRP} \gls{PwrC Opt} netwerk 
waarbij 3 $\mu W/kg$ gemeten wordt voor 600 gebruikers. De $SAR^{otherUE}$ neemt het minste toe in een microstrip \gls{Exp Opt} met 
slechts 1 $\mu W/kg$ voor 600 gebruikers.
\begin{figure}[h!]
\centering
  \includegraphics[width=0.9\linewidth]{../results/s3/uFourSources.png}
  \caption{Elke grafiek komt overeen met een specifieke configuratie en toont aan hoe de \acs{SAR} 
  van verschillende bronnen be\"invloed wordt door een toenemende populatie. %Een ongelimiteerd aantal \acs{UABS}s is beschikbaar.
}
  \label{fig:s3b_fourSourcesMatrix}
\end{figure}

\FloatBarrier
\section{Conclusie}

Een capacity-based deployment tool is gebruikt voor het onderzoeken van de 
\gls{SAR} van gebruikers die geoptimaliseerd zijn naar 
 \gls{DL} elektromagnetische straling en totaal energieverbruik.
 Dit is onderzocht voor verschillende vlieghoogtes, populatiegroottes en aantal beschikbare \gls{UABS}'s.
De resultaten bevestigen de vaststellingen uit \cite{J1}  waarbij elektromagnetische straling en totaal energieverbruik
leiden tot tegenstrijdige vereisten. De voorgestelde fitness functie werkt zoals bedoeld.
Voor een netwerk met standaard configuratie kan de elektromagnetische straling in een 
energiezuinig netwerk kan gereduceerd worden tot wel 
23\% voor \gls{isotropicradiator}s en 30\% voor microstrip patch antennes
door te optimaliseren naar elektromagnetische straling. 
Hierdoor zal het bereik van de 
 \gls{UABS} afnemen aangezien het energieverbruik van een individuele \gls{UABS} zakt tussen de 
 0.07 en 0.12 $W$. Hierdoor zullen 
gemiddeld 18 extra drones nodig zijn die in totaal tot 4 $W$  extra energieconsumptie zullen leiden.

%	How does the netwerk behave differently after the introduction of a realistic antenna?
Een directionele microstrip patch antenne wordt geïntroduceerd omdat het verschillende voordelen biedt 
in vergelijking met een omnidirectionele antenne. Directionele antennes zijn in staat om hun energie te richten waar het nodig is, namelijk de grond.
Een microstrip patch antenne is verder dun en heeft een licht gewicht. Deze antenne, met een openingshoek van \ang{90}, veroorzaakt 
minder elektromagnetische blootstelling, een lager bereik en vereist meer energie.
Voor een netwerk met standaard configuratie kan de microstrip patch antenne de elektromagnetische
straling van een \gls{isotropicradiator} tussen 30\% en 34\% reduceren. Dit zal het energieverbruik met 12 $W$ doen toenemen.
De vereiste energie per \gls{UABS} zal  toenemen met $0.022\ W$ voor een \gls{PwrC Opt} netwerk en met 
$0.007\ W$ voor een \gls{Exp Opt} netwerk.
\begin{figure}[hb!]
\centering
  \includegraphics[width=\linewidth]{../images/fourCasesMatrixSol.png}
  \caption{Matrix met de vier mogelijke configuraties. Gekleurd op basis van de resultaten.}
  \label{fig:resultIllustration}
\end{figure}

Figuur \ref{fig:resultIllustration} toont een overzicht gebaseerd op de resultaten van de twee optimalisatiestrategieën en de twee soorten antennes.
Opmerkelijk is dat de \gls{EIRP} \gls{Exp Opt} netwerk zich gelijkaardig gedraagt als een microstrip patch antenne in een \gls{PwrC Opt} netwerk.
Hierdoor wordt een microstrip patch antenne in een energiezuinig netwerk aangeraden. De microstrip patch antenne zal minder elektromagnetische straling veroorzaken dankzij het design.
Verder zal het optimaliseren naar energieverbruik ervoor zorgen dat minder drones nodig zullen zijn.
Een microstrip patch antenne met een openingshoek van \ang{90} is verondersteld een goede oplossing te zijn
 maar als het budget beperkt is, kan een antenne met een grotere openingshoek een geschikt alternatief vormen
 om kosten verder te reduceren zonder de limieten opgelegd door de Vlaamse overheid te overschreiden.
De \gls{SAR} van de configuratie met de meeste blootstelling is nog steeds een honderdduizendste van de maximale toegestane \gls{SAR} voor het volledige lichaam  ($0.08\ W/kg$).

%What is the contribution of each source towards the total electromagnetic exposure?
Figuur \ref{fig:pie} geeft een overzicht van de \gls{SAR}-bijdrage in percentages ten opzichte van de totale \gls{SAR}.
De waarden zijn uitgemiddeld over de vier overwogen configuraties.
De gebruiker wordt voornamelijk blootgesteld aan zijn eigen mobiel apparaat met 
52\% van de totale elektromagnetische blootstelling.
Deze conclusie werd reeds gemaakt door de auteurs van 
\cite{J17_kuehn2019modelling} en  \cite{J10.1.1}.
Verder wordt er in \cite{J10.1.1} ook geconcludeerd dat, dankzij power control, 
de elektromagnetische straling van het mobiele apparaat heel dicht komt bij de blootstelling van de \gls{UABS}.
Dit wordt ook bevestigd door deze resultaten. Figuur \ref{fig:pie} toont aan dat de elektromagnetische straling van 
alle andere \gls{UABS}'s gezamenlijk de 48\% omvat waarvan 15\% komt van de \gls{UABS} die deze gebruiker aan het behandelen is.
De elektromagnetische blootstelling van mobiele apparaten die tot andere mensen behoren 
is verwaarloosbaar in tegenstelling tot de veel grotere blootstelling komende van alle andere bronnen
en draagt slechts 0.0001\% bij aan de totale blootstelling.

\begin{figure}[hb!]
\centering
  \includegraphics[width=\linewidth]{pie.png}
\caption{
  Bijdrage van elke bron aan de totale SAR waaraan de gemiddelde gebruiker blootgesteld is. 
  De percentages zijn uitgemiddeld over de vier overwogen configuraties.}
\label{fig:pie}
\end{figure}

%How does the \gls{UABS} flying height and number of users influence electromagnetic exposure and power consumption?
De resultaten tonen verder aan dat energieverbruik en elektromagnetische straling toenemen wanneer 
meer mensen aanwezig zijn in het netwerk. Wanneer de populatie toeneemt van 50 naar 600 gebruikers 
zal de elektromagnetische straling tussen 80 en 130 $mV/m$ toenemen afhankelijk van de configuratie.
Het energieverbruik neemt toe met 110 $W$ voor alle configuraties.
De bron die het meest be\"invloed wordt  door het aantal gebruikers is de \gls{SAR} van andere \gls{UABS}'s 
en neemt toe tussen de 1en3 $\mu W/kg$. 
Verder heeft de vlieghoogte een positief effect op het aantal nodige drones die op hun beurt een positief 
effect hebben op het energieverbruik. Wanneer de vlieghoogte toeneemt van 20 m naar 200 m, neemt het aantal 
drones af met 59\%. Deze afname werd ook vastgesteld in \cite{J2}.
Ook de auteurs van \cite{J17_kuehn2019modelling} concludeerden dat elektromagnetische straling afneemt wanneer 
het padverlies minder wordt.
De elektromagnetische straling van de  \gls{UABS}'s blijven min of meer gelijk voor alle vlieghoogtes tussen 80 en 200 meter. 
Meeste \gls{UABS}'s zijn in \gls{LOS} en dankzij power control zullen deze niet meer energie verbruiken dan strikt noodzakelijk.
De elektromagnetische straling van de gebruiker zijn eigen apparaten neemt echter wel toe om de hoog vliegende drones te kunnen bereiken.
Rond 80 meter zal de straling van de gebruiker zijn eigen apparaat de straling van de behandelde \gls{UABS} voorbijsteken.
Wanneer meerdere \gls{UABS} beschikbaar zijn in het netwerk, zal de blootstelling van andere \gls{UABS}'s toenemen 
naarmate de vlieghoogte toeneemt. Dit komt omdat bij hogere vlieghoogtes steeds meer \gls{UABS}'s in \gls{LOS} komen.
De vlieghoogte laten toenemen van 20 naar 200 m zal de 
 \gls{SAR} van andere \gls{UABS}'s tussen 46 en 49 keer groter maken voor een \gls{isotropicradiator} en tussen
70 en 85 keer groter voor een microstrip patch antenne.
Wanneer ook de resultaten van \cite{U1} overwogen worden waarbij een vlieghoogte van 80 meter voorgesteld wordt als optimale 
vlieghoogte  voor optimaal bereik en bachaul verbindingen zal ook hier de vlieghoogte van 80 meter voorgesteld worden 
voor het stadscentrum van Gent.

Tot besluit wordt een microstrip patch antenne
met een openingshoek van \ang{90} als geschikt startpunt beschouwd.
Deze directionele  antenne focust de elektromagnetische straling daar waar het nodig is.
en reduceert hierdoor ongewenste zijwaartse straling.
Het wordt aangeraden om de antenne toe te passen in een netwerk dat energieverbruik minimaliseert
omdat hierbij minder drones nodig zijn en daardoor goedkoper is.
De optimale vlieghoogte voor het stadscentrum in Gent bevindt zich rond 80  meter.
Lagere vlieghoogtes vereisen veel meer drones terwijl hogere vlieghoogtes de 
elektromagnetische straling laten toenemen.
Wanneer deze configuratie wordt toegepast op een netwerk met 224 gebruikers zal 
de gemiddelde gebruiker een \gls{SAR} ondervinden  van ongeveer $0.2\ \mu W/kg$ en een
 \gls{DL} elektromagnetische blootstelling van $69.5\ W$. 
 Het netwerk vereist gemiddeld 96 \gls{UABS}'s met een totaal energieverbruik van
 $114\ V/m$. Dat is $1.19\ W$ per \gls{UABS}.

Voor toekomstig onderzoek kunnen nog extra parameters onderzocht worden. Verschillende 
 \gls{pusch}-waarden worden verondersteld om een grote invloed te hebben op de 
 \gls{UL} straling en ook de blootstelling van  
backhaul verbindingen moeten nog overwogen worden.
Verder is de tool klaar om MiMo en massive MiMo te ondersteunen aangezien 
de tool eenvoudig uitgebreid kan worden om meer complexe stralingspatronen zoals beamforming te ondersteunen.
Als laatste is er ook nog ruimte om de tijdscomplexiteit van het programma te verbeteren door het exacte algoritme 
te vervangen door heuristische methodes. 


\section*{Dankwoord}

Ik wens de onderzoeksgroep WAVES van de Universiteit Gent te bedanken voor 
het beschikbaar stellen van hun capacity based deployment tool om zo dit onderzoek mogelijk te maken.

\bibliographystyle{ieeetr}
\bibliography{referenties}


\end{document}
