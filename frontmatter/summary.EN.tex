\begin{center}
\textsc{\textbf{\Huge Evaluating Human Electromagnetic Exposure in a UAV-aided Network}}\\
by Thomas Detemmerman

Master's dissertation submitted in order to obtain the academic degree of Master of Science in Information Engineering Technology
industri\"ele wetenschappen: informatica\\
Academiejaar 2019-2020

Supervisors: Prof. dr. ir. Wout Joseph, Prof. dr. ir. Luc Martens\\
Counsellors: Dr. ir. Margot Deruyck, MPhil. German Dario Castellanos Tache\\
Faculty of engineering and architecture\\
Ghent University
\end{center}
\textsc{\textbf{\LARGE Abstract}}\\
Society relies more than ever on the availability of wireless networks.
Due to the mobility of a UAV, a UAV-aided network is able to provide this necessary access in case the existing terrestrial network gets damaged.
However,  the public is 
concerned about the potential health effects of the electromagnetic radiation caused by these networks.
Therefore, mobile devices and base stations have to comply to strict legislation enforced by the government.

This research investigates how different scenarios influence power consumption, electromagnetic exposure and specific absorption rate.
These different scenarios are defined by various flying heights, number of UAVs available and population sizes. Further, also 
the proper microstrip patch antenna is defined and attached to the UAV. 
The antenna will be responsible for the communication between the UAV and the users it covers.
Its performance is compared to  
an equivalent isotropic radiator.
Thereafter, the network will be optimized towards goals like electromagnetic exposure of the average user or 
power consumption of the entire network; which results in conflicting requirements.

To accomplish this goal, the capacity based deployment tool of the WAVES research group at Ghent University
will be extended so it would be able to calculate electromagnetic exposure.
Further, the tool now also provides support to optimize the networks towards electromagnetic exposure or power consumption.

It looks from the results that 
the microstrip patch antenna with an aperture angle of \ang{90} is a suitable starting point for an antenna. 
This directional antenna focusses electromagnetic radiation where it is needed. Unwanted sideways radiation 
is therefore reduced by design.
The sufficiently large aperture angle covers enough users. The antenna is recommended to be deployed in a power consumption 
optimized network since less drones are required and therefore also less expensive.
The optimal flying height for the city centre of Ghent is believed to be situated at 80 metres since lower flying heights require much more UABSs and
higher flying heights have a negative influence on the electromagnetic exposure.  

\textsc{\textbf{\LARGE Keywords}}\\
deployment tool,  electromagnetic exposure, LTE
microstrip patch antenna, power consumption,
radiation pattern, specific absorption rate (SAR),
UAV, unmanned aerial base stations
wireless access network, emergency network