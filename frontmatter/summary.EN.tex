\begin{center}
\textsc{\textbf{\Huge Evaluating Human Electromagnetic Exposure in a UAV-aided Network}}\\
by Thomas Detemmerman

Master's dissertation submitted in order to obtain the academic degree of Master of Science in Information Engineering Technology
industri\"ele wetenschappen: informatica\\
Academiejaar 2019-2020

Supervisors: Prof. dr. ir. Wout Joseph, Prof. dr. ir. Luc Martens\\
Counsellors: Dr. ir. Margot Deruyck, MPhil. German Dario Castellanos Tache\\
Faculty of engineering and architecture\\
Ghent University
\end{center}
\textsc{\textbf{\LARGE Abstract}}\\
Society relies more than ever on the availability of wireless networks.
Due to the mobility of a UAV, a UAV-aided network is able to provide this necessary access in case the existing terrestrial network gets damaged.
However,  the public is 
concerned about the potential health effects of the electromagnetic radiation caused by these networks.
Therefore, mobile devices and base stations have to comply to strict legislation enforced by the government.

This research investigates how different scenarios influence power consumption, electromagnetic exposure and specific absorption rate.
These different scenarios are defined by various flying heights, number of UAVs available and population densities. Further, also 
an analysis on the difference
between a realistic microstrip patch antenna and a fictional equivalent isotropic radiator is performed.
Thereafter, the network will be optimized towards goals like electromagnetic exposure of the average user or 
power consumption of the entire network; which results in conflicting requirements.

To accomplish this goal, a capacity based deployment tool will be used which simulates an entire UAV-aided network.
In this way, all important sources for electromagnetic radiation, like all user equipment and all flying base stations, 
will be considered.

It looks from the results that a power consumption optimized network with a fixed flying height of 80 metres is the recommended approach
for the city centre of Ghent.
A microstrip patch 
antenna with a sufficiently large aperture angle is a good starting point. However, different antenna array configurations still have to 
be investigated.

\textsc{\textbf{\LARGE Keywords}}\\
deployment tool,  electromagnetic exposure, LTE
microstrip patch antenna, power consumption,
radiation pattern, specific absorption rate (SAR),
UAV, unmanned aerial base stations
%wireless access network, emergency network