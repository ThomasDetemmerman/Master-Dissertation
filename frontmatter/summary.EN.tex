\begin{center}
\textsc{\textbf{\Huge Evaluating the Total Human Electromagnetic Exposure in a UAV-aided Network}}\\

by\\
Thomas Detemmerman

Master's dissertation submitted in order to obtain the academic degree of Master of Science in Information Engineering Technology
industri\"ele wetenschappen: informatica\\
Academiejaar 2019-2020

Supervisors: Prof. dr. ir. Wout Joseph, Prof. dr. ir. Luc Martens\\
Counsellors: Dr. ir. Margot Deruyck, MPhil. German Dario Castellanos Tache\\
Faculty of engineering and architecture\\
Ghent University
\end{center}

\textsc{\textbf{\LARGE Samenvatting}}\\

Society relies more than ever on the availability of the wireless networks but is at the same time also 
concerned about the potential health effects of the electromagnetic radiation caused by these networks.
The government has enforced strict legislations to which mobile devices and base stations have to satisfy.

This research investigates the specific absorption rate caused by these electromagnetic waves by taking all mobile devices and base stations into account.
To accomplish this goal, the deployment tool developed by the WAVES research group at Ghent University will be used. This tool simulates an entire network 
where transmission towers are represented by femtocell base stations attached to drones. This research also investigates how these drones can be guided 
in order to reach certain goals like minimizing power consumption or electromagnetic exposure.

It looks from the results that ... (todo)



\textsc{\textbf{\LARGE Trefwoorden}}\\

LTE, Electromagnetic Radiation, power consumption, drones, femtocell, microstrip patch antenna, radiation pattern, specific absorption rate (SAR)
