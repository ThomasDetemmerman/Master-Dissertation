\begin{center}
\textsc{\textbf{\Huge Evaluatie van de electromagnetische blootstelling van de mens in een netwerk van drones}}\\

door\\
Thomas Detemmerman

Masterproef ingediend tot het behalen van de academische graad van Master of Science in de
industri\"ele wetenschappen: informatica\\
Academiejaar 2019-2020

Promotoren: Prof. dr. ir. Wout Joseph, Prof. dr. ir. Luc Martens\\
Begeleider: Dr. ir. Margot Deruyck, MPhil. German Dario Castellanos Tache\\
Faculteit Ingenieurswetenschappen en architectuur\\
Universiteit Gent
\end{center}

\textsc{\textbf{\LARGE Samenvatting}}\\
\textcolor{red}{TODO: Ook updaten op plato}
\textcolor{red}{TODO: Translate}

%De hedendaagse samenleving vertrouwt meer dan ooit op de aanwezigheid van draadloze netwerken. 
%Tevens groeit ook de bezorgdheid bij de menigte over de electromagnetische straling die hierbij gebruikt wordt. De overheid hanteerd dan ook
%strenge richtlijnen waaraan mobiele toestellen en zendmasten moeten voldoen.

%Dit onderzoek tracht de specifieke absorptie snelheid van elektromagnetische straling in kaart te brengen door rekening te houden met alle mobiele 
%toestellen en zendmasten. Om dit te verwezelijken wordt gebruik gemaakt van een tool ontwikkeld door de onderzoeksgroep WAVES aan de UGent. Deze tool
%simmuleert een volledig netwerk waarbij zendmasten bevestigd worden aan drones. Dit onderzoek observeert verder hoe deze drones kunnen worden aangestuurd
%zodoende dat bepaalde doelstellingen zoals het minimaliseren van energieverbruik of electromagnetische straling bereikt kunnen worden.

%Uit de resultaten blijkt dat...

\textsc{\textbf{\LARGE Trefwoorden}}\\

LTE, electromagnetische blootstelling, energieverbruik, drone, femtocell, microstrip patch antenna, stralingspatronen, specific absorption rate (SAR)

