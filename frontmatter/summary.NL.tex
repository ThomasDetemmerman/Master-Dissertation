\begin{center}
\textsc{\textbf{\Huge Evaluatie van de electromagnetische blootstelling van de mens in een netwerk van drones}}\\

door\\
Thomas Detemmerman

Masterproef ingediend tot het behalen van de academische graad van Master of Science in de
industri\"ele wetenschappen: informatica\\
Academiejaar 2019-2020

Promotoren: Prof. dr. ir. Wout Joseph, Prof. dr. ir. Luc Martens\\
Begeleider: Dr. ir. Margot Deruyck, MPhil. German Dario Castellanos Tache\\
Faculteit Ingenieurswetenschappen en architectuur\\
Universiteit Gent
\end{center}

De hedendaagse samenleving vertrouwt meer dan ooit op de aanwezigheid van draadloze netwerken. 
Dankzij de mobiliteit van  drones kan een drone-gestuurd netwerk een de nodige mobiele data voorzien 
indien het bestaande netwerk beschadigd is.
Er is echter een groeinde vrees naar mogelijk gezondheidseffecten veroorzaakt door deze
mobiliele netwerken. De overheid stelt strikte wetgevingen op waaraan deze mobiele netwerken dienen te voldoen.

Dit onderzoek bekijkt hoe veschilende scenarios het energieverbruik, electromagnetische bloodstelling en 
specifieke absorptietempo kunnen be\"invloeden.
Drie verschillende scenarios zijn gedefineerd waarbij verschillende vlieghoogtes, aantal drones en 
populatiegroottes onderzocht worden.
Verder is er ook een microstrip patch antenne gedefinieerd en bevestigd op een drone. 
De antenne zal de communicatie tussen de drone en de gebruikers verzorgen.
De performantie van deze antenne zal vergeleken worden met een istorope atenne.
Vervolgens zal het netwerk geoptimaliseerd worden naar electromagnetische straling van het individu of 
naar het energieverbruik van het gehele netwerk. Deze twee doelstellingen resulteren in 
tegenstrijdige vereisten. 

Om dit doel te bereiken is de capacity based deployment tool van de onderzoekgsgroep WAVES op de 
Universiteit Gent verder uitgebreid zodoende dat electromagnetische straling berekend kan worden.
Verder is de tool nu ook in staat om te opimaliseren naar electromagnetische straling of energieverbruik. 

Uit de resultaten blijkt dat een microstrip patch antenne
met een openingshoek van \ang{90} een geschikt startpunt is voor een antenne.
Deze direcitonele  antenne focust de electromagnetische straling daar waar het nodig is.
Ongewenste zijwaardse straling wordt gereduceerd door het design.
Het wordt aangeraden om de antenne toe te passen in een netwerk die energieverbruik minimaliseerd
omdat hierbij minder drones nodig zijn en daardoor goedkoper uitvalt.
De optimale vlieghoogte voor het standscentrum in Gent bevindt zich rond 80  meter.
Lagere vlieghoogtes vereisten veel meer drones terwijl hogere vlieghoogtes de 
electromagnetische straling laten toenemen.

\textsc{\textbf{\LARGE Trefwoorden}}\\

LTE, electromagnetische blootstelling, energieverbruik, drone, femtocell, microstrip patch antenna, stralingspatronen, specific absorption rate (SAR)

