\chapter{Conclusie}

De bedoeling van deze thesis is het aanbieden van een eenvoudig en relatief snelle methode om nieuwe cloud-allocatieschema's te evalueren op een echt OpenStack-testbed. Zoals beschreven in het gerelateerd werk in Hoofdstuk~\ref{chap:rel_work}, is er de laatste tijd veel onderzoek gebeurd naar nieuwe cloud resource-allocatieschema's. Deze nieuwe schema's werden in de meeste gevallen getest en geëvalueerd op een simulator in plaats van op een fysiek cloud-testbed. Een belangrijke reden voor het gebruik van een simulator in plaats van een fysiek cloud-testbed is de kostprijs en de vereiste configuratie. In deze thesis kan er, met behulp van DevStack, snel en eenvoudig een OpenStack-cloud worden uitgerold op een fysiek cloud-testbed. Het voordeel van OpenStack is dat het een gratis softwarepakket is, compatibel met relatief goedkope hardware. Dankzij DevStack valt er veel configuratie weg en daarmee worden de twee grote redenen om geen fysiek cloud-testbed te gebruiken vermeden. Het te testen allocatieschema moet ``vertaald'' worden zodat het kan samenwerken met Nova, een OpenStack onderdeel verantwoordelijk voor het zware rekenwerk. Deze samenwerking tussen het te testen allocatieschema en Nova kan, dankzij de open-source code van OpenStack, eenvoudig gebeuren, mits enkele aanpassingen. Na het inpluggen van dat nieuwe schema kan er via de template een schaalbare applicatie worden uitgerold op de cloud. Deze schaalbare applicatie zorgt voor echte werklast op de fysieke hypervisors door het berekenen van enkele complexe fractalen waardoor de applicatie kan in- en uitgeschaald worden door de OpenStack-componenten Heat, Ceilometer en Aodh. De monitorapplicatie bewaakt dit gehele proces en bewaart statistieken in een databank. Nadien kunnen deze statistieken worden opgevraagd zodat dit alles kan geëvalueerd worden.

Om dit alles te bewijzen is er gedurende deze thesis een Proof-of-Concept uitgewerkt met een Round-Robin allocatieschema. De resultaten hiervan worden besproken in Hoofdstuk~\ref{chap:evaluation} en zijn zoals verwacht aansluitend bij de gestelde hypothese. Round Robin is een simpel allocatieschema dat in sommige contexten zeer bruikbaar is, maar het grote nadeel blijft dat het volledig contextloos werkt.

Het eigenlijke besluit van deze thesis luidt dat via OpenStack uitgerold met behulp van DevStack, de schaalbare FaaFo-applicatie, Rally en de monitortechniek er een nieuw schema kan worden uitgetest en geëvalueerd. Het enige dat nog dient te gebeuren is het nieuwe schema inpluggen in Nova en deze service herstarten.