\chapter{Results and Discussion}
\label{chap:results}

This chapters shows the achieved results from all considered scenarios and cases.
The first section talks about a small network with only one user and one \gls{UAV}s. In 
the following section, this network is expanded for a larger population but still with one \gls{UAV}s.
The last section is also for a large population and with an unlimited number of \gls{UAV}s.

%%%%%%%%%%%%%%%%%%%%%%%%%%%%%%%%%%%%%%%%%%%%%%%%%%%%%%%%%%%%%%%%%%%%%%%%%%%%%%%%%%%%%%%%%%%%%%%%%%%%%%%%%%%%%%%%%%%%%%%%%%%%%%%%%%%%
\section{Scenario 1: One User and One Drone}
The network contains only one user for this scenario. This means that there is only one location possible for the drone which is just above 
the user. This section will investigate minimal required transmission power and SAR values from different sources.
Finally, also the power consumption of the entire network is measured. The  ``entire network" refers to all drones and antennae. The entire network 
will for the first scenario be constructed out of a singe \gls{UABS}.

\subsection{The Influence of the Maximum Transmission Power}
\label{s1a}
\gls{LTE} makes usages of power control meaning that no more power will be used then strictly necessary. The actual 
transmit power $P_{tx}$ therefore ranges between zero and the maximum allowed input power. $P_{tx}$ is zero when the \gls{UABS} doesn't cover anybody.
For instance when the flying height is too high and therefore also the path loss that comes with it, the maximum allowed $P_{tx}$ is not enough to cover 
the distance. In such case, the \gls{UABS} is shut down since it cannot meet the requirements.
Increasing the maximum transmission power will not influence the actual used $P_{tx}$ or $SAR_{10g}$ because the \gls{UABS} will not use more
then strictly required. It is therefore more useful to match the actual transmission power against a variable flying height. 

Figure \ref{fig:ptxfh} shows the logarithmic relationship between $P_{tx}$ and flying height.
As already discussed in \ref{sec:scenarios_s1}, the user is outdoor and just below the \gls{UABS}. There is thus a free line-of-sight between both
radiators. It is clear  from figure \ref{fig:ptxfh} that a discontinue step function is achieved. This is because multiple flying heights correspond to the same transmission power.
When the flying height increases, so does the path loss. \gls{LTE} tries to counteract this by increasing the power level. Each time 
the path loss becomes too high, the power level of the antenna increases with one dBm. Doing so, decreases path loss allowing the antenna to reach
the user again. 

\begin{figure}[t]
  \centering
  \includegraphics[width=\textwidth]{../results/s1/ptx.png}
  \caption{Minimal required transmission power by the antenna to reach the ground just below him. The red line shows the default maximum transmission power.}
  \label{fig:ptxfh}
\end{figure}

After a jump in the step function, there is an overestimation meaning the input power increased more then necessary. So multiple flying heights correspond with the same $P_{tx}$.
Further, dBm is also a logarithmic scale meaning that while 10 dBm equals 10 mW, 20 dBm equals 100 mW. This explains why the black lines become longer at higher flying altitudes.
Each time the power level increases with one dBm, the overestimation becomes larger. If the tool would make usage of a smaller step size, a more continuous 
logarithmic function would be achieved. This would however worsen the time complexity because it would take much more iterations before 
the power level exceeds the path loss. 

The red line in figure \ref{fig:ptxfh} indicates the default maximum transmission power used during simulations. 
In a free line-of-sight scenario with only one user, a \gls{UABS} can fly up to 387 metres before losing connection.

This scenario is investigated with a microstrip patch antenna using power consumption optimization. 
 However, the chosen optimization strategy doesn't really matter as already explained in  \ref{sec:scenarios_s1}. This is because the decision 
 algorithm decides which user 
needs to be connected to which user. Since only one \gls{UABS} is available, both optimization strategies will behave identical.
Further, also the used antenna will not make any difference
despite the fact that a microstrip patch antenna has attenuation while an \gls{isotropicradiator} doesn't.
The user is namely positioned in the perfect center of the main beam where there is 
no attenuation experienced in either cases. So the results are applicable for the four possible cases from figure \ref{fig:fourCasesMatrix}.

\FloatBarrier
\subsection{Influence of the Flying Height}
\label{sub:senario1_influenceOfFlyHeight}

This section investigates how the flying height of a \gls{UABS} influences $SAR_{10g}$ and power consumption.
The $SAR_{10g}$, which is actually induced electromagnetic radiation into our user, is represented in figure \ref{fig:s1_fhsar}
and shows that for a low flying drone, \gls{UE} is the main source of electromagnetic radiation.
This changes around 80 meters where the \gls{UL} electromagnetic radiation from the \gls{UE}
exceeds the \gls{DL} radiation in order to still be able to reach the high flying \gls{UABS}s.

\gls{SAR}-values are caused by the transmit power  $P_{tx}$ of the antenna. The $P_{tx}$ in section \ref{s1a}
showed a discontinue behaviour that sometimes radiate more as strictly necessary. This has thus a direct influence
on the \gls{DL} \gls{SAR}. Hence the same discontinue behaviour. The \gls{DL} \gls{SAR} can be simplified to a perfect constant line.
This constant behaviour can once again be explained with power control. When the \gls{UABS} flies lower, there is  less path loss and the \gls{UABS} 
will therefore reduce the $P_{tx}$. This results in formula \ref{eq:exposureBasicFormula} where the electromagnetic exposure is a constant fraction of power and distance.

\begin{equation}
\vec{E} (V/m) = \frac{\Delta U (V) }{\Delta x (m)}
\label{eq:exposureBasicFormula}
\end{equation}

%The SAR values are the result of multiplying the electromagnetic exposure with a constant as explained in equation \ref{eq:DLconvertion}. So both have a linear relationship.

Figure \ref{fig:s1_fhsar} doesn't show radiation from neighbours, because there are none present in this scenario. 
Finally, all these values are added up as explained in formula \ref{eq:overallSARwb} resulting in the total \gls{SAR}
to which our user is exposed.

\begin{figure}[]
  \includegraphics[width=\textwidth]{../results/s1/fhvssar.png}
  \caption{How SAR values from different sources are influenced by different flying altitudes.}
  \label{fig:s1_fhsar}
\end{figure}

The power consumption of the entire network includes both the power required by the drone itself and the antenna he is carrying.
The power consumption of the entire network is here of course for the only \gls{UABS} available. Figure \ref{fig:ptxfh} shows an 
exponential relationship between both power consumption and flying height. 

\begin{figure}[t]
  \centering
  \includegraphics[width=\textwidth]{../results/s1/fhvspc.png}
  \caption{Minimal required transmission power by the antenna to reach the ground just below him. The red line shows the default maximum transmission power.}
  \label{fig:ptxfh}
\end{figure}

%%%%%%%%%%%%%%%%%%%%%%%%%%%%%%%%%%%%%%%%%%%%%%%%%%%%%%%%%%%%%%%%%%%%%%%%%%%%%%%%%%%%%%%%%%%%%%%%%%%%%%%%%%%%%
\FloatBarrier
\section{Scenario 2: Increased Traffic}

This scenario has just like the previous scenario only one drone available. However, more users will be present in the network.
First, a variable flying altitude is investigated for a fixed number of 224 users. 
Secondly, the flying height is set to 100 metres with a variable number of users.
When designing the network, there will be as much possible drone locations as there are users in the network and the tool
will consider all of them. It's only when the programme is finished, that one drone remains.

\subsection{Influence of the Flying Altitude}
The first case investigates how the network, consisting out of one \gls{UABS}, behaves when applied on an ordinary day during rush hour. 
Different fixed flying heights are considered while 224 active users are distributed uniformly over the city center of Ghent. 

A power consumption optimized network with an \gls{EIRP} antenna (green) has the highest exposure. 
This is logical when comparing with an EIRP antenna in an exposure optimized network (red). 
However, when looking at figure \ref{fig:s2a_dlAndPc} on the right, the power consumption in a power consumption optimized network is worse 
than in an exposure optimized network. To understand this, the behaviour of the deployment tool needs to be understood first. 
A power consumption optimized network will result in a few high powered \gls{UABS}s because increasing the input power of an antenna cost 
less then activating a new  drone. Likewise, an exposure optimized network 
generates a lot of low powered \gls{UABS}s because the lower the antenna's power, the lower the exposure. This has the consequence that the cover radius 
is less and therefore requiring more drones which costs more energy.
When only a limited amount of \gls{UABS}s are available, 
like only one in this scenario, the tool will only keep \gls{UABS}s which cover most of the users. 
Therefore is the power consumption in a power consumption optimized network way higher. 


\begin{figure}[h!]
  \includegraphics[width=\textwidth]{../results/s2/fhvsdlAndPc.png}
  \caption{The influence of the flying height on the weighted average downlink exposure of users in the network.}
  \label{fig:s2a_dlAndPc}
\end{figure}



%\newpage
The \gls{DL} exposure in figure \ref{fig:s2a_dlAndPc} increases along with the flying height. One might expect a more constant 
behaviour like it was the case in figure \ref{fig:s1_fhsar} of scenario 1. To understand this, the scenario has been deducted 
with only two users and is illustrated in figure \ref{fig:schematicprove}.
The two users who will be referred to by `red' and `blue' are 90 metres separated from each other with a building between them.
Scenario 1 already explained that the charts can be simplified and the blue line from fig. \ref{fig:prove} remains in fact constant between the zero and 130 metres.
The chart shows that the \gls{UABS} is positioned above the blue user. The red user is in \gls{NLOS} as long as the \gls{UABS} remains below 20 metres.

\begin{figure}[]
  \includegraphics[width=\textwidth]{../results/s2/prove.png}
  \caption{Scenario 2 with only 2 users. The coloured areas are only applicable for the red user. The blue user is connected during the entire time.}
  \label{fig:prove}
\end{figure}


\begin{wrapfigure}{r}{0.48\textwidth}
  \begin{center}
    \includegraphics[width=0.48\textwidth]{../results/s2/proveScenario.png}
  \end{center}
  \caption{Schematic overview of scenario 2 with only 2 users.}
  \label{fig:schematicprove}
\end{wrapfigure}
Once the \gls{UABS} increases its flying altitude, the red user becomes into \gls{LOS} but still remains uncovered. This is because the tool initially locate a possible 
\gls{UABS} above each user and thereafter performs the  fitness function. The applied fitness function must have decided that it is better to connect 
each user to the \gls{UABS} above him. At a final state, the tool check whether the number of online drones does not exceed the capacity of the facility
which is here the case. The tool therefore deactivates one \gls{UABS} causing the red user to be uncovered. One could argue that the 
the red user should be connected to the online drone who is only 90 metres away. This would however require the online drone to increase his power consumption which 
would make the decisions made by the optimization strategy obsolete.
When the drone flies higher, the difference in distance between both users and the base station decreases. In other words, the Pythagorean theorem shows that when the flying height of the 
\gls{UABS} increases, the distance with the blue user increases faster compared to the distance between that same \gls{UABS} and the red user. This is also illustrated in figure \ref{fig:schematicprove}.
At 130 metres, the tool decides to connect both users to the same \gls{UABS}. Therefore, it increases it's power consumption so the red user would  have the minimal 
required electromagnetic exposure. This has of course a negative influence for the blue user who is way closer and experience now a much higher exposure level in fig \ref{fig:prove}.
Around 180 metres, the  red and blue line switch because the drone changes position. As explained before, the tool assigns two possible drones, one above 
each user. The tool must have decided that connecting both users to the other drones improves the fitness function of the entire network even though that difference might be 
very little. This brings us back to figure \ref{fig:s2a_dlAndPc} where the electromagnetic exposure in the weighed average of all users. In other words, there are 223 users who behave like the  red user while only
one user behaves like the blue one. The similarity between the red line from fig. \ref{fig:prove}  and  the electromagnetic exposure in fig  \ref{fig:s2a_dlAndPc} is clear.

\begin{figure}[h]
  \includegraphics[width=\textwidth]{../results/s2/fhvscov.png}
  \caption{This graph shows the percentage of covered users by one drone for different flying heights.}
  \label{fig:s2fhvscov}
\end{figure}

Figure  \ref{fig:s2fhvscov} shows that the flying height has a positive influence on the user coverage. 
When a \gls{UABS} flies higher, there is less path loss between the user and the drone caused by buildings but also the path loss to neighbouring 
users decreases as explained 
with figure \ref{fig:prove} and \ref{fig:schematicprove}. 
Also the increasing \gls{DL} exposure  from figure \ref{fig:s2a_dlAndPc} from earlier indicated that the
user coverage should grow.

When replacing the fictional \gls{EIRP} antenna with a microstrip patch antenna, the percentage of covered users drops for both 
optimization strategies. This is because users who have a higher horizontal distance between themselves and the \gls{UABS}, 
experience a higher attenuation. When a microstrip patch antenna is positioned higher, the range of the antenna increases 
since the angle between the user and the \gls{UABS}s main lob decreases. The user will therefore experience less attenuation.

Eventually, figure \ref{fig:s2shfourSourcesMatrix} shows the total whole body $SAR_{10g}$, deducted from all electromagnetic sources. This being the exposure 
of the only \gls{UABS} available in the network, 
 the \gls{UL} exposure from the user’s own device and the exposure of the devices from all other users. 
 Thereafter, the weighted average whole body \gls{SAR} for each individual source in the network is calculated with the 50th and 95th percentile 
 being the most important values. This is because not only the mean value is important but also users who experience higher 
 levels of whole body $SAR_{10g}$.

When investigating the three different sources of which the total \gls{SAR}-values are based on, we see 
that the radiation from the \gls{UABS} is the main factor followed by the near field radiation from the user's own device.
The far field radiation from other \gls{UE} has barely influence. 
It looks like it is zero but it is just very low compared to the other two values and in fact does increases when the flying height becomes larger.

The weighted average $SAR^{ul}_{10g}$ from the own device is zero in an exposure optimized network with a microstrip patch antenna which is even lower that the $SAR^{neighbours}_{10g}$.
This is because the coverage in this scenario is so low that the weighted average only consist of uncovered users and an uncovered user his device has no power consumption.
\begin{figure}[]
  \includegraphics[width=\textwidth]{../results/s2/fhFourSources.png}
  \caption{This figure shows how different sources are influenced by an increasing flying height.}
  \label{fig:s2shfourSourcesMatrix}
\end{figure}

%%%%%%%%%%%%%%%%%%%%%%%%%%%%%%%%%%%%%%%%%%%%%%%%%%%%%%%%%%%%%%%%%%%%%%%%%%%%%%%%%%%%%%%%%%%%%%%%%%%%%%%%
\FloatBarrier
\subsection{Influence of the Number of Users}
\label{s2b}

The number of covered users increase linearly compared to the number of users present in the network as shown in figure 
\ref{fig:s2uvsnumcovusers} on the right. It illustrates how an \gls{isotropicradiator} is able of reaching more users 
compared to a microstrip patch antennae.
Also, power consumption optimized networks are able of reaching more users compared to exposure optimized networks.
This is because power consumption optimized network will result in few high powered base station while an 
exposure optimized network result in a lot of low powered base stations. This behaviour will further be explained in section \ref{s3}

\begin{figure}[h!]
  \includegraphics[width=\textwidth]{../results/s2/uvsnumdronesAndCov.png}
  \caption{The influence of increasing traffic on user coverage.}
  \label{fig:s2uvsnumcovusers}
\end{figure}

The linear regression lines from \ref{fig:s2uvsnumcovusers} can be predicted with the equations in \ref{eq:numcovusers}.

\begin{equation}
\text{number of users =}
    \begin{cases}
      y = 0,0233x + 2,3553 & \text{if EIRP and pc}\\
      y = 0,0197x + 2,6144  & \text{if EIRP and exp}\\
      y = 0,0131x + 2,4371  & \text{if micro and pc}\\
      y = 0,0119x + 2,4652  & \text{if micro and exp}
    \end{cases} 
    \label{eq:numcovusers}      
\end{equation}

Figure \ref{fig:s2uvsnumcovusers} on the left show the percentage of covered users that follows out of \ref{fig:s2uvsnumcovusers} on the right by taking the equations from \ref{eq:numcovusers} and dividing them by $x$.
This results in a decreasing logarithmic behaviour because the regression lines from  \ref{fig:s2uvsnumcovusers} have a slope of less then 0.5.
So in other words, the  percentage of covered users for a sparsely populated network is more compared to the percentage of users in high dense populations.

\begin{figure}[h!]
  \includegraphics[width=\textwidth]{../results/s2/uvsdlAndPc.png}
  \caption{This figure shows how different sources are influenced by an increasing number of users. }
  \label{fig:s2b_dlAndPc}
\end{figure}

The downlink exposure is shown in figure \ref{fig:s2b_dlAndPc} on the left and is directly influenced by the percentage of covered users. 
The average electromagnetic exposure decreases when more users become uncovered. Since an \gls{isotropicradiator} in a power consumption optimized network (green)
will have the highest coverage, also the \gls{DL} electromagnetic radiation from \gls{UABS}s will be higher compared to other configurations.
Despite the fact that the percentage of covered users decreases, the effective number of covered users increases. The power consumption of the only 
active \gls{UABS} slightly increases in order to serve those covered users.


Figure \ref{fig:s2fourSourcesMatrix} investigates the assets of each source contributing to the total \gls{SAR}. All four 
configurations show that base stations are the main source of electromagnetic radiation.
Figure \ref{fig:s2uvsnumcovusers} already 
showed that for sparsely populated networks, a higher percentage is covered so the weighted average of the \gls{UL} \gls{SAR} will also be higher. 
When the population becomes more dense,
more users become uncovered which decreases the weighted average of the \gls{UL} \gls{SAR}.
The chart also proves once again that the far field radiation from \gls{UE} can be neglected. The \gls{SAR} from 
neighbouring devices is not zero as it looks from figure \ref{fig:s2fourSourcesMatrix} but is just really low compared to the much higher
\gls{SAR}-values from other sources.

\begin{figure}[h!]
\centering
  \includegraphics[width=\textwidth/6*5]{../results/s2/uFourSources.png}
  \caption{This figure shows how different sources are influenced by an increasing number of users. }
  \label{fig:s2fourSourcesMatrix}
\end{figure}

While the population grows, more and more users become uncovered causing the average SAR to drop. However, this does not conclude that the same applies for 
users who are covered. To investigate this, a user is positioned in the middle of the city center of Ghent and a drone is positioned above him. Initially, only 
49 people are active around him. The \gls{SAR} of our central user is monitored while the population around him grows.
Figure \ref{fig:connectionMap} shows with the black lines which users are connected. The left map is for only 50 users and 
shows that only one user is connected besides our central user. The map on the right is taken with 600 users and shows much more connected users.



It might look that by increasing the population, the SAR of a user who is directly beneath a \gls{UABS} would be less but that is 
certainly not the case as demonstrated in the experiment below. A central user will be placed in the middle of Ghent with one drone above him.
The different \gls{SAR}-values of this central user are monitored while the population around him grows.

\begin{figure}[h!]
\centering
  \includegraphics[width=\textwidth/6*5]{../results/s2/uvsulsarcentralUser.png}
  \caption{SAR-values for the user who is directly beneath the only UABS available.}
  \label{fig:uvsulsarcentralUsers}
\end{figure}

Scenario 1 already showed that the \gls{SAR} from the user's own device is only influenced by the flying height. 
The flying height for this experiment is fixed and the \gls{UL} \gls{SAR} from his device should therefore be also a constant. 
A hypothesis that is confirmed by figure \ref{fig:uvsulsarcentralUsers}.
The \gls{SAR} from the \gls{UABS} experience a slight  increase. When the population grows, more users become available 
of which some will spawn near the central user. The \gls{UABS} will likely decide to cover these user  as well as visible in figure \ref{fig:connectionMap}.
These user might have a slightly 
worse path loss because of obstructing buildings or somewhat bigger distance. The \gls{UABS} reacts to this by increasing 
his power consumption causing an increase in the \gls{DL} \gls{SAR} for the central user.

The far-field radiation from \gls{UE} is very low as mentioned before and is therefore not visible in figure \ref{fig:connectionMap} 
on the left and is therefore illustrated in a separate char on the right. 
It shows that the \gls{SAR}  from other \gls{UE} indeed increases. This is normal 
behaviour considering that more and more people become available around the central user of which some will be connected to the \gls{UABS}
and therefore also emitting radiation.

\begin{figure}[!htb]
\minipage{0.50\textwidth}
  \includegraphics[width=\linewidth]{../images/connectionsMap50Users.png}
\endminipage\hfill
\minipage{0.50\textwidth}%
  \includegraphics[width=\linewidth]{../images/connectionsMap600Users.png}
\endminipage
  \caption{Overview of which users are connected to the \gls{UABS}. The map on the left is for 50 active users while the map on the right is with 600 active users.}
  \label{fig:connectionMap}
\end{figure}
%%%%%%%%%%%%%%%%%%%%%%%%%%%%%%%%%%%%%%%%%%%%%%%%%%%%%%%%%%%%%%%%%%%%%%%%%%%%%%%%%%%%%%%%%%%%%%%%%%%%%%%%%%%%%s
\section{Scenario 3: Unlimited Drones}
\label{s3}

This scenario has just like the previous scenario much more users in the network 
and investigates the same cases which includes the variable flying height and the variable number of  users.
The only difference is that the restriction of only one \gls{UABS}s is dropped.

\subsection{Influence of the Flying Altitude}
\label{S3A}

The first case of this scenario examines how the network behaves for various flying heights and a fixed number of 224 users.
Scenario 2 already explained that when only one drone is available, a power consumption optimized network won’t result in a low 
powered network. In this scenario, there is no limitation on the number of drones and the network remains thus unaltered after the decision 
algorithm is done. Figure \ref{fig:s3a_dlAndPc} clearly proves that the different optimization strategies work as intended.
Power consumption optimized networks have indeed a lower power consumption but therefore results in higher electromagnetic radiation.
In contrast to an exposure optimized network that, as expected, will reduce the electromagnetic exposure by using more drones and thence also increasing the network's power consumption.
This conclusion was already made  in \cite{J1} and is supported by these results.

\begin{figure}[h!]
  \includegraphics[width=\textwidth]{../results/s3/fhvsdlAndPc.png}
  \caption{The influence of the flying height on the downlink electromagnetic radiation of the average user.. This graph shows the percentage of covered users by one drone for different flying heights.}
  \label{fig:s3a_dlAndPc}
\end{figure}

Figure \ref{fig:s3a_dlAndPc}  shows that the number of drones required decreases when the flying altitude becomes higher. A behaviour which was also determined in \cite{J2}.
At a low flying altitude, users in an exposure optimized network experience significantly lower exposure-values compared to a power consumption optimized network.
The exposure in a power consumption  optimized network starts high and 
decreases while an exposure optimized network behaves the opposite. This difference becomes less 
around 70 metres.
This can be explained when looking at figure \ref{fig:s3a_numDronesAndCov}.
At a flying height of 20 metres, the exposure optimized network has on average 220 to 224 \gls{UABS}s. That is (almost) one \gls{UABS} for each user
so it's logical that the electromagnetic exposure is very low.

The number of drones in a power consumption optimized network is much less in order 
to save energy but figure \ref{fig:s3a_numDronesAndCov} shows on the  left the same percentage of coverage for this flying altitude.
So these drones will try to cover users much further away and some of these connections will even be more worsened by obstructing buildings.
Because of this, users who are close and in \gls{LOS} will experience much higher electromagnetic radiation.
This path loss reduces when the \gls{UABS}s start to fly higher then the average  building and therefore exposure decreases.
Not only the power consumption optimized networks profit from higher flying altitudes, also the exposure optimized network does. For only a little bit 
more electromagnetic exposure, much less drones are required.

\begin{figure}[]
  \includegraphics[width=\textwidth]{../results/s3/fhvsnumdronesAndCov.png}
  \caption{This graph shows how much drones are required for different flying heights while trying to achieve a 100\% coverage.}
  \label{fig:s3a_numDronesAndCov}
\end{figure}

Both  \ref{fig:s3a_dlAndPc} and \ref{fig:s3a_numDronesAndCov}  show that the network profits from increasing the flying altitude. 
Not only less drones are needed but also the power consumption is lower. Both can be explained by the lower path loss when \gls{UABS}s fly higher.

Scenario 1 already proved that with low flying drones, the main source of electromagnetic radiation are \gls{UABS}. 
This changed around 80 meters where \gls{UL} electromagnetic radiation of the \gls{UE}
exceeds \gls{DL} radiation in order to still be able to reach the high flying \gls{UABS}s. 
When looking at the different individual sources in \ref{fig:s3a_fourSourcesMatrix}, we see 
that \gls{UL} \gls{SAR} is logarithmic increasing despite the fact that figure \ref{fig:s1_fhsar} showed that the  \gls{UL} \gls{SAR} 
increases exponentially. This was however deducted with only one user is present in the network as opposed to this scenario 
where 224 users are present. The covered users will still behave like in scenario 1 but much more users are uncovered (fig. \ref{fig:s3a_numDronesAndCov}) 
which decreases the average \gls{SAR}. The average  \gls{UL} \gls{SAR}  is thus lower as visible on figure \ref{fig:s3a_numDronesAndCov}.

\begin{figure}[]
  \includegraphics[width=\textwidth]{../results/s3/fhFourSources.png}
  \caption{Each chart shows the total SAR to which the average user is exposed. ``My UABS" stands for the UABS that is serving our average user while ``other UABSs'' stand for 
  all other UABSs to which that user is exposed to but not served by. ``Other" UE refer the exposure from all mobile devices that does not belong that user.}
  \label{fig:s3a_fourSourcesMatrix}
\end{figure}

We can see from \ref{fig:s3a_fourSourcesMatrix} that once the buildings level out (around 70 to 80 metres), the SAR from the serving UABS remain 
more or less constant. A behaviour that was already determined in scenario 1 and scenario 2: experiment \ref{fig:uvsulsarcentralUsers}. A negative consequence of 
this is that exposure from other \gls{UABS}s (who are serving  other users) increases. 
Also here will the lower path loss from  less obstructing buildings be the reason.
The figures from \ref{fig:s3a_fourSourcesMatrix} further also clearly show that a microstrip patch antenna cause less electromagnetic exposure because 
there is much less radiation from  ``other UABSs''.
The electromagnetic energy is more focussed towards the ground and there is less sideways radiation because of attenuation.


%%%%%%%%%%%%%%%%%%%%%%%%%%%%%%%%%%%%%%%%%%%%%%%%%%%%%%%%%%%%%%%%%%%%%%%%%%%%%
\FloatBarrier
\subsection{Influence of the Number of Users}
\label{S3B}

The last case of scenario 3 investigates variable number of users for a fixed flying height of 100 m. There is no 
restriction on the number of available drones just like in the previous case meaning that there are at most 
as much drones as users in the network. The correct behaviour of the decision algorithm became already clear in the previous subsection \ref{S3A} but is also
confirmed by this case.

Figure \ref{fig:s3b_numDronesAndCov} shows on the left how the tool tries to reach a 100\% coverage. The tool reaches this goal 
better for larger populations. The difference remains however very little. The tool also requires more drones for these large 
populations which is a logical consequence of scenario \ref{s2b} with only one \gls{UABS} where the percentage of covered users decreases for these larger populations.
 The difference in optimization strategy is very little for small amounts of people but increases very fast. Further, \ref{fig:s3b_dlAndPC}, shows an increase 
 in exposure for larger populations because more drones come with these larger populations as visible in \ref{fig:s3b_numDronesAndCov}.

\begin{figure}[h!]
  \includegraphics[width=\textwidth]{../results/s3/uvsnumdronesAndCov.png}
  \caption{This graph shows how much drones are required for different flying heights while trying to achieve a 100\% coverage.}
  \label{fig:s3b_numDronesAndCov}
\end{figure}

In a scenario with 600 active users, a clear difference is noticeable between the four configurations. 
For instance, an EIRP power consumption optimized 
network requires the least amount of drones (Figure \ref{fig:s3b_numDronesAndCov} on the right). This is logical when looking at figure \ref{fig:s3b_dlAndPC} where drones in such a configuration cause the highest amount of 
electromagnetic radiation. This behaviour was already discussed in subsection \ref{s2b}. 

On the complete opposite, we have a microstrip patch antenna in an exposure optimized network. 
This strategy prioritize the minimization of electromagnetic exposure. In addition, a microstrip antenna has a much more limited range.
Therefore, much more 
drones are required in order to reach 100\% coverage (Figure \ref{fig:s3b_numDronesAndCov}) and therefore requires much more energy 
to power all the drones (Figure \ref{fig:s3b_numDronesAndCov} on the right).

So both cases from  scenario 3 learn that a power consumption optimized network indeed result in less drones. The average power consumption 
is however much higher. At the same time, scenario 2 showed that the active \gls{UABS}s have a higher power consumption.
The statement that a power consumption optimized network will result in a few high powered devices is therefore confirmed.

Likewise for an exposure optimized network where we can conclude that the network has indeed a lower electromagnetic exposure but the power consumption 
of the entire network is higher. In scenario 2 became already clear that the active \gls{UABS}s have a low power consumption in order to 
guarantee low electromagnetic exposure towards the users.  The statement that an exposure optimized network will result in a lot low powered devices is thus also confirmed.

\begin{figure}[h!]
  \includegraphics[width=\textwidth]{../results/s3/uvsdlAndPc.png}
  \caption{The influence of the flying height on the downlink electromagnetic radiation of the average user.}
  \label{fig:s3b_dlAndPC}
\end{figure}

When looking at the different contributions to the total \gls{SAR} in figure \ref{fig:s3b_fourSourcesMatrix}, we seen that the weighted average 
\gls{SAR} from the users own device remains constant. The flying altitude is always the same so the
 \gls{UE} will, on average, radiate at the same intensity for all simulations.
 Further, the \gls{DL} \gls{SAR} from the serving \gls{UABS} is also almost constant  because the \gls{UABS} flies at 100 metres which is
above the average building. There is thus a \gls{LOS} between the \gls{UABS} and all his connected users most of the time.
The only \gls{SAR} value that increases is the \gls{DL} \gls{SAR} from other \gls{UABS}s. Because of the higher flying height, there is less path loss, the 
exposure from other \gls{UABS}s will be much higher. This is much less for microstrip patch antennae as already explained before.

\begin{figure}[h!]
  \includegraphics[width=\textwidth]{../results/s3/uFourSources.png}
  \caption{Each chart shows the total SAR to which the average user is exposed. ``My UABS" stands for the UABS that is serving our average user while ``other UABSs'' stand for 
  all other UABSs to which that user is exposed to but not served by. ``Other" UE refer the exposure from all mobile devices that does not belong that user.}
  \label{fig:s3b_fourSourcesMatrix}
\end{figure}