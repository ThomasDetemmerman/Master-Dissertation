\chapter{Scenarios}
\label{chap:scenarios}

\section{Downlink exposure versus power consumption}
\label{sec:dlexpvspc}
The authors of \cite{J2} proposed a fitness function for a joint optimization of the global network exposure and the overal power consumption.
This idea works fine for high powered basestations capable of covering multiple users. For example when optimizing towards exposure, the algorithm will decide
to activate multiple low powered basestations. This has a negative influence on the overal power consumption of the network
because each basestations requires a minimal ammount of power to be active. Analogously whill the algorithm shut down basestations when optimizing
towards power consumption. Active basestations will slighty increase their individual power consumption and therefor also increase electromagnic radiation to be able
to cover more users. This power increase is ofcourse less then the activation of a new basestation.\\

The deploymentool described in this paper makes uses of UAV's, equiped with femtocell base stations. These are low powered devices compared to basestations used in the terrestrial network 
with a more limited range. Most users are therefor connected to the \gls{UABS} placed directly above them. By increasing eighter the ammount of users
or the maximum input power. \color{red}The graph below show shows the amount of users compared to the ammount of users that are connected to their own basestations.


It becomes clear that this fitness function will primarly optimize towards exposure (unless with a lot of users)\color{black}





\section{Downlink exposure versus drone kost}
\label{sec:dlexpvspc}

- Optimizing towards exposure and kost might be a better idea
- oorspronklijke code deed dit ook al door eerst drones te overwegen die reeds actief waren
- zijn resultaten gelijkend?
- ook deze functie wordt gelimiteerd door het maximum bereik van een femtocell

\section{PL model versus exposure optimized model}
\label{sec:modelcomparison}


\section{Uplink and downlink exposure versus fly height}
\label{sec:expvsflyheight}

\section{UL and downlink exposure vs horizontal distance}
\label{sec:sarvsexp}

\section{Amount of users versus exposure}
\label{sec:dlexpvspc}

