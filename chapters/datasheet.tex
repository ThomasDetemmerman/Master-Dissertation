\chapter{Radiation patterns: datasheet}
\label{ch:radpattern}
Table \ref{tab:datasheetRadiation} gives an overview of the attenuation in the E and H plane. The first radiation pattern 
is with a square groundplane with an edge of 0.060 meter while the second pattern is more of a rectangular shape with a width of 0.0524m and a lenght of 0.0438m.
All other settings are equal as defined in \ref{sub:definingAntenna}
\begin{table*}[!ht]
\centering
\caption{Overview of attenuation in dBm}
\begin{tabular}{|l|l|l|l|l|}
\hline
 & \multicolumn{2}{c|}{pattern 1} & \multicolumn{2}{|c|}{pattern 2}\\\hline

angle & E & H & E & H \\ \hline
0 & 0,00 & 0,00 & 0  & 0 \\ \hline
10 & -0,17 & -0,14 & -0.1561 & -0.158 \\ \hline
20 & -0,67 & -0,57 & -0.5797 & -0.6257 \\ \hline
30 & -1,48 & -1,27 & -1.263 & -1.386 \\ \hline
40 & -2,57 & -2,22 & -2.193 & -2.412 \\ \hline
50 & -3,90 & -3,39 & -3.357 & -3.665 \\ \hline
60 & -5,40 & -4,73 & -4.741 & -5.099 \\ \hline
70 & -7,09 & -6,23 & -6.337 & -6.658 \\ \hline
80 & -8,82 & -7,87 & -8.136 & -8.278 \\ \hline
90 & -10,54 & -9,70 & -10.11 & -9.88 \\ \hline
100 & -12,20 & -11,84 & -12.14 & -11.34 \\ \hline
110 & -13,73 & -14,37 & -13.81 & -12.47 \\ \hline
120 & -15,04 & -17,65 & -14.42 & -13.00 \\ \hline
130 & -16,01 & -21,83 & -13.72 & -12.82 \\ \hline
140 & -16,47 & -23,63 & -12.41 & -12.08 \\ \hline
150 & -16,42 & -20,37 & -11.15 & -11.15 \\ \hline
160 & -16,05 & -17,49 & -10.21 & -10.33 \\ \hline
170 & -15,69 & -15,93 & -9.683 & -9.786 \\ \hline
180 & -15,54 & -15,54 & -9.596 & -9.596 \\ \hline
190 & -15,69 & -16,30 & -9.963 & -9.784 \\ \hline
200 & -16,05 & -18,44 & -10.79 & -10.33 \\ \hline
210 & -16,42 & -22,85 & -12.07 & -11.15 \\ \hline
220 & -16,47 & -31,23 & -13.71 & -12.07 \\ \hline
230 & -16,00 & -24,07 & -15.25 & -12.80 \\ \hline
240 & -15,03 & -18,05 & -15.65 & -12.99 \\ \hline
250 & -13,72 & -14,42 & -14.3 & -12.45 \\ \hline
260 & -12,20 & -11,81 & -12.11 & -11.33 \\ \hline
270 & -10,54 & -9,70 & -9.882 & -9.866 \\ \hline
280 & -8,82 & -7,87 & -7.859 & -8.267 \\ \hline
290 & -7,09 & -6,23 & -6.069 & -6.649 \\ \hline
300 & -5,40 & -4,73 & -4.502 & -5.093 \\ \hline
310 & -3,90 & -3,39 & -3.154 & -3.661 \\ \hline
320 & -2,57 & -2,22 & -2.029 & -2.409 \\ \hline
330 & -1,48 & -1,27 & -1.138 & -1.384 \\ \hline
340 & -0,67 & -0,57 & -0.4963 & -0.6246 \\ \hline
350 & -0,17 & -0,14 & -1143 & -0.1575 \\ \hline
\end{tabular}
\label{tab:datasheetRadiation}
\end{table*}


%%%%%%%%%%%%%%%%%%%%%%%%%%%%%%%%%%%%%%%%%%%%%%%%%%%%%%%%%%%%%%%%%%%%%%%%%%%%%%%%%%%%%%%%%%%%

\chapter{Radiation patterns: example configuration}

In listing \ref{c:exampleRadiationConfig} is a possible configuration described for a radiation pattern.
It is important to notice that this example configuration does not represent the used configuration in this master dissertation.
The \verb|radiatoinPattern|-tag consist of a \verb|slices|-tag. This tag can contain as much slices as desired.
In this example, 3 slices are defined indicated with the \verb|attenuation|-tag. This tag contains a mandatory attribute \verb|az| 
which defines the azimuth angle to which all underlying attenuation values belong.
Inside the \verb|attenuation|-tag are all attenuation values written in a \verb|value|-tag.

The tool distributes all values equally over the \ang{180} of that slice. In the example below, each \verb|attenuation|-tag contains 10 values
meaning that the exact attenuation is known every \ang{20}.

The highlighted value of -14,42 is therefore measured at an azimuth angle of \ang{0} and an elevation angle of \ang{120} (counterclockwise).

\begin{listing}[h!]
\begin{minted}[frame=single,framesep=10pt,xleftmargin=20pt,linenos,highlightlines={10}]{xml}
<radiationPattern>
    <slices>
        <attenuation az="0">
            <value>0</value>
            <value>-0.5797</value>
            <value>-2.193</value>
            <value>-4.741</value>
            <value>-8.136</value>
            <value>-12.14</value>
            <value>-14.42</value>
            <value>-12.41</value>
            <value>-10.21</value>
            <value>-9.596</value>
        </attenuation>
        <attenuation az="90">
            <value>0</value>
            <value>-0.6257</value>
            <value>-2.412</value>
            <value>-5.099</value>
            <value>-8.278</value>
            <value>-11.34</value>
            <value>-13.00</value>
            <value>-12.08</value>
            <value>-10.33</value>
            <value>-9.596</value>
        </attenuation>
        <attenuation az="180">
            <value>0</value>
            <value>-0.4963</value>
            <value>-2.029</value>
            <value>-4.502</value>
            <value>-7.859</value>
            <value>-12.11</value>
            <value>-15.65</value>
            <value>-13.71</value>
            <value>-10.79</value>
            <value>-9.596</value>
        </attenuation>
    </slices>
</radiationPattern>
\end{minted}
\caption{Example configuration of a radiation pattern.}
\label{c:exampleRadiationConfig}
\end{listing}
