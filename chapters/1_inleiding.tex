% Inspirerende caption 
%
%\begin{savequote}[0.55\linewidth]
%	``If you think you've seen this movie before, you are right. Cloud computing is based on the time-sharing model we leveraged years ago before we could afford our own computers. The idea is to share computing power among many companies and people, thereby reducing the cost of that computing power to those who leverage it. The value of time share and the core value of cloud computing are pretty much the same, only the resources these days are much better and more cost effective.''
%	\qauthor{\textasciitilde David Linthicum, author of Cloud Computing and SOA Convergence in Your Enterprise: A Step-by-Step Guide}
%\end{savequote}

\chapter{Introduction}
\label{chap:intro}

\section{Outline of the issue} %1p
\label{sec:issue}



% inspiratie kan je hier ook nog vinden:
% PREDICTION AND COMPARISON OF DOWNLINK ELECTRICFIELDAND UPLINK LOCALISED SARVALUES FOR REALISTIC INDOORWIRELESS PLANNING
Society is constantly getting more and more dependent on electronic communication. On any given moment in any given location, an electronic device
can request to connect to the bigger wireless network. Devices need more then ever to be connected starting from small IOT sensors up to self-driving cars
which needs to be supported by the existing infrastructure. 

Once again it becomes clear why we're on the eve of a new generation of cellular communication named 5G. 
This new technology is capable of handling millions of connections every square meter %to do: klopt deze hoeveelheid want lijkt wel heel veel.
while satisfying only a few microseconds of a delay and providing connections up to 10Gbps \cite{5GFeatures}.

Also in exceptional and possibly life-threatening situations, we rely on the cellular network. For example during the terrorist attacks in Zaventem, a Belgian city.
Mobile network operators saw all telecommunications drastically increasing causing moments of contention. Some operators decided to temporarily exceed the limited exposure in
order to handle all connections. \cite{baseZaventem}

Electromagnetic exposure can however not be neglected. Research shows how exesive electromagnetic radiation can cause diverse biological side effects \cite{bioeffects}.
Because of public concern, the World Health Organization had launched a large, multidisciplinary research effort which eventually concluded that there was no sufficient evidence that confirmed 
that exposure to low level electromagnetic fields harmfull is \cite{WHO}. Nevertheless remains the public very concerned about potetial health risks.

\section{Objective}
\label{sec:objective}

In this master dissertation the electromagnetic exposure of  a user is investigated taking all prominent sources into account which include the user's own mobile device, basestations
and other users their \gls{UE}.

In order to determine the magnitude of exposure to which users in a certain area are exposed, various values need to be known. 
Not only the used technology but also the position of users and basestations need to be known. 
To make this research possible, an existing planning tool is used which gives insight in users and basestation distributions. Bitrates of idividual users, power useage of 
the different electronical devices and which basestations handels which users. The tool describes in other words a fully configured network.
In this way, all needed parameters will be known.

The electromagnetic exposure will then be analysed by applieing the tool in different scenarios. During the simulations
it is investigated how various input variables influence the network.

The calculation of electromagnic exposure originating from base stations is discussed in variously discussed in litterture. Papers who convert electromagnetic exposure
 into a single value is rather limited.
Not only how electromagnetic exposure behaves but also related values like power consumption or even coverage.



\textbf{research question 1:} How can a \gls{UABS} network be optimized to minimize global exposure and overal power consumption? What are the effects on the network?\\

\textbf{research question 2:} What are the advantages and disadvantages of a model as described in research question 1 compared the the already existing pahtloss oriented model.\\

\textbf{research question 3:} How does the \gls{UABS} fly height influence uplink and downlink exposure?


%todo: onderstaande tekst (in commentaar) is een letterlijke copy van J10-RDP. Het is hier gezet als referentie. Moet nog correct verwoorden:
%In this paper, prediction algorithms are created to
%simulate and visualise electric-field strengths due to
%downlink traffic and localised SARvalues due to uplink
%traffic. Downlink exposure are expressed in terms of
%whole-body exposure due to the electric-fields E originating
%from the base stations or APs, whereas uplink
%exposure are expressed in terms of localised SAR10g
%[SAR in 10 g of tissue(8)] values due to the mobile
%device’s transmitted signal. To

\section{Structure}
\label{sec:structure}

TODO: update this section

The following chapter \ref{chap:stateoftheart} exists of several succesive sections explaining how the electromagnetic exposure of a single human being is calculated. The first section \ref{sec:calculatingexposure}
explains how the exposure is calculated between a user and a single femtocell. Section \ref{sec:combiningexposure}  defines how to combine all exposures from the different femtocells towards a single users.
Finaly, section \ref{sec:radiationpatterns} explains how directional antenna's are taken into account.

