% Inspirerende caption 
%
%\begin{savequote}[0.55\linewidth]
%	``If you think you've seen this movie before, you are right. Cloud computing is based on the time-sharing model we leveraged years ago before we could afford our own computers. The idea is to share computing power among many companies and people, thereby reducing the cost of that computing power to those who leverage it. The value of time share and the core value of cloud computing are pretty much the same, only the resources these days are much better and more cost effective.''
%	\qauthor{\textasciitilde David Linthicum, author of Cloud Computing and SOA Convergence in Your Enterprise: A Step-by-Step Guide}
%\end{savequote}

\chapter{Introduction}
\label{chap:intro}

\section{Outline of the issue}
\label{sec:issue}

% inspiratie kan je hier ook nog vinden:
% PREDICTION AND COMPARISON OF DOWNLINK ELECTRICFIELDAND UPLINK LOCALISED SARVALUES FOR REALISTIC INDOORWIRELESS PLANNING
Society is constantly getting more and more dependent on electronic communication. On any given moment in any given location, an electronic device
can request to connect to a bigger wireless medium. Devices need more then ever to be connected starting from small IOT sensors up to self-driving cars.

Once again it becomes clear why we're on the eve of a new generation of cellular communication named 5G. 
This new technology is capable of handling millions of connections every square meter %to do: klopt deze hoeveelheid want lijkt wel heel veel.
while satisfying only a few microseconds of a delay and providing connections up to 10Gbps \cite{5GFeatures}.

Also in exceptional and possibly life-threatening situations, we rely on the cellular network. For example during the terrorist attacks in Zaventem, a Belgian city.
Mobile network operators saw all telecommunications drastically increasing causing moments of contention. Some operators decided to temporarily exceed the limited exposure in
order to handle all connections. \cite{baseZaventem}

Electromagnetic exposure can however not be neglected. Research shows how exesive electromagnetic radiation can cause diverse biological side effects \cite{bioeffects} and human exposure towards these electromagnetic waves should be limited. The \gls{ICNIRP} 
suggests a limitation of 61 V/m. Also on national levels restrictions have been enforced but differ from location to location. In Brussels for example is a far more restrictive limitation enforced of 6 V/m for all sources \cite{J1, J5}.
\section{Objective}
\label{sec:objective}
The general goal is to temporarily take over the wireless communication in the event of a disaster causing the normal network infrastructure to malfunction.

One way of generating such an ad-hoc network that is easily distributed over a given area is with the aid of a drone. By attaching femtocell on these UAV's, a mobile base station is achieved. Such a device is called an \gls{UABS}

The optimal placement for each \gls{UABS} needs to be defined to make sure that as many users as possible are properly reconnected to the backbone network while satisfying certain restrictions. To make these calculations are as realistic as possible the architecture of the several buildings present in the area is described in a shapefile. 
A deployment tool calculates the optimal position of the \gls{UABS} by taking the 3D models of the building into account along with some femtocell specifications and user distribution. This deployment tool is developed by the WAVES research group, a department within Ghent University.

The deployment tool does not calculate the electromagnetic exposure of the different active users in the area.

TODO: waarin differentieert mijn MP zich????DS

\section{Structure}
\label{sec:structure}

The following chapter \ref{chap:stateoftheart} exists of several succesive sections explaining how the electromagnetic exposure of a single human being is calculated. The first section \ref{sec:calculatingexposure}
explains how the exposure is calculated between a user and a single femtocell. Section \ref{sec:combiningexposure}  defines how to combine all exposures from the different femtocells towards a single users.
Finaly, section \ref{sec:radiationpatterns} explains how directional antenna's are taken into account.

