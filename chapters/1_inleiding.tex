% Inspirerende caption 
%
%\begin{savequote}[0.55\linewidth]
%	``If you think you've seen this movie before, you are right. Cloud computing is based on the time-sharing model we leveraged years ago before we could afford our own computers. The idea is to share computing power among many companies and people, thereby reducing the cost of that computing power to those who leverage it. The value of time share and the core value of cloud computing are pretty much the same, only the resources these days are much better and more cost effective.''
%	\qauthor{\textasciitilde David Linthicum, author of Cloud Computing and SOA Convergence in Your Enterprise: A Step-by-Step Guide}
%\end{savequote}

\chapter{Introduction}
\label{chap:intro}

\section{Outline of the issue}
\label{sec:issue}

Society is constantly getting more dependent on electronic communication. On any given moment in any given location, an electronic device
can request to connect to a bigger wireless medium. More and more devices need to be connected like IOT devices starting from small sensors up to self-driving cars.

Once again it becomes clear why we're on the eve of a new generation of cellular communication named 5G. 
This new technology is capable of handling millions of connections every square meter %to do: klopt deze hoeveelheid want lijkt wel heel veel.
while satisfying only a few microseconds of a delay and providing connections up to 10Gbps \cite{bioeffects}.

Also in exceptional and possibly life-threatening situations, we rely on the cellular network. For example during the terrorist attacks in Zaventem, a Belgian city.
Mobile network operators saw all telecommunications drastically increasing causing moments of contention. Some operators decided to temporarily exceed the limited exposure in
order to handle all connections.. \cite{baseZaventem}

Electromagnetic exposure can however not be threaded lightly. Research shows how electromagnetic radiation can cause diverse biological side effects \cite{bio-effects} and human exposure to these electromagnetic waves should be limited. The ICNIRP suggests a limitation of 61 V/m.  The Brussels limitation is however more restrictive with 6 V/m for all sources.
\section{Objective}
\label{sec:objective}

In order to provide a network, even if the existing network is damaged, a deployment tool has been developed by the UGent. The idea is to attach base stations to unmanned aircraft. Such a device is called an \gls{UABS}. The tool calculates where drones need to be positioned to connect an active user to the backbone network. 

This tool requires two input files. Firstly, a so-called shapefile of the disaster area describing the location of different buildings and their design. Secondly, the time period of the disaster is provided. The tool generates random users in different locations requiring certain bitrates.

Hereafter, the optimal locations for the different \gls{UABS} are calculated. It is assumed that the entire existing network infrastructure down is and all active users, therefore, need to be reconnected.

The tool does not take human exposure into account while generating the network.
