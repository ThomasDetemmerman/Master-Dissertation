\chapter{Introduction}
\label{chap:intro}

\section{Outline of the Issue} %1p
\label{sec:issue}

% inspiratie kan je hier ook nog vinden:
% PREDICTION AND COMPARISON OF DOWNLINK ELECTRICFIELDAND UPLINK LOCALISED SARVALUES FOR REALISTIC INDOOR WIRELESS PLANNING
Society is constantly getting more and more dependent on wireless communication. On any given moment, in any given location, an electronic device
can request to connect to the bigger network. Devices need more than ever to be connected, starting from small IOT sensors up to self-driving cars
which all need to be supported by the existing infrastructure. 
It is not surprising that the city center of Ghent has an average coverage of 97\% of 4G over all telecom operators.
\cite{testaankoop}. Once again it becomes clear why we're on the eve of a new generation of cellular communication named 5G. 
%This new technology is capable of handling millions of connections every square meter %to do: klopt deze hoeveelheid want lijkt wel heel veel.
%while satisfying only a few microseconds of a delay and providing connections up to 10Gbps \cite{5GFeatures}.

Also in exceptional and possibly life-threatening situations, the public relies on the cellular network. For example during the terrorist attacks at Brussels Airport,
mobile network operators saw all telecommunications drastically increasing causing moments of contention. Some operators decided to temporarily exceed the exposure limits in
order to handle all connections \cite{baseZaventem}.

Electromagnetic exposure can however not be neglected. Research shows how excessive electromagnetic radiation can cause diverse biological side effects \cite{bioeffects}.
Because of public concern, the World Health Organization had launched a large, multidisciplinary research effort which eventually concluded that there was no sufficient evidence that confirmed 
that exposure to low level electromagnetic fields is harmful \cite{WHO}. A large part of the population remains nevertheless very concerned about potential health risks.

\section{Objective}
\label{sec:objective}
People are constantly getting exposed to several sources of electromagnetic radiation and it is important to consider this when designing a network. For this research, three prominent sources of radiation in a telecommunication
network are investigated, being: the user's own phone, all base stations and all devices from other users in the network. In order to calculate electromagnetic 
exposure from all these sources, various parameters need to be known. Not only the used technology but also the position of the users and base stations 
are required. There are several publications discussing how the electromagnetic exposure originating from base stations can be calculated. 
Papers who cover electromagnetic exposure from all these different sources and convert it into a single value are rather limited.

To make this research possible, an existing planning tool is used which gives insight in user and base station distributions.
The tool also provides information about path loss between radiators, power usage of the different electrical devices and which base station serves which user. In other words, the tool describes 
a fully configured network.
In this way, all needed parameters will be known.

The electromagnetic behaviour of the network will be analysed by applying the tool in different scenarios to give insight which variables influence the exposure and how
the network can be optimized accordingly. This leads to the following research questions:

%\textbf{Research question 1:} How can a \gls{UABS} network be optimized to minimize global exposure and overall power consumption? What are the effects on the network?\\

%\textbf{Research question 2:} What are the advantages and disadvantages of a model as described in research question 1 compared to the already existing path loss oriented model.\\

%\textbf{Research question 3:} How does the \gls{UABS} fly height influence uplink and downlink exposure?

\section{Structure}
\label{sec:structure}
Related research to the subject is discussed in chapter \ref{chap:stateoftheart}: State of the Art, explaining
electromagnetic exposure and its absorption into the body. Also the used technology such as type of antenna, type of base station and 
which infrastructure will be examined. The chapter also discusses why this master dissertation differs from other papers.
Thereafter, chapter \ref{chap:scenarios} talks about the different scenarios that will be investigated. 
Eventually, the methodology covers in chapter \ref{chap:methodology} the calculations and implementation of the different aspects excerpted in State of the Art.
Chapter \ref{chap:results} shows the results of this 
implementation for the scenarios described in chapter \ref{chap:scenarios}. 
Finally, a conclusion of these results is formed in chapter \ref{chap:conclusions}.

