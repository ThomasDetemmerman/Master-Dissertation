\chapter{Conclusions}
\label{chap:conclusions}

\section{Conlusion}
%How can a UABS network be optimized to minimize global exposure or overall power consumption? 
All conclusions are based on the default configuration as described in table \ref{table:defaultconf} unless mentioned otherwise.
Literature showed that a network can be optimized towards either the power consumption of the entire network 
or the electromagnetic exposure of the average user using a fitness function. This is because the power required to activate a new 
base station is much higher then expending it's range \cite{J1}.
The fitness function was originally applied for fixed transmission towers but can also be used 
for \gls{UABS}s as this research shows.
However, the fitness function should be used with care considering that \gls{UABS}s can be placed anywhere compared to 
the transmission towers from \cite{J1} who have a determined position. This causes that a lot of users get a \gls{UABS}
all by themselves in an exposure optimized network because this is the best approach to minimize exposure.
A power consumption optimized network on the other hand will try to limit the number of drones 
in order to save energy. So as a rule of thumb: an exposure optimized network will result in a lot of low powered devices (increasing the overall power consumption)
while a power consumption optimized network results in a few high powered devices (increasing the exposure of the average user).
A power consumption optimized network is thus cheaper because less drones are involved. 
Moreover, the results show that the electromagnetic radiation in a power consumption optimized network (with high powered \gls{UABS}s)
is far below the thresholds enforced by the Flemish government.

The user's main sources of exposure are the user's own device and the \gls{UABS} who is serving him followed by all
other \gls{UABS}s in the network. 
When the population increases, also the exposure from other people their \gls{UE} increases. However, the electromagnetic
 exposure from these devices can be ignored compared to the much higher electromagnetic exposure from the other sources. 
A bigger population also cause an increase in number of drones. So when the population grows, the exposure of the 
user increases mainly because of a growing exposure from other \gls{UABS}s that are not serving the user. Other sources barely contribute anything to the overall exposure
of this user. An exposure optimized network will limits the total exposure mainly by trying to reduce the exposure from other \gls{UABS}s.

%1)	How does the network behave differently after the introduction of a realistic antenna?
An equivalent isotropic radiator has higher exposure and coverage for less power compared to realistic antennae like microstrip patch antenna.
This is because  of the absence of attenuation in \gls{EIRP} antennae. In other words, an \gls{EIRP} antenna can achieve the same coverage with less
resources like power and number of drones but is unfortunately not a realistic antenna. The network will thus profit from an antenna which has a big 
aperture angle so the attenuation for the users on the ground would be minimal.


An power consumption optimized network has the lowest exposure around 70 to 80 metres. An electromagnetic exposure in an 
exposure optimized network only increases when the flying height increases. The number of required drones decrease 
when the flying height becomes larger. When also considering the results from \cite{J27_backhaul} where an flying altitude from 
80 metres is suggested for an optimal access and backhaul connectivity, a flying height 
of 80 metres is also here proposed for the city centre of Ghent.

In conclusion, a power consumption optimized network is proposed with a fixed flying height of 80 metres. A microstrip patch 
antenna with a sufficient large aperture angle is a good starting point. However, different antenna configurations should 
be investigated 

\section{Future work}
The tool has been extended so any possible antenna in any possible direction is supported. Comparing different types 
of antennae is however outside the scope of this research.
Conclusions on how the network performs has thus been investigated for the already existing fictional omnidirectional antenna and a 
realistic microstrip patch antenna. The reason for the chosen microstrip patch antenna is solely based on literature.

