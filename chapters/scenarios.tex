\chapter{Scenarios}
\label{chap:scenarios}

\section{A single user}

A first scenario will investigate how $SAR_{10g}$ is influenced in an isolated environment meaning there is nor influence 
from other base stations nor other \gls{UE}. The tool will provision one single drone and position it directly above the user.
These results will however depend on the position of the user. If the randomly generated location of the user is indoor, the expected height
of the user is half of the height of the building. Investigating the electromagnetic exposure and the drone power consumption depends however on the
distance between both user and \gls{UABS}. For a more consistent result, the user will therefore be positioned outside when systematically 
increasing the fly height. Another considered variable is the transmitting power of the \gls{UABS}.

This scenario deducted with two type of antennas. First, an \gls{isotropicradiator} will be used and thereafter a realistic antenna.
It is expected that after the introduction of an realistic antenna, the user coverage will decrease.

The tool will therefore run different simmulations. The first group is with an \gls{isotropicradiator}. A first set of simmulations
is with a fixed flyheight of 100m which is the proposed flyheight by \cite{J2} but with a variable transmit power of the base station.
The set investigates the influence of a variable flyheight with a constant maximum transmit power of 33dBm as defined in \cite{J2}.

Both series will also be investigated with a realistic radiation pattern. The user coverage will be compared. 



\section{Increasing traffic with only one drone available}
The previous scenario will be extended for an increasing amount of users. 


\section{Increasing traffic with an undifend amount of drones}
When more drones are available, an optimization strategy can be applied. The tool checks the capacity of the basestations and decides thereafter
wich basestation the user should be connected to. The original algorithm checks all pahts between the user that need to be connected with 
all drones. Thereafter, the drones which path experience the least pathloss and still has the capacity to cover an addition user will be selected.
The authors from \cite{J1} proposed however annother optimization strategy which tries to minimize electromagnetic exposure and 
power consumption.

The input variables flyheight, transmit power and number of users will be used to see how electromagnetic exposure, power consumption en number of drones are influenced for
different optimization strategies and type of antennas.

Since there is no fixed budget limitation, the number of drones are unlimited. The tool will therefore try to connect each user and
coverage will be expressed in number of drones required to cover as much users as possible instead of having a limited number of drones  
as in scenario and therefore has only a limited coverage expressed in percentage.