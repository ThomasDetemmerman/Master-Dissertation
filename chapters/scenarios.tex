\chapter{Scenarios}
\label{chap:scenarios}
The tool supports multiple configurations and the behavior will be different for most these configurations. Three main scenarios 
will be investigated, order based on the network complexity. Within each scenario, different configurations will be applied.
For the first scenario, only one user with one drone will be present in the network. The network will thereafter be expanded
for multiple users but with still only one drone available. Eventually, that last restriction will be dropped meaning 
that the third scenario covers multiple users with unlimited number of drones. 
Table \ref{table:defaultconf} show the default configuration which values are always applicable unless mentioned otherwise.

\begin{table}[!htb]
\centering
\begin{tabular}[t]{ll}
        \toprule
        \multicolumn{2}{l}{\textbf{Broadband cellular network}} \\
        \hline
        \hspace{3mm}  technology        & LTE     \\
        \hspace{3mm}  frequency         & 2.6 GHz \\
        \hline
        \multicolumn{2}{l}{\textbf{Carrier}} \\
        \hline  
        \hspace{3mm}  Carrier power        & 13.0 A   \\
        \hspace{3mm}  average carrier speed        & 12.0 m/s \\
        \hspace{3mm}  Average carrier power usage      & 17.33 Ah    \\
        \hspace{3mm}  Carrier battery voltage       & 22.2 V \\
        \hline
        \multicolumn{2}{l}{\textbf{Femtocell antenna}} \\
        \hline  
        \hspace{3mm}  maximum $P_{tx}$          & 33 dBm   \\
        \hspace{3mm}  antenna  direction        & downwards (az: \ang{0}; el: \ang{90})    \\ 
        \hspace{3mm}  gain                      & 4 dBm   \\ 
        \hspace{3mm}  feeder loss               & 2 dBm   \\ 
        \hspace{3mm}  implementation loss       & 0 dBm   \\
        \hspace{3mm}  radiation pattern         & \acs{EIRP} or \\ microstrip patch antenna\\
        \hspace{3mm}  height                    & 100m  \\
        \hline
        \multicolumn{2}{l}{\textbf{\acs{UE} Antenna}} \\
        \hline 
        \hspace{3mm} height                     & 1.5m from the floor       \\ 
        \hspace{3mm}  gain                      & 0 dBm   \\ 
        \hspace{3mm}  feeder loss               & 0 dBm   \\ 
        \hspace{3mm}  radiation pattern         & \acs{EIRP}  \\
        \hspace{3mm}  number present in the network         & 224 m  \\
        \toprule
\end{tabular}
\caption{Overview of default configuration values.}
\label{table:defaultconf}
\end{table}

%%%%%%%%%%%%%%%%%%%%%%%%%%%%%%%%%%%% scenario 1
\section{A single user}
\label{sec:s1}

This first scenario will investigate how $SAR_{10g}$ and power consumption is influenced in an isolated environment meaning there is nor influence 
from other base stations nor other \gls{UE}. The tool will provision one single drone and position it directly above the user.
These results will however depend on the position of the user. If the randomly generated location of the user is indoor, 
the flying height of the drone might obstructed by the building where the user resides, causing the user to be uncovered. If this is not the case,
the expected altitude of the user is half of the height of the building meaning that the user would be closer to the \gls{UABS} as 
if he would have been outdoors. For a more consistent result, the user will therefore be positioned outside when systematically 
increasing the flying height. 

Another considered variable will be the transmit power of the antenna.
\gls{LTE} makes usages of power control meaning that no more power will be used then strictly necessary. The actual 
transmit power therefore ranges between 0 and the maximum input power. This power is zero when either no user is 
present or the user is so far away that the actual transmit power would exceed the maximum transmission power.
Increasing the maximum transmission power won't influence the power consumption or $SAR_{10g}$ because the \gls{UABS} won't use more
then strictly required. It is therefore more useful to match the actual transmit power against a variable flying height.

This scenario investigates $SAR_{10g}$, power consumption and minimal transmission power. The used optimization strategy is not important.
The optimization algorithm decides which user will be connected to which drone in order to reach a certain goal. Since only one user and one
\gls{UABS} are available, both optimization strategies will behave identical. These values will be checked when using a fictional \gls{isotropicradiator} and 
a realistic antenna.

The user gets a fixed position. The exact location doesn't matter as long as it is outside. For this experiment is choses for the 
`Koningin Maria Hendrikaplein', a square just next to the train station of Ghent.  Doing so will force the \gls{UE} 
to always be at the same height of 1.5 meters. The conclusions will be based on $SAR_{10g}$, power consumption and transmission power.
These output values depend on fly height and type of antenna. An overview can be found in table \ref{table:confOverviewScenario1}

Note that there is no explicit restriction on the number of drones in table \ref{table:confOverviewScenario1}. The deployment tool initialy places 
\gls{UABS}s above each user and it is the optimzation strategy that decides which of these potentiol positions remain. Since there is only one user,
there can also be only one drone.

\begin{table}[!htb]
    \begin{minipage}{.5\linewidth}
      \centering
        \begin{tabular}{|l|c|l|}
        \hline
        \textbf{Parameter}              & \textbf{Value}          \\   \hline 
        x position user               & 3.711198       \\    
        y position user               & 51.036747          \\ 
        shadow margin user             & -3.0398193 \\
        number of users                & 1 \\
        \hline
        \end{tabular}
    \end{minipage}%
    \begin{minipage}{.5\linewidth}
      \centering
            \begin{tabular}{|l|l|}
            \hline
            \textbf{Input variables  }              & \textbf{Output variables}          \\   \hline 
            type of antenna                & $SAR_{10g}$               \\ 
            flying height                  & power consumption             \\ 
                                           &  minimal $P_{tx}$ \\ 
                                           & \\
            \hline
            \end{tabular}
    \end{minipage} 
        \caption{Overview of the configuration.}
        \label{table:confOverviewScenario1}
\end{table}


%%%%%%%%%%%%%%%%%%%%%%%%%%%%%%%%%%%% scenario 2


\section{Increasing traffic with only one drone available}

This scenario investigate the same behavior  as the previous. Still with one drone but for a higher number of users. The scenario can be divided into two groups. One for a variable 
flying height but with a fixed number of 224 users which is the number of active users on an average day at 5 p.m. meaning
 which means it is rush hour resulting in the highest number of simmulatanious users for the day\cite{J2}. The other 
scenario has a fixed flying height of 100 m as recommended by \cite{J2} but with a variable number of users. To enforce the tool to only use one drone, a facility capacity is set to one 
which implies that there is only one spot available in the facility where the \gls{UABS}s are stored. The tool will still generate as much potential places 
as there are users in the network. But when the optimization algorithm is done, only one drone will remain.

\begin{table}[!htb]
    \begin{minipage}{.5\linewidth}
      \centering
        \begin{tabular}{|l|c|l|}
        \hline
        \textbf{Parameter}              & \textbf{Value}          \\   \hline 
        facility capacity               & 1        \\    
        &  \\ 
        & \\ 
        & \\ 
        \hline
        \end{tabular}
    \end{minipage}%
    \begin{minipage}{.5\linewidth}
      \centering
            \begin{tabular}{|l|l|}
            \hline
            \textbf{Input variables  }              & \textbf{Output variables}          \\   \hline 
            type of antenna                & $SAR_{10g}$               \\ 
            flying height                     & power consumption             \\ 
            number of users                & user coverage            \\
            optimization strategy         &                               \\ 
            \hline
            \end{tabular}
    \end{minipage} 
        \caption{Overview of the configuration.}
        \label{table:confOverviewScenario2}
\end{table}

Four possible configurations are possbile because there are two antennae available (\gls{isotropicradiator} and a realistic antenna) which can both operate in an power consumption optimzed network or an exposure optimized 
network. These four configurations are investigated for the two groups mentioned above.
Both groups can further be divided in four series where 
The $SAR_{10g}$, power consumption and user coverage will be investigated for both groups.
The only available drone will be positioned at the fly height of 100 m as recommended in \cite{J2}. For the second case, the same output variables are investigated 
for a varying fly height but with a fixed number of 224 users.
Both cases will be investigated for the two types of antennae: the fictional \gls{isotropicradiator} and the microstrip patch antenna.


%%%%%%%%%%%%%%%%%%%%%%%%%%%%%%%%%%%% scenario 3


\section{Increasing traffic with an undifend amount of drones}
\begin{table}[!htb]
      \centering
            \begin{tabular}{|l|l|}
            \hline
            \textbf{Input variables  }              & \textbf{Output variables}          \\   \hline 
            type of antenna                & $SAR_{10g}$               \\ 
            flying height                   & power consumption             \\ 
            number of users                & user coverage            \\ 
            optimization strategy           & \\
            \hline
            \end{tabular}
        \caption{Overview of the configuration.}
        \label{table:confOverviewScenario2}
\end{table}


When more drones are available, an optimization strategy can be applied. The tool checks the capacity of the basestations and decides thereafter
wich basestation the user should be connected to. The original algorithm checks all pahts between the user that need to be connected with 
all drones. Thereafter, the drones which path experience the least pathloss and still has the capacity to cover an addition user will be selected.
The authors from \cite{J1} proposed however annother optimization strategy which tries to minimize electromagnetic exposure and 
power consumption.

The input variables flying height, transmit power and number of users will be used to see how electromagnetic exposure, power consumption en number of drones are influenced for
different optimization strategies and type of antennas.

Since there is no fixed budget limitation, the number of drones are unlimited. The tool will therefore try to connect each user and
coverage will be expressed in number of drones required to cover as much users as possible instead of having a limited number of drones  
as in scenario and therefore has only a limited coverage expressed in percentage.
