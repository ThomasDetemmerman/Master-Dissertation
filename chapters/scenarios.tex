\chapter{Scenarios}
\label{chap:scenarios}
The tool supports multiple configurations and the tool will behave differently for most configurations. Three main scenarios 
will be investigated, order based on the network complexity. Within each scenario, different configurations will be applied.
For the first scenario, only one user with one drone will be present in the network. The network will thereafter be expanded
for multiple users but with still only one drone available. Eventually, that last restriction will be dropped meaning 
that the third scenario covers multiple users with unlimited number of drones. Table \ref{table:defaultconf} show the default configuration which 
always applicable unless mentioned otherwise.

\begin{table}[!htb]
\centering
\begin{tabular}[t]{ll}
        \toprule
        \multicolumn{2}{l}{\textbf{broadband cellular network}} \\
        \hline
        \hspace{3mm}  technology        & LTE     \\
        \hspace{3mm}  frequency         & @ 2.6 GHz \\
        \hline
        \multicolumn{2}{l}{\textbf{Carrier}} \\
        \hline  
        \hspace{3mm}  Carrier power        & 13.0 A   \\
        \hspace{3mm}  average carrier speed        & 12.0 m/s \\
        \hspace{3mm}  Average carrier power usage      & 17.33 Ah    \\
        \hspace{3mm}  Carrier battery voltage       & 22.2 V \\
        \hline
        \multicolumn{2}{l}{\textbf{femtocell antenna}} \\
        \hline  
        \hspace{3mm}  maximum $P_{tx}$          & 33 dBm   \\
        \hspace{3mm}  antenna  direction        & downwards (az: \ang{0}; el: \ang{90})    \\ 
        \hspace{3mm}  gain                      & 4 dBm   \\ 
        \hspace{3mm}  feeder loss               & 2 dBm   \\ 
        \hspace{3mm}  implementation loss       & 0 dBm   \\
        \hspace{3mm}  radiation pattern         & \shortstack[l]{\gls{EIRP} or \\ microstrip patch antenna } \\
        \hspace{3mm}  height                    & 100m  \\
        \hline
        \multicolumn{2}{l}{\textbf{\gls{UE} Antenna}} \\
        \hline 
        \hspace{3mm} height                     & 1.5m from the floor       \\ 
        \hspace{3mm}  gain                      & 0 dBm   \\ 
        \hspace{3mm}  feeder loss               & 0 dBm   \\ 
        \hspace{3mm}  radiation pattern         & \gls{EIRP}  \\
        \toprule
\end{tabular}
\caption{Overview of default configuration values.}
\label{table:defaultconf}
\end{table}

%%%%%%%%%%%%%%%%%%%%%%%%%%%%%%%%%%%% scenario 1
\section{A single user}
\label{sec:s1}

This first scenario will investigate how $SAR_{10g}$ is influenced in an isolated environment meaning there is nor influence 
from other base stations nor other \gls{UE}. The tool will provision one single drone and position it directly above the user.
These results will however depend on the position of the user. If the randomly generated location of the user is indoor, 
the flying height of the drone might obstructed by the building where the user resides, causing the user to be uncovered. If this is not the case,
the expected altitude of the user is half of the height of the building meaning that the user would be closer to the \gls{UABS} as 
if he would have been outdoors. For a more consistent result, the user will therefore be positioned outside when systematically 
increasing the fly height. Another considered variable is the transmitting power of the \gls{UABS}.

This scenario deducted with two type of antennas. First, an \gls{isotropicradiator} will be used and thereafter a realistic antenna.
It is expected that after the introduction of an realistic antenna, the user coverage will decrease.

The scenario consist of two groups with each 3 series of simulations. The first group is with an \gls{isotropicradiator} and the second 
will be identical but with a realistic radiation pattern. The first series of simulations
investigates $SAR_{10g}$ and power consumption of the network for a variable flying height but a fixed maximum transmission power of 
33dBm as defined in \cite{J2}. The second series 
investigates the minimal require transmit power by the used antenna for a fixed flying height of 100 m 
which is the proposed flying height by \cite{J2} but with a variable transmit power of the base station.

The user gets a fixed position. The exact location doesn't matter as long as it is outside. For this experiment is choses for the 
`Koningin Maria Hendrikaplein', a square just next to the train station of Ghent.  Doing so will force the \gls{UE} 
to always be at the same height of 1.5 meters. The conclusions will be based on $SAR_{10g}$, power consumption and transmission power.
These output values depend on fly height and type of antenna. An overview can be found in table \ref{table:confOverviewScenario1}

\begin{table}[!htb]
    \begin{minipage}{.5\linewidth}
      \centering
        \begin{tabular}{|l|c|l|}
        \hline
        x possition user               & 3.711198       \\    
        y possition user               & 51.036747          \\ 
        shadow margin user             & -3.0398193315963473 \\
        height of the \gls{UE}         & 1.5m                      \\ 
        frequency                      & 2600Hz                   \\ 
        antenna  direction             & downwards (az: \ang{0}; el: \ang{90})    \\ 
        \hline
        \end{tabular}
    \end{minipage}%
    \begin{minipage}{.5\linewidth}
      \centering
            \begin{tabular}{|l|l|}
            \hline
            Input variables                & Output variables          \\   \hline 
            type of antenna                & $SAR_{10g}$               \\ 
            fly height                     & power consumption             \\ 
               &   \\ 
            \hline
            \end{tabular}
    \end{minipage} 
        \caption{Overview of the configuration.}
        \label{table:confOverviewScenario1}
\end{table}


%%%%%%%%%%%%%%%%%%%%%%%%%%%%%%%%%%%% scenario 2


\section{Increasing traffic with only one drone available}
The previous scenario will be extended for an increasing amount of users. 

\begin{table}[!htb]
    \begin{minipage}{.5\linewidth}
      \centering
        \begin{tabular}{|l|c|l|}
        \hline
        todo               & todo        \\    
        todo               & todo\\ 
        todo               & todo                     \\ 
        frequency                      & 2600Hz                   \\ 
        \hline
        \end{tabular}
    \end{minipage}%
    \begin{minipage}{.5\linewidth}
      \centering
            \begin{tabular}{|l|l|}
            \hline
            Input variables                & Output variables          \\   \hline 
            type of antenna                & $SAR_{10g}$               \\ 
            fly height                     & power consumption             \\ 
            number of users                & user coverage            \\ 
            \hline
            \end{tabular}
    \end{minipage} 
        \caption{Overview of the configuration.}
        \label{table:confOverviewScenario2}
\end{table}

The $SAR_{10g}$, power consumption and user coverage will be investigated for an increasing amount of users ranging from 50 to 650 in steps of 50.
The only available drone will be positioned at the fly height of 100 m as recommended in \cite{J2}. For the second case, the same output variables are investigated 
for a varying fly height but with a fixed number of 224 users. This case will be applied in the city center of Ghent, assuming it is an average 
day at 5 p.m. which means it is rush hour resulting in the highest number of simmulatanious users for the day\cite{J2}. 
Both cases will be investigated for the two types of antennae: the fictional \gls{isotropicradiator} and the microstrip patch antenna.


%%%%%%%%%%%%%%%%%%%%%%%%%%%%%%%%%%%% scenario 3


\section{Increasing traffic with an undifend amount of drones}

\begin{table}[!htb]
    \begin{minipage}{.5\linewidth}
      \centering
        \begin{tabular}{|l|c|l|}
        \hline
        todo               & todo        \\    
        todo               & todo\\ 
        todo               & todo                     \\ 
        frequency                      & 2600Hz                   \\ 
        \hline
        \end{tabular}
    \end{minipage}%
    \begin{minipage}{.5\linewidth}
      \centering
            \begin{tabular}{|l|l|}
            \hline
            Input variables                & Output variables          \\   \hline 
            type of antenna                & $SAR_{10g}$               \\ 
            fly height                     & power consumption             \\ 
            number of users                & number of drones            \\ 
            \hline
            \end{tabular}
    \end{minipage} 
        \caption{Overview of the configuration.}
        \label{table:confOverviewScenario2}
\end{table}


When more drones are available, an optimization strategy can be applied. The tool checks the capacity of the basestations and decides thereafter
wich basestation the user should be connected to. The original algorithm checks all pahts between the user that need to be connected with 
all drones. Thereafter, the drones which path experience the least pathloss and still has the capacity to cover an addition user will be selected.
The authors from \cite{J1} proposed however annother optimization strategy which tries to minimize electromagnetic exposure and 
power consumption.

The input variables flying height, transmit power and number of users will be used to see how electromagnetic exposure, power consumption en number of drones are influenced for
different optimization strategies and type of antennas.

Since there is no fixed budget limitation, the number of drones are unlimited. The tool will therefore try to connect each user and
coverage will be expressed in number of drones required to cover as much users as possible instead of having a limited number of drones  
as in scenario and therefore has only a limited coverage expressed in percentage.