\chapter{State of the art}
\label{chap:stateoftheart}

\section{Deployment tool for an UAV network}
\label{chap:stateoftheart:deploymenttool}

The tool is also able to calculate a more precisly pathloss since 


The calculation of electromagnetic radiation require several input values which need to be known. To fullfill this, a deployment tool developped by
The WAVES research group at UGent has therefore developed a deployment tool which distributes UAVs equiped with femtocell base stations. These kind of UAVs will be called 
a \gls{UABS}. 

A deployment tool for an UAV-aided emergency network is described in \cite{J2}. The idea is that in case of a disaster, the existing network might be damaged and won't be able 
to handle all users who are trying to reconnect to the backbone network. A fast deployable network is suggested in \cite{J2} by using \gls{UABS}s. These are UAVs equiped with femtocell base stations
and will be distributed over the disaster area, orchestrated by the deployment tool. 

%The optimal placement for each \gls{UABS} needs to be defined to make sure that as many users as possible are properly reconnected to the backbone network while satisfying certain restrictions. 
%To make these calculations as realistic as possible the architecture of the several buildings present in the area is described in a shapefile. 
%A deployment tool calculates the optimal position of the \gls{UABS} by taking the 3D models of the building into account along with some femtocell specifications and user distribution. This deployment tool is developed by the WAVES research group, a department within Ghent University.

The deployment tool will try to calculate the optimal placement for each \gls{UABS} and requires therefore a description of the area where the UAV-aided network needs to 
be deployed. This is done with the use of so-called shape files. Theses files contains tree dimensional descriptions of the buildings present in the area and are
key values in approaching results as realistic as possible. Furthermore, the tool also requires a time period and a configuration file containing technical specifications of the type of \gls{UABS} that is being used. 
The tool will thereafter randomly distribute users over the area and assigns a certain bitrate to them. \\
\\
In a second phase, the optimal possition for each \gls{UABS} is calculated. This is done by trying to locate a \gls{UABS} above each active user. Two options are possible.
If a flighheigt is defined, a basestations is placed above each user at the given height, unless a building is abstructing it's location. Then, no basestation will be located above that user.
If no flighheigt is given to the tool, the basestation is located 4 meters above the outdoor user or 4 meters above the building where the indoor user resides. 
The later is only allowed if the suggested heigt remains below the given maximum allowed height. \\
\\
Finally, all  \gls{UABS} are sorted on wether they were active or not, followed by the increasing pathloss from each \gls{UABS} to that user.
So the algirtm starts by checking for each active \gls{UABS} if it can cover the user. If this is the case, the user will be connected to this \gls{UABS}. If not,
the second active basestation with a (slightly) worse pathloss is considered. If no active basestation is suitable, inactive \gls{UABS} are considered. The user remains uncovered if no \gls{UABS}
is found. The reasoning behind first only considering basestations that are already active is the hight cost that comes allong with each drone. \\
\\
Up till now, the tool has only calculated some suggestions. The effective provisioning is done in the fourth phase where drones are sorted by the ammount of users it covers. As long as \gls{UABS}
are available in the facility where they reside, \gls{UABS} are provisioned and its users are marked as covered.


\section{Electromagnetic exposure}
\subsection{general} % (fold)
\label{sub:general}
The goals of this master disertation is the investigation of electromagnetic exposure. 
% subsection general (end)

\subsection{Uplink exposure} % (fold)
\label{sub:Uplink exposure}

\subsubsection{Specific absorption rate}

todo: de 10g slaat al op localized, vandaar dat het maar 10g is, anders is het whole-body
todo: we kunnen niet sar10gmax gebruiken want This means that the SAR calculations will be worst-case and possibly an overestimation of the real localised SAR. (herwoorden voor plagiaat)
Human exposure caused by downlink traffic is a not negligible asset. However, telecommunications is not a one-way street. When connecting to a UMTS network, also uplink data caused by the \gls{UE} should be considered.
\gls{UE} generates, just like femtocells, electromagnetic waves to which a user is exposed. A part of this radiation goes to the femtocell, another part enters the body of its user. How much electormagnic strenghts enters the body is defined as \gls{SAR} and is measured with 10g biological tissue which represents the human skin. This value will from now on be expressed as $SAR_{10g}$. 
A mobile device induces two types of exposure: local and whole-body. 


\subsection{Downlink exposure} % (fold)
\label{sub:Downlink exposure}

% subsection Downlink exposure (end)

\subsection{Joining uplink and downlink exposure} % (fold)
\label{ssub:Joining uplink and downlink exposure}

% subsubsection Joining uplink and downlink exposure (end)

\subsection{Regulations} % (fold)
\label{ssub:Regulations}

% subsubsection Regulations (end)
\section{Technologies}

