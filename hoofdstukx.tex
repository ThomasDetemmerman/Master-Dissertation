\chapter{Dit is een hoofdstuk}

\section{Sectie}

Een sectie.

\section{Nog een sectie}
\label{sec:nogeensectie}

Nog een.

\subsection{Een subsectie}

\begin{figure}[htb]
\begin{center}
%optioneel om evt wat witruimte van je figuur af te snoepen
%\vspace{-.3cm}
 \includegraphics[keepaspectratio,width=0.5\textwidth]{fig/ruglogo}
% analoog
%\vspace{-0.6cm}
 \caption{Het RUG logo}
 \label{fig:ruglogo}
%analoog
%\vspace{-.6cm}
\end{center}
\end{figure}

En wat tekst natuurlijk.

\section{En nu de nuttige dingen}

Wist je dat omniORB\cite{omniorb} nergens in een RFC, zoals in \cite{2-BIT}, beschreven staat ? Om meer referenties\cite{omniorb, 2-BIT} te showen. Bovendien zorgt $\backslash$ gevolgd door een spatie ervoor dat een spatie na
een punt niet als begin van een nieuwe zin gezien wordt, zoals bv.\ hier (normaal zou je dit bv. hebben). Bij het begin van een zin laat LateX iets meer ruimte. Een tilde daarentegen
zorgt ervoor dat een spatie nooit opgebroken wordt in een nieuwe lijn, te gebruiken bv.\ in Figuur~\ref{fig:ruglogo} zodat het figuurnummer niet
gesplitst wordt van ``Figuur''.

Meteen is het invoegen van figuren (inclusief caption en labelen) ook gedemonstreerd. En men kan ook secties labelen om er naar te verwijzen: bv.\ in
sectie~\ref{sec:nogeensectie} staat er eigenlijk niks.

Ook zeer mooi is het gebruik van ldots om 3 puntjes --- o ja, voor ik het vergeet, meerdere ``-'' na elkaar geven langere horizontale strepen, bv.\ 1--2 --- op een
deftige manier te zetten,~\ldots\ Die laatste $\backslash$ is enkel nodig omdat het het einde van de zin is, en LaTeX anders daar geen witruimte
invoegt. Zie je de handigheid van LaTeX al (of wordt het net te ingewikkeld) ?

In verband met splitsen, ditmagnooitgesplitstworden zal nooit gesplitst worden. Ditwoordsplitsthier daarentegen wel. ditmagnooitgesplitstworden
ditmagnooitgesplitstworden ditmagnooitgesplitstworden ditmagnooitgesplitstworden (zie hier waarom je soms zal *moeten* aanduiden waar je een woord kan
splitsen. Of je krijgt een Overfull hbox (78.19391pt too wide) in paragraph at lines 43--45. Dit kan met hyphenation in boek.tex.






