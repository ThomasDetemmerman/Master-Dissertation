\chapter{State of the art}
\label{chap:stateoftheart}

\section{Deployment tool for an UAV network}
\label{chap:stateoftheart:deploymenttool}


A deployment tool for an UAV-aided emergency network is described in \cite{J2}. The idea is that in case of a disaster, the existing network might be damaged and won't be able 
to handle all users who are trying to reconnect to the backbone network. A fast deployable network is suggested in \cite{J2} by using \gls{UABS}s. These are UAVs equiped with femtocell base stations
and will be distributed over the disaster area, orchestrated by the deployment tool. 

%The optimal placement for each \gls{UABS} needs to be defined to make sure that as many users as possible are properly reconnected to the backbone network while satisfying certain restrictions. 
%To make these calculations as realistic as possible the architecture of the several buildings present in the area is described in a shapefile. 
%A deployment tool calculates the optimal position of the \gls{UABS} by taking the 3D models of the building into account along with some femtocell specifications and user distribution. This deployment tool is developed by the WAVES research group, a department within Ghent University.

The deployment tool will try to calculate the optimal placement for each \gls{UABS} and requires therefore a description of the area where the UAV-aided network needs to 
be deployed. This is done with the use of so-called shape files. Theses files contains tree dimensional descriptions of the buildings present in the area and are
key values in approaching results as realistic as possible. Furthermore, the tool also requires a time period and a configuration file containing technical specifications of the type of \gls{UABS} that is being used. 
The tool will thereafter randomly distribute users over the area and assigns a certain bitrate to them. \\
\\
In a second phase, the optimal possition for each \gls{UABS} is calculated. This is done by trying to locate a \gls{UABS} above each active user. Two options are possible.
If a flighheigt is defined, a basestations is placed above each user at the given height, unless a building is abstructing it's location. Then, no basestation will be located above that user.
If no flighheigt is given to the tool, the basestation is located 4 meters above the outdoor user or 4 meters above the building where the indoor user resides. 
The later is only allowed if the suggested heigt remains below the given maximum allowed height. \\
\\
Finally, all  \gls{UABS} are sorted on wether they were active or not, followed by the increasing pathloss from each \gls{UABS} to that user.
So the algirtm starts by checking for each active \gls{UABS} if it can cover the user. If this is the case, the user will be connected to this \gls{UABS}. If not,
the second active basestation with a (slightly) worse pathloss is considered. If no active basestation is suitable, inactive \gls{UABS} are considered. The user remains uncovered if no \gls{UABS}
is found. The reasoning behind first only considering basestations that are already active is the hight cost that comes allong with each drone. \\
\\
Up till now, the tool has only calculated some suggestions. The effective provisioning is done in the fourth phase where drones are sorted by the ammount of users it covers. As long as \gls{UABS}
are available in the facility where they reside, \gls{UABS} are provisioned and its users are marked as covered.

%%%%%%%%%%%%%%%%%%%%%%%%%%%%%%%%%%%%%%%%%%%%%%%%%%%%%%%%%%%%%%%%%%%%%%%%%%%%%%%%%%%%%%%%%%%%%%%%%%%%%%%%%%%%%%%%%%%%%%%%%%%%%%%%%%%%%%%%%%%%%%%%%

\section{Electromagnetic exposure}
\label{chap:capbaseddeploymenttool}

The described network in \ref{chap:stateoftheart:deploymenttool} tries to connect the user to the best basestation based on pathloss while provisioning as least basestations as possible.
However, when developing a network it is also important to take electromagnetic field exposure and power consumption into account. Several models have been published on how to calculate downlink electromagnetic field strengts. 
The calculations in this paper are based on \cite{J1} which not only describes \gls{DL} exposure but also defines a fitness function describing on how the network is performing. 
Grid points are distributed over the area seperated with a fixed distance. Firstly, the \gls{DL} exposure in a grid point fom each basestation is calculated and each value is combined resulting
in the total exposure in that point. This is repeaded for every single gridpoint.
Secondly, the electromagnetic exposure for each grid point is calculated and the $50^{th}$ and $95^{th}$ percentile are used to find the global network exposure. \\

Based on the formula described in equation \ref{eq:singleexposure} of section \ref{sec:calculatingexposure}, it will become clear that exposure increases exponentially compared to the input power
of an antenna and optimizing towards powerconsumption should therefor also decrease it's exposure\\

However, when taking the electromagnetic exposure and powerconsumption of the entire network, both value's become inversely proportional. Take for instance two users with each their own nearby basestation.
When optimizing towards exposure, both basestations will radiate just enough to cover the user near them. When optmizing towards power consumption, it will be better to shut down one of the basestations because
each basestation requires a minimal ammount of power. The other basestation will rediate slightly more by using slightly more energy but less then keeping the other basestation alive. \\

This results in a fitness functions with waiting factors indicating the importance of eighter electromagnetic fields or powerconsumption for a joint optimization of both values.
It is espected that this approach is also usefull in a UAV-aided network since drones have a limited battery capacity. \\


%- de paper gaat over pc en expsoure bij bs maar is ook van toepas voor drones met betperkte batterij. Mogelijke conslusie is dat optimaliseren naar 
%energie belangrijker is. Exposure is toch weinig bij femtocells en een drone heeft maar een beperkte batterij
%- For evaluating the power consumption, the model of [1, 21] is used for the base station in operational
%-while it is assumed that the base station consumes 45 percent of their maximal SISO (Single-Input Single-Output) power consumption (i.e., 45 percent of 1674 W = 753.3 W) during sleep mode as proposed by [22].
%- resultaten van deze paper?

%%%%%%%%%%%%%%%%%%%%%%%%%%%%%%%%%%%%%%%%%%%%%%%%%%%%%%%%%%%%%%%%%%%%%%%%%%%%%%%%%%%%%%%%%%%%%%%%%%%%%%%%%%%%%%%%%%%%%%%%%%%%%%%%%%%%%%%%%%%%%%%%%

\section{whipp}
Users are not only exposed to electric fields originating from femtocells but also by their own devices. The authors of \cite{J10_RDP} make use of a \gls{whipp} tool. A collection
of heuristic algorithms used for designing an optimal indoor network by placing femtocells on the ground plan of the considered building. The paper describes how to calculate localised SAR values originating from a \gls{UE} to an UMTS femtocell.
The paper's model is validated using a Nokia N95. The uplink SAR10g depend on the maximum possible SAR10g and transmit power. The maximum SAR10g is device dependent. Since
the deployment tool in this paper is designed voor outdoor usage connecting different unknown types of \gls{UE} a median of possible $SAR^{max}_{10g}$ phones should be calculated.
The actual $SAR^{max}_{1g}$ can't be used since femtocell are short range. Therefor \gls{UE} will seldom achieve it's maximum possible radiation. Just like \cite{J10_RDP}, a real  $SAR_{10g}$ should be predicted.